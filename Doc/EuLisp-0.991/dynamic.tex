\defModule{dynamic}{Dynamic Binding}
%
\begin{optDefinition}
%
The name of this module is {\tt dynamic}.
%
\specop{dynamic}
%
\Syntax
\defSyntax{dynamic}{
\begin{syntax}
    \scdef{dynamic-form}: \ra{} \classref{object} \\
    \>  ( \specopref{dynamic} \scref{identifier} )
\end{syntax}}%
\showSyntaxBox{dynamic}
%
\begin{arguments}
    \item[identifier] A symbol naming a dynamic binding.
\end{arguments}
%
\result%
The value of closest dynamic binding of \scref{identifier} is returned.  If no
visible binding exists, an error is signaled (condition:
\conditionref{unbound-dynamic-variable}
\indexcondition{unbound-dynamic-variable}).

\specop{dynamic-setq}
%
\Syntax
\defSyntax{dynamic-setq}{
\begin{syntax}
    \scdef{dynamic-setq-form}: \ra{} \classref{object} \\
    \>  ( \specopref{dynamic-setq} \scref{identifier} \scref{form} )
\end{syntax}}%
\showSyntaxBox{dynamic-setq}
%
\begin{arguments}
    \item[identifier] A symbol naming a dynamic binding to be updated.

    \item[form] An expression whose value will be stored in the dynamic binding
    of \scref{identifier}.
\end{arguments}
%
\result%
The value of \scref{form}.
%
\remarks%
The \scref{form} is evaluated and the result is stored in the closest dynamic
binding of \scref{identifier}.  If no visible binding exists, an error is
signalled (condition: \conditionref{unbound-dynamic-variable}
\indexcondition{unbound-dynamic-variable}).

\condition{unbound-dynamic-variable}{general-condition}
%
\begin{initoptions}
    \item[symbol, symbol] A symbol naming the (unbound) dynamic variable.
\end{initoptions}
%
\remarks%
Signalled by \specopref{dynamic} or \specopref{dynamic-setq} if the given
dynamic variable has no visible dynamic binding.

\specop{dynamic-let}
%
\Syntax
\defSyntax{dynamic-let}{
\begin{syntax}
    \scdef{dynamic-let-form}: \ra{} \classref{object} \\
    \>  ( \specopref{dynamic-let} \scseqref{binding} \\
    \>\>  \scref{body} )
\end{syntax}}%
\showSyntaxBox{dynamic-let}
%
\begin{arguments}
    \item[binding\/$^*$] A list of binding specifiers.

    \item[body] A sequence of forms.
\end{arguments}
%
\result%
The sequence of \scref{form}s is evaluated in order, returning the value of the
last one as the result of the \specopref{dynamic-let} form.
%
\remarks%
A binding specifier is either an identifier or a two element list of an
identifier and an initializing form.  All the initializing forms are evaluated
from left to right in the current environment and the new bindings for the
symbols named by the identifiers are created in the dynamic environment to hold
the results.  These bindings have dynamic scope and dynamic extent
\index{general}{scope and extent!of \specopref{dynamic-let} bindings}.  Each
form in \scref{body} is evaluated in order in the environment extended by the
above bindings.  The result of evaluating the last form in \scref{body} is
returned as the result of \specopref{dynamic-let}.

\defop{defglobal}
%
\Syntax
\defSyntax{defglobal}{
\begin{syntax}
    \scdef{defglobal-form}: \ra{} \classref{object} \\
    \> ( \defopref{defglobal} \scref{identifier} \scref{level-1-form} )
\end{syntax}}%
\showSyntaxBox{defglobal}
%
\begin{arguments}
    \item[identifier] A symbol naming a top dynamic binding containing the value
    of \scref{form}.

    \item[form] The \scref{form} whose value will be stored in the top dynamic
    binding of \scref{identifier}.
\end{arguments}
%
\remarks%
The value of \scref{form} is stored as the top dynamic value of the symbol named
by \scref{identifier} \index{general}{binding!top dynamic}.  The binding created
by \defopref{defglobal} is mutable.  An error is signaled (condition:
\conditionref{dynamic-multiply-defined}
\indexcondition{dynamic-multiply-defined}), on processing this form more than
once for the same \scref{identifier}.
%
\begin{note}
    The problems engendered by cross-module reference necessitated by a single
    top-dynamic environment are leading to a reconsideration of the defined
    model.  Another unpleasant aspect of the current model is that it is not
    clear how to address the issue of importing (or hiding) dynamic
    variables---they are in every sense global, which conflicts with the
    principle of module abstraction.  A model, in which a separate top-dynamic
    environment is associated with each module is under consideration for a
    later version of the definition.
\end{note}

\condition{dynamic-multiply-defined}{general-condition}
%
\begin{initoptions}
    \item[symbol, symbol] A symbol naming the dynamic variable which has already
    been defined.
\end{initoptions}
%
\remarks%
Signalled by \defopref{defglobal} if the named dynamic variable already
exists.
\end{optDefinition}
