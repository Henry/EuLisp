\defModule{mathlib}{Elementary Functions}
%
\begin{optPrivate}
APL is probably the source for these definitions, which have been
subsequently adapted for Lisp.  More detail is required.

910520--Improved matters some by simplifying some of the stuff from
CLtl2.  Also checked out Fortran-90 but that is under-specified for my
liking.
\end{optPrivate}
\begin{optDefinition}
The defined name of this module is {\tt mathlib}.  The functionality defined for
this module is intentionally precisely that of the trigonmetric functions,
hyperbolic functions, exponential and logarithmic functions and power functions
defined for {\tt <math.h>} in ISO/IEC 9899~:~1990 with the exceptions of {\tt
    frexp}, {\tt ldexp} and {\tt modf}.

\constant{pi}{double-float}
%
\remarks%
The value of \constantref{pi} is the ratio the circumference of a circle to its
diameter stored to double precision floating point accuracy.

\generic{acos}
%
\begin{genericargs}
    \item[float, \classref{float}] A floating point number.
\end{genericargs}
%
\result%
Computes the principal value of the arc cosine of {\em float} which is a value
in the range $[0,\pi]$ radians.  An error is signalled (condition-class:
\conditionref{domain-condition}\indexcondition{domain-condition}) if {\em float}
is not in the range $[-1,+1]$.

\generic{asin}
%
\begin{genericargs}
    \item[float, \classref{float}] A floating point number.
\end{genericargs}
%
\result%
Computes the principal value of the arc sine of {\em float} which is a value in
the range $[-\pi/2,+\pi/2]$ radians.  An error is signalled (condition-class:
\conditionref{domain-condition}\indexcondition{domain-condition}) if {\em float}
is not in the range $[-1,+1]$.

\generic{atan}
%
\begin{genericargs}
    \item[float, \classref{float}] A floating point number.
\end{genericargs}
%
\result%
Computes the principal value of the arc tangent of {\em float}
which is a value in the range $[-\pi/2,+\pi/2]$ radians.

\generic{atan2}
%
\begin{genericargs}
    \item[float$_1$, \classref{float}] A floating point number.
    \item[float$_2$, \classref{float}] A floating point number.
\end{genericargs}
%
\result%
Computes the principal value of the arc tangent of {\em float$_1$}/{\em
    float$_2$}, which is a value in the range $[-\pi,+\pi]$ radians, using the
signs of both arguments to determine the quadrant of the result.  An error might
be signalled (condition-class:
\conditionref{domain-condition}\indexcondition{domain-condition}) if either {\em
    float$_1$} or {\em float$_2$} is zero.

\generic{cos}
%
\begin{genericargs}
    \item[float, \classref{float}] A floating point number.
\end{genericargs}
%
\result%
Computes the cosine of {\em float} (measured in radians).

\generic{sin}
%
\begin{genericargs}
    \item[float, \classref{float}] A floating point number.
\end{genericargs}
%
\result%
Computes the sine of {\em float} (measured in radians).

\generic{tan}
%
\begin{genericargs}
    \item[float, \classref{float}] A floating point number.
\end{genericargs}
%
\result%
Computes the tangent of {\em float} (measured in radians).

\generic{cosh}
%
\begin{genericargs}
    \item[float, \classref{float}] A floating point number.
\end{genericargs}
%
\result%
Computes the hyperbolic cosine of {\em float}.  An error might be signalled
(condition class:
\conditionref{range-condition}\indexcondition{range-condition}) if the magnitude
of {\em float} is too large.

\generic{sinh}
%
\begin{genericargs}
    \item[float, \classref{float}] A floating point number.
\end{genericargs}
%
\result%
Computes the hyperbolic sine of {\em float}.  An error might be signalled
(condition class:
\conditionref{range-condition}\indexcondition{range-condition}) if the magnitude
of {\em float} is too large.

\generic{tanh}
%
\begin{genericargs}
    \item[float, \classref{float}] A floating point number.
\end{genericargs}
%
\result%
Computes the hyperbolic tangent of {\em float}.

\generic{exp}
%
\begin{genericargs}
    \item[float, \classref{float}] A floating point number.
\end{genericargs}
%
\result%
Computes the exponential function of {\em float}.  An error might be signalled
(condition class:
\conditionref{range-condition}\indexcondition{range-condition}) if the magnitude
of {\em float} is too large.

\generic{log}
%
\begin{genericargs}
    \item[float, \classref{float}] A floating point number.
\end{genericargs}
%
\result%
Computes the natural logarithm of {\em float}.  An error is signalled (condition
class: \conditionref{domain-condition}\indexcondition{domain-condition}) if {\em
    float} is negative. An error might be signalled (condition class:
\conditionref{range-condition}\indexcondition{range-condition}) if {\em float}
is zero.

\generic{log10}
%
\begin{genericargs}
    \item[float, \classref{float}] A floating point number.
\end{genericargs}
%
\result%
Computes the base-ten logarithm of {\em float}.  An error is signalled
(condition class:
\conditionref{domain-condition}\indexcondition{domain-condition}) if {\em float}
is negative. An error might be signalled (condition class:
\conditionref{range-condition}\indexcondition{range-condition}) if {\em float}
is zero.

\generic{pow}
%
\begin{genericargs}
    \item[float$_1$, \classref{float}] A floating point number.
    \item[float$_2$, \classref{float}] A floating point number.
\end{genericargs}
%
\result%
Computes {\em float$_1$} raised to the power {\em float$_2$}.  An error is
signalled (condition class:
\conditionref{domain-condition}\indexcondition{domain-condition}) if {\em
    float$_1$} is negative and {\em float$_2$} is not integral.  An error is
signalled (condition class: \conditionref{domain-condition}) if the result
cannot be represented when {\em float$_1$} is zero and {\em float$_2$} is less
than or equal to zero.  An error might be signalled (condition class:
\conditionref{range-condition}\index{general}{range-condition}) if the result
cannot be represented.

\generic{sqrt}
%
\begin{genericargs}
    \item[float, \classref{float}] A floating point number.
\end{genericargs}
%
\result%
Computes the non-negative square root of {\em float}.  An error is signalled
(condition class:
\conditionref{domain-condition}\indexcondition{domain-condition}) if {\em float}
is negative.
%
\end{optDefinition}
