\sclause{Formatted-IO}
\index{general}{formatted-io}
\index{general}{formatted-io!module}
\label{formatted-io}
\index{general}{level-0 modules!formatted-io}
%
\begin{optDefinition}
\noindent
The defined name of this module is {\tt formatted-io}.
%
\function{scan}
%
\begin{arguments}
    \item[format-string] A string containing format directives.
    \item[\optional{stream}] A stream from which input is to be taken.
\end{arguments}
%
\result%
Returns a list of the objects read from {\em stream}.
%
\remarks This function provides support for formatted input.  The {\em
    format-string\/} specifies reading directives, and inputs are matched
according to these directives.  An error is signaled (condition:
\conditionref{scan-mismatch}\indexcondition{scan-mismatch}) if the class of the
object read is not compatible with the specified directive.  The second
(optional) argument specifies a stream from which to take input.  If {\em
    stream\/} is not supplied, input is taken from \instanceref{stdin}.  Scan
returns a list of the objects read in.
%
\begin{description}
    \item[{\tt\textasciitilde a} any] any object.

    \item[{\tt\textasciitilde b} binary] an integer in binary format.

    \item[{\tt\textasciitilde c} character] a single character

    \item[{\tt\textasciitilde d} decimal] an integer decimal format.

    \item[{\tt\textasciitilde }\optional{{\em n}}{\tt e}] a exponential-format
    floating-point number.

    \item[{\tt\textasciitilde }\optional{{\em n}}{\tt f}] a fixed-format
    floating-point number.

    \item[{\tt\textasciitilde o} octal] an integer in octal format.

    \item[{\tt\textasciitilde r} radix] an integer in specified radix format.

    \item[{\tt\textasciitilde x} hexadecimal] an integer in hexadecimal format.

    \item[{\tt\textasciitilde \%} newline] matches a newline character in the
    input.
\end{description}

\condition{scan-mismatch}{stream-condition}
%
\begin{initoptions}
    \item[format-string, string] The value of this option is the format string
    that was passed to \functionref{scan}.
    \item[input, list] The value of this option is a list of the items read by
    \functionref{scan} up to and including the object that caused the condition
    to be signaled.
\end{initoptions}
%
\remarks%
This condition is signalled by \functionref{scan} if the format string does not
match the data input from {\em stream}.

\function{sformat}
%
\begin{arguments}
    \item[stream] A stream.
    \item[format-string] A string containing format directives.
    \item[\optional{object$_1$ \ldots}] A sequence of objects to be output
    on {\em stream}.
\end{arguments}
%
\result%
Returns {\em stream} and has the side-effect of outputting {\em object\/}s
according the formats specified in {\em format-string} to {\em stream}.
Characters are output as if the string were output by the \functionref{sprin}
function with the exception of those prefixed by {\em tilde\/}---graphic
representation {\tt\textasciitilde}---which are treated specially as detailed in
the following list.  These formatting directives are intentionally compatible
with the facilities defined for the function {\tt fprintf} in ISO/IEC
9899~:~1990 except for the prefix {\tt\textasciitilde} rather than {\tt\%}.
%
\begin{description}
    \item[{\tt\textasciitilde a} any]%
    uses \functionref{sprin} to output the argument.

    \item[{\tt\textasciitilde b} binary]%
    the argument must be an integer and is output in binary notation
    (syntax table~\ref{integer-syntax}).

    \item[{\tt\textasciitilde c} character]%
    the argument must be a character and is output using \functionref{write}
    (syntax table~\ref{character-syntax}).

    \item[{\tt\textasciitilde d} decimal]%
    the argument must be an integer and is output using \functionref{write}
    (syntax table~\ref{integer-syntax}).

    \item[{\tt\textasciitilde}\optional{{\em m}}\optional{{\tt.}{\em n}}{\tt e}
    exponential-format floating-point]%
    the argument must be a floating point number.  It is output in the style
    \mbox{\em\optional{-}d.ddd{\tt{e}}$\pm$dd}, in a field of width {\em m\/}
    characters, where there are {\em n\/} precision digits after the decimal
    point, or 6 digits, if {\em n\/} is not specified (syntax
    table~\ref{float-syntax}).  If the value to be output has fewer characters
    than {\em m\/} it is padded on the left with spaces.

    \item[{\tt\textasciitilde}\optional{{\em m}}\optional{{\tt.}{\em n}}{\tt f}
    fixed-format floating-point]%
    the argument must be a floating point number.  It is output in the style
    \mbox{\em\optional{-}ddd.ddd}, in a field of width {\em m\/} characters,
    where the are {\em n\/} precision digits after the decimal point, or 6
    digits, if {\em n\/} is not specified (syntax table~\ref{float-syntax}).
    The value is rounded to the appropriate number of digits.  If the value to
    be output has fewer characters than {\em m\/} it is padded on the left with
    spaces.

    \item[{\tt\textasciitilde}\optional{{\em m}}\optional{{\tt.}{\em n}}{\tt g}
    generalized floating-point]%
    the argument must be a floating point number.  It is output in either
    fixed-format or exponential notation as appropriate
    (syntax table~\ref{float-syntax}).

    \item[{\tt\textasciitilde o} octal]%
    the argument must be an integer and is output in octal notation
    (syntax table~\ref{integer-syntax}).

    \item[{\tt\textasciitilde}{\em n}{\tt r} radix]%
    the argument must be an integer and is output in radix {\em n\/} notation
    (syntax table~\ref{integer-syntax}).

    \item[{\tt\textasciitilde s} s-expression]%
    uses \functionref{write} to output the argument (syntax
    table~\ref{object-syntax}).

    \item[{\tt\textasciitilde}\optional{\em{n}}\true\/ tab]%
    output sufficient spaces to reach the next tab-stop, if {\em n\/} is not
    specified, or the {\em n$^{th}$} tab stop if it is.

    \item[{\tt\textasciitilde x} hexadecimal]%
    the argument must be an integer and is output in hexadecimal notation
    (syntax table~\ref{integer-syntax}).

    \item[{\tt\textasciitilde \%} newline]%
    output a newline character.

    \item[{\tt\textasciitilde \&} conditional newline]%
    output a newline character using, if it cannot be determined that the output
    stream is at the beginning of a fresh line.

    \item[{\tt\textasciitilde \textasciitilde} tilde]%
    output a tilde character using \functionref{sprin}.
\end{description}

\function{format}
%
\begin{arguments}
    \item[format-string] A string containing format directives.
    \item[\optional{object$_1$ \ldots}] A sequence of objects to be output
    on \instanceref{stdout}.
\end{arguments}
\result%
Returns \instanceref{stdout} and has the side-effect of outputting {\em
    object\/}s according the formats specified in {\em format-string} (see
\functionref{format}) to \instanceref{stdout}.

\function{fmt}
%
\begin{arguments}
    \item[format-string] A string containing format directives.
    \item[\optional{object$_1$ \ldots}] A sequence of objects to be formatted
    into a string.
\end{arguments}
%
\remarks%
Return the string created by formatting the {\em object\/}s according the
formats specified in {\em format-string} (see \functionref{format}).

\end{optDefinition}
