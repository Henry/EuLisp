% -----------------------------------------------------------------------
%%% Programming Language EuLisp Version 0.991
\documentclass[twocolumn,wd,9pt]{isov2}

\setlength\textwidth{180mm}
\setlength\oddsidemargin{-15mm}
\setlength\evensidemargin{-5mm}

\usepackage{etex}
\usepackage{makeidx}
\usepackage{multind}
\usepackage{url}

\ifpdf
    %\pdfoutput=0
    \usepackage[
        plainpages=false,
        pdfpagelabels,
        bookmarksnumbered,
        hyperindex=true
     ]{hyperref}

     \usepackage[usenames,dvipsnames]{color}
     %\definecolor{linkcolor}{rgb}{1,0,0}

    \hypersetup{
        bookmarks=true,         % show bookmarks bar?
        unicode=false,          % non-Latin characters in Acrobat’s bookmarks
        pdftoolbar=true,        % show Acrobat’s toolbar?
        pdfmenubar=true,        % show Acrobat’s menu?
        pdffitwindow=false,     % window fit to page when opened
        pdfstartview={FitH},    % fits the width of the page to the window
        pdftitle={My title},    % title
        pdfauthor={Author},     % author
        pdfsubject={Subject},   % subject of the document
        pdfcreator={Creator},   % creator of the document
        pdfproducer={Producer}, % producer of the document
        pdfkeywords={keywords}, % list of keywords
        pdfnewwindow=true,      % links in new window
        colorlinks=true,        % false: boxed links; true: colored links
        linkcolor=red,          % color of internal links
        citecolor=green,        % color of links to bibliography
        filecolor=magenta,      % color of file links
        urlcolor=cyan           % color of external links
    }
\fi

\usepackage{eulisp}
\includecomment{optDefinition}
% \includecomment{optRationale}
% \includecomment{optPrivate}

\standard{Programming Language EuLisp}
\yearofedition{2010}
\languageofedition{(E)}
\newcommand{\version}{0.991}
\renewcommand{\extrahead}{Version \version}

\setcounter{tocdepth}{3} % add more levels to table of contents

% ensure generation of indexes
\makeindex{module}
\makeindex{class}
\makeindex{special}
\makeindex{function}
\makeindex{generic}
\makeindex{condition}
\makeindex{constant}
\makeindex{general}

% -----------------------------------------------------------------------
%%% Begin document
\begin{document}

% -----------------------------------------------------------------------
%%% Cover page
\begin{cover}
    \vspace*{\fill}
    \hrule
    \vspace{0.3cm}
    {\Huge \thestandard} \\
    \vspace{0.3cm}
    {\LARGE Version \version} \\
    \ISname \\
    \vspace{0.5cm}
    \hrule
    \clearpage
\end{cover}

\begin{foreword}
\label{history}
\begin{optDefinition}
The \eulisp\ group first met in September 1985 at IRCAM in Paris to discuss the
idea of a new dialect of Lisp, which should be less constrained by the past than
Common Lisp and less minimalist than Scheme.  Subsequent meetings formulated the
view of \eulisp\ that was presented at the 1986 ACM Conference on Lisp and
Functional Programming held at MIT, Cambridge, Massachusetts \bref{desiderata}
and at the European Conference on Artificial Intelligence (ECAI-86) held in
Brighton, Sussex \bref{stoyan}.  Since then, progress has not been steady, but
happening as various people had sufficient time and energy to develop part of
the language.  Consequently, although the vision of the language has in the most
part been shared over this period, only certain parts were turned into physical
descriptions and implementations.  For a nine month period starting in January
1989, through the support of INRIA, it became possible to start writing the
\eulisp\ definition.  Since then, affairs have returned to their previous state,
but with the evolution of the implementations of \eulisp\ and the background of
the foundations laid by the INRIA-supported work, there is convergence to a
consistent and practical definition.
\end{optDefinition}

\begin{optDefinition}
The acknowledgments for this definition fall into three categories:
intellectual, personal, and financial.

The ancestors of \eulisp\ (in alphabetical order) are
\cl\index{general}{cl@\cl}~\bref{CLtl},
\interlisp\index{general}{interlisp@\interlisp}~\bref{interlisp},
\lelisp\index{general}{lelisp@\lelisp}~\bref{le-lisp},
\lisp/VM\index{general}{lispVM@\lisp/VM}~\bref{lisp/vm}, Scheme~\bref{scheme-3},
and T\index{general}{T}~\bref{t-manual} \bref{t-book}.  Thus, the authors of
this report are pleased to acknowledge both the authors of the manuals and
definitions of the above languages and the many who have dissected and extended
those languages in individual papers.  The various papers on Standard
ML\index{general}{Standard ML}~\bref{std-ml} and the draft report on
Haskell\index{general}{Haskell}~\bref{haskell} have also provided much useful
input.

The writing of this report has, at various stages, been supported by Bull S.A.,
Gesellschaft f\"ur Mathematik und Datenverarbeitung (GMD, Sankt Augustin), Ecole
Polytechnique (LIX), ILOG S.A., Institut National de Recherche en Informatique
et en Automatique (INRIA), University of Bath, and Universit\'e Paris VI (LITP).
The authors gratefully acknowledge this support.  Many people from European
Community countries have attended and contributed to \eulisp\ meetings since
they started, and the authors would like to thank all those who have helped in
the development of the language.

In the beginning, the work of the \eulisp\ group was supported by the
institutions or companies where the participants worked, but in 1987 DG XIII
(Information technology directorate) of the Commission of the European
Communities agreed to support the continuation of the working group by funding
meetings and providing places to meet.  It can honestly be said that without
this support \eulisp\ would not have reached its present state.  In addition,
the \eulisp\ group is grateful for the support of:

British Council in France (Alliance programme),
British Council in Spain (Acciones Integradas programme),
British Council in Germany (Academic Research Collaboration programme),
British Standards Institute,
Deutscher Akademischer Austauschdienst (DAAD),
Departament de Llenguatges i Sistemes Inform\`atics (LSI, Universitat
Polit\`ecnica de Catalunya),
Fraunhofer Gesellschaft Institut f\"ur Software und Systemtechnik,
Gesellschaft f\"ur Mathematik und Datenverarbeitung (GMD),
ILOG S.A.,
Insiders GmbH,
Institut National de Recherche en Informatique et en Automatique (INRIA),
Institut de Recherche et de Coordination Acoustique Musique (IRCAM),
Ministerio de Educacion y Ciencia (MEC),
Rank Xerox France,
Science and Engineering Research Council (UK),
Siemens AG,
University of Bath,
University of Technology, Delft,
University of Edinburgh,
Universit\"at Erlangen
and
Universit\'e Paris VI (LITP).

The following people (in alphabetical order) have contributed in
various ways to the evolution of the language:
Giuseppe Attardi,
Javier B\'ejar,
Neil Berrington,
Russell Bradford,
Harry Bretthauer,
Peter Broadbery,
Christopher Burdorf,
J\'er\^ome Chailloux,
Odile Chenetier,
Thomas Christaller,
Jeff Dalton,
Klaus D\"a{\ss}ler,
Harley Davis,
David DeRoure,
John Fitch,
Richard Gabriel,
Brigitte Glas,
Nicolas Graube,
Dieter Kolb,
J\"urgen Kopp,
Antonio Moreno,
Eugen Neidl,
Greg Nuyens,
Pierre Parquier,
Keith Playford,
Willem van der Poel,
Christian Queinnec,
Nitsan Seniak,
Enric Sesa,
Herbert Stoyan,
and
Richard Tobin.

The editors of the \eulisp\ definition wish particularly to acknowledge the work
of Harley Davis on the first versions of the description of the object system.
The second version was largely the work of Harry Bretthauer, with the assistance
of J\"urgen Kopp, Harley Davis and Keith Playford.

Julian Padget ({\bf \tt jap@maths.bath.ac.uk})\\
School of Mathematical Sciences\\
University of Bath\\
Bath, Avon, BA2 7AY, UK\\

\noindent
Harry Bretthauer ({\bf \tt harry.bretthauer@gmd.de})
GMD mbH\\
Postfach 1316\\
53737 Sankt Augustin\\
Germany\\

\noindent
editors.
\end{optDefinition}
\end{foreword}


% -----------------------------------------------------------------------
%%% Title page
\title{}{\thestandard}{\extrahead}

% -----------------------------------------------------------------------
%%% Introduction
\begin{introduction}
\label{sec:intro}
\begin{optPrivate}
    The behaviour of syntax operators is never explained, especially with
    respect to how many times they might be expanded.

    The hierarchy of types (classes) is not specified anywhere.  The coercion
    ladder, if taken as part of the hierarchy, is upside down in some places (if
    the class systems is taken as prototypical of types).

    Nothing about the behaviour of \telos\ is specified.  How is the CPL
    calculated?  When are methods invoked?  What is method combination?  (Note:
    it seems someone else wrote this part.  It seems much less smooth than the
    rest).

    Threads are hardly discussed.  How can tail-recursion be specified
    linguistically (RPG).

    The next paragraph is somewhat awkward.  [jwd]

    Why are modules not first-class? (RPG).  Abstraction to optimization is a
    poor link (RPG)---action: remove (JAP).  The ...recursive
    behaviour... definition is incorrect (RPG).  Implicit here is that {\tt
        symbol-value} and {\tt dynamic} are disjoint (RPG).

    JWD: We ought to consider whether we really want to require tail recursion
    optimization.  It is somewhat difficult to define precisely what the
    requirement means (see mail on the Scheme Standard list), and it makes it
    harder to use certain implementation strategies (eg, compile to C).  On the
    other hand, we don't have any other means for expressing iteration and I
    suspect many of us are inclined to follow Scheme.  A Rationale entry is
    required, at least.

    JWD: The Rationale should explain why dynamic variables are not by-module.
    It is possible to implement dynamic variables so that different vars with
    the same name can exist in different modules.  We would have to have some
    way to refer to (or import) variables in another module.

    Rewrote (once again) the description of the binding environments and gave
    them new names.  This removes the false distinction between lexical and
    module bindings that people complained about at Dec '90 meeting.  Any
    further complaints about this must be accompanied by revised wording.
\end{optPrivate}

\begin{optDefinition}
\eulisp\ is a dialect of Lisp and as such owes much to the great body of work
that has been done on language design in the name of Lisp over the last forty
years.  The distinguishing features of \eulisp\ are (i) the integration of the
classical Lisp type system and the object system into a single class hierarchy
(ii) the complementary abstraction facilities provided by the class and the
module mechanism (iii) support for concurrent execution.

Here is a brief summary of the main features of the language.
\begin{itemize}

    \item Classes\index{general}{class} are first-class objects.  The class
    structure integrates the primitive classes\index{general}{class!primitive}
    describing fundamental datatypes, the predefined classes and user-defined
    classes.

    \item Modules\index{general}{module} together with classes are the building
    blocks of both the \eulisp\ language and of applications written in \eulisp.
    The module system exists to limit access to objects by name.  That is,
    modules allow for hidden definitions.  Each module defines a fresh, empty,
    lexical environment\index{general}{module!environments}.

    \item Multiple control threads can be created in \eulisp\ and the
    concurrency model has been designed to allow consistency across a wide range
    of architectures.  Access to shared data can be controlled via locks
    (semaphores). Event-based programming is supported through a generic waiting
    function.

    \item Both functions and continuations are first-class in \eulisp, but the
    latter are not as general as in Scheme because they can only be used in the
    dynamic extent of their creation.  That implies they can only be used once.

    \item A condition mechanism which is fully integrated with both classes and
    threads, allows for the definition of generic handlers and supports both
    propagation of conditions and continuable handling.

    \item Dynamically scoped bindings\index{general}{binding!dynamically scoped}
    can be created in \eulisp, but their use is restricted, as in Scheme.
    \eulisp\ enforces a strong distinction between lexical bindings and dynamic
    bindings by requiring the manipulation of the latter via special forms.

\end{itemize}
%
\eulisp\ does not claim any particular Lisp dialect as its closest relative,
although parts of it were influenced by features found in
\cl\index{general}{cl@\cl}, \interlisp\index{general}{interlisp@\interlisp},
\lelisp\index{general}{lelisp@\lelisp},
\lisp/VM\index{general}{lispVM@\lisp/VM}, Scheme, and
T\index{general}{T}. \eulisp\ both introduces new ideas and takes from these
Lisps.  It also extends or simplifies their ideas as seen fit.  But this is not
the place for a detailed language comparison.  That can be drawn from the rest
of this text.

\eulisp\ breaks with \lisp\ tradition in describing all its types (in fact,
classes) in terms of an object system.  This is called The \eulisp\ Object
System, or \telos. \telos\ incorporates elements of the Common Lisp Object
System (CLOS)\index{general}{CLOS}~\bref{clos},
ObjVLisp\index{general}{ObjVLisp}~\bref{objvlisp},
Oaklisp\index{general}{Oaklisp}~\bref{oaklisp},
MicroCeyx\index{general}{MicroCeyx}~\bref{lelisp-manual}, and
MCS\index{general}{MCS}~\bref{mcs}.
%
\end{optDefinition}

\end{introduction}


% -----------------------------------------------------------------------
%%% level-0 Syntax

\clause{Language Structure}
\label{subsec:lang-struct}
\index{general}{language structure}
%
\begin{optPrivate}
    Restriction on level-0 generics seems wierd.

    I [jwd] changed the paragraph on level-0 functions and level-0 applications
    to be, I think, clearer.  But we still don't say what a level-n function is
    for n > 0, we don't define level-n application for any n, and the claim that
    a level-n function is a level-(n-1) application may not be the right one.

    Someone queried use of define and proposed describe, but can't see how this
    helps.
\end{optPrivate}
%
\begin{optDefinition}
The \eulisp\ definition comprises the following items:
\begin{description}
    \item[Level-0]\index{general}{EuLisp@\eulisp!level-0} comprises all the
    level-0 classes, functions, macros and special forms\index{general}{special
        form}, which is this text minus Annex \ref{annex:level-1}.  The object
    system can be extended by user-defined structure classes, and generic
    functions.

    \item[Level-1]\index{general}{EuLisp@\eulisp!level-1} extends level-0 with
    the classes, functions, macros and special forms defined in Annex
    \ref{annex:level-1}.  The object system can be extended by user-defined
    classes and metaclasses.  The implementation of level-1 is not
    necessarily written or writable as a conforming level-0 program.
\end{description}
%
A {\em level-0 function\/} is a (generic) function defined in this text to be
part of a conforming processor for level-0.  A function defined in terms of
level-0 operations is an example of a {\em level-0 application}.

Much of the functionality of \eulisp\ is defined in terms of
modules\index{general}{standard
    module}\index{general}{EuLisp@\eulisp!libraries}.  These modules might be
available (and used) at any level, but certain modules are required at a given
level.  Whenever a module depends on the operations available at a given level,
that dependency will be specified.

\eulisp\ level-0\index{general}{level-0 modules} is provided by the module {\tt
    eulisp0}\index{general}{level-0 modules!eulisp0}.  This module imports and
re-exports the modules specified in table~\ref{level-0-modules}.

Modules comprising {\tt eulisp0}:
\begin{center}
    \label{level-0-modules}
    \begin{tabular}{|ll|}\hline
        Module & Section(s)\\\hline
        {\tt character} & \ref{character}\\
        {\tt collection} & \ref{collection}\\
        {\tt compare} & \ref{compare}\\
        {\tt condition} & \ref{condition}\\
        {\tt convert} & \ref{convert}\\
        {\tt copy} & \ref{copy}\\
        {\tt double} & \ref{double-float}\\
        {\tt fpi} & \ref{fpi}\\
        {\tt formatted-io} & \ref{formatted-io}\\
        {\tt function} & \ref{function}\\
        {\tt keyword} & \ref{keyword}\\
        {\tt list} & \ref{list}\\
        {\tt lock} & \ref{lock}\\
        {\tt mathlib} & \ref{elementary-functions}\\
        {\tt number} & \ref{number}\\
        {\tt telos0} & \ref{obj-0}\\
        {\tt stream} & \ref{stream}\\
        {\tt string} & \ref{string}\\
        {\tt symbol} & \ref{symbol}\\
        {\tt table} & \ref{table}\\
        {\tt thread} & \ref{thread}\\
        {\tt vector} & \ref{vector}\\\hline
    \end{tabular}
\end{center}
%
\noindent
This definition is organized in three parts:
\begin{description}
    %
    \item[\S \ref{syntax}--\ref{control-0}] describes the core of level-0
    of \eulisp, covering modules, simple classes, objects and generic functions,
    threads, conditions, control forms and events.  These sections contain the
    information about \eulisp\ that characterizes the language.
    %
    \item[Annex A] describes the additional classes required at level-0 and the
    operations defined on instances of those classes\index{general}{level-0
        classes}.  The annex is organized by module in alphabetical order.
    These sections contain information about the predefined classes in \eulisp\
    that are necessary to make the language usable, but is not central to an
    appreciation of the language.
    %
    \item[Annex B] describes the reflective aspects of the object system and how
    to program the metaobject protocol and some additional control forms.
\end{description}
%
Prior to these, sections \ref{scope}--\ref{sec:definitions} define the scope of
the text, cite normative references, conformance definitions, error definitions,
typographical and layout conventions and terminology definitions used in this
text.
\end{optDefinition}
%
\clause{Scope}\label{scope}
\begin{optDefinition}
This text specifies the syntax and semantics of the computer
programming language \eulisp\ by defining the requirements for a
conforming \eulisp\ processor and a conforming \eulisp\ program (the
textual representation of data and algorithms).
%
% {\em\begin{verse}
%         NOTE---the term ``program'' is avoided in this definition because of
%         the ambiguity in the representation of such an entity.
%     \end{verse}}
%
\noindent
This text does not specify:
\begin{enumerate}
    %
    \item The size or complexity of an \eulisp\ program that will exceed the
    capacity of any specific configuration or processor, nor the actions to be
    taken when those limits are exceeded.
    %
    \item The minimal requirements of a configuration that is capable of
    supporting an implementation of a \eulisp\ processor.
    %
    \item The method of preparation of a \eulisp\ program for execution or the
    method of activation of this \eulisp\ program once prepared.
    %
    \item The method of reporting errors, warnings or exceptions to the client
    of a \eulisp\ processor.
    %
    \item The typographical representation of a \eulisp\ program for human
    reading.
    %
    \item The means to map module names to the filing system or other object
    storage system attached to the processor.
\end{enumerate}
%
To clarify certain instances of the use of English in this text the following
definitions are provided:
\begin{description}
    \item[must] a verbal form used to introduce a {\em required} property.  All
    conforming processors must satisfy the property.
    %
    % \item[shall:] A verbal form used to introduce a {\em required}
    % property.  All conforming processors must satisfy the property.
    %
    \item[should] A verbal form used to introduce a {\em strongly recommended}
    property.  Implementors of processors are urged (but not required) to
    satisfy the property.
    %
\end{description}
\end{optDefinition}
%
\clause{Normative References}\index{general}{normativeReferences}
\begin{optDefinition}
The following standards contain provisions, which through references
in this text constitute provisions of this definition.  At the
time of writing, the editions indicated were valid.  All standards are
subject to revision and parties making use of this definition are
encouraged to apply the most recent edition of the standard listed
below.
\begin{references}
    %
    \bibitem[ISO 646] {iso646} Information processing --- ISO 7-bit coded
    character set for information interchange, 1983.  Note: this standard is
    currently under revision and the interested reader should see the 1990 Draft
    International Standard version of ISO/IEC 646.
    %
    \bibitem[ISO 2382] {iso2382} Data processing --- vocabulary.
    %
    \bibitem[ISO TR 10034 : 1990] {iso10034} Information technology ---
    Guidelines for the preparation of conformity clauses in programming language
    standards.
    %
    \bibitem[ISO TR 10176 : 1991] {iso10176} Information technology ---
    Guidelines for the preparation of programming language standards. Note: this
    is currently a draft technical report.
    %
    \bibitem[ISO/IEC 9899:1999] {iso9899} Programming Languages --- C.
    %
\end{references}
\end{optDefinition}
%
\clause{Conformance Definitions}\index{general}{conformance}
\label{subsubsec:cse}
\begin{optPrivate}
    JWD: What I think's happened is that a number of convenient definitions have
    been put it, but we haven't yet sorted out just which ones we want and (in
    some cases) just what they should mean.  Anyway, here are some questions:

    Why isn't "processor-undefined" just "undefined"?

    "Portable code" refers to "conforming code", but the latter is not defined.

    It seems that, by omission, "conforming programs" can depend on
    "processor-dependent" behaviour.

    If we're going to have both "portable code" and "conforming program", there
    ought to be some straightforward relationship between them.

    We need to relate the later parts of 2.3.1 to the earlier parts.  For
    example, what are the implications for conformance when something "is an
    error"?

    The first two paragraphs of [section 2.5] give two different definitions of
    "variable".  The second is the Scheme definition.

    The definition of "variable" in the description env isn't quite a
    definition.  It says "denotes a binding ...".

    "Extent" can be for things other than bindings.

    "Dynamic scope" is defined in terms of "control region" which is not
    defined.

    The order of definitions in the section 2.5 description env looks reasonable
    of you look just at the items, but I think it would be easier to read of the
    different kinds of "scope" ("extent") were defined together rather than the
    different applications of "indefinite" ("dynamic").

    Updated conformance section as per draft guidelines referenced in the text.

    910527---added Klaus' terminology with slight adaptations.
\end{optPrivate}
%
\begin{optDefinition}
The following terms are general in that they could be applied to the
definition of any programming language.  They are derived from ISO/IEC
TR 10034: 1990.
%
\begin{definitions}
    \definition{configuration}\index{general}{configuration} Host and target
    computers, any operating systems(s) and software (run-time system) used to
    operate a language {\em processor}.
    %
    \definition{conformity clause}\index{general}{conformity clause} Statement
    that is not part of the language definition but that specifies requirements
    for compliance with the language standard.
    %
    \definition{conforming program}\index{general}{conforming program} Program
    which is written in the language defined by the language standard and which
    obeys all the {\em conformity} {\em clauses} for programs in the language
    standard.
    %
    \definition{conforming processor}\index{general}{conforming processor} {\em
        Processor} which processes {\em conforming} {\em programs} and program
    units and which obeys all the {\em conformity} {\em clauses} for {\em
        processors} in the language standard.
    %
    \definition{error}\index{general}{error} Incorrect program construct or
    incorrect functioning of a program as defined by the language standard.
    %
    \definition{extension}\index{general}{extension} Facility in the {\em
        processor} that is not specified in the language standard but that does
    not cause any ambiguity or contradiction when added to the language
    standard.
    %
    \definition{implementation-defined}\index{general}{implementation-defined}
    Specific to the {\em processor}, but required by the language standard to be
    defined and documented by the implementer.
    %
    \definition{processor}\index{general}{processor} Compiler, translator or
    interpreter working in combination with a {\em configuration}.
    %
\end{definitions}
\end{optDefinition}
%
\clause{Error Definitions}\index{general}{errors}
\begin{optDefinition}
Errors in the language described in this definition fall into one of
the following three classes:
%
\begin{definitions}
%
    \definition{static error}\index{general}{static
        error}\index{general}{error!static} An error which is detected during
    the execution of a \eulisp\ program or which is a violation of the defined
    behaviour of \eulisp.  Static errors have two classifications:
    \begin{enumerate}
        \item Where a {\em conforming} {\em processor} is required to detect the
        erroneous situation or behaviour and report it.  This is signified by
        the phrase {\em an error is signalled}\index{general}{error!signalled}.
        The condition class to be signalled is specified.  Signalling an error
        consists of identifying the condition class related to the error and
        allocating an instance of it.  This instance is initialized with
        information dependent on the condition class.  A conforming \eulisp\
        program can rely on the fact that this condition will be signalled.

        \item Where a {\em conforming} {\em processor} might or might not detect
        and report upon the error.  This is signified by the phrase {\em \ldots
            is an error}.  A processor should provide a mode where such errors
        are detected and reported where possible\index{general}{error!can be
            signalled}.
    \end{enumerate}
    If the result of preparation is a runnable program, then that program must
    signal any static error.

    \definition{environmental error}\index{general}{environmental
        error}\index{general}{error!environmental} An error which is detected by
    the configuration supporting the \eulisp\ processor.  The processor must
    signal the corresponding static error which is identified and handled as
    described above.

    \definition{violation}\index{general}{error}\index{general}{violation}
    \index{general}{error!violation} An error which is detected during the
    preparation of a \eulisp\ program for execution, such as a violation of the
    syntax or static semantics of \eulisp\ in the program under preparation.  A
    {\em conforming} {\em processor} is required to issue a diagnostic if a
    violation is detected.
\end{definitions}

All errors specified in this definition are static unless explicitly stated
otherwise.
\end{optDefinition}
%
\clause{Compliance}\index{general}{compliance}
\begin{optDefinition}
An \eulisp\ processor can conform\index{general}{compliance} at either of the
two levels defined under Language Structure in the Introduction.
Thus a level-0\index{general}{conformance!level-0} conforming processor
must support all the basic expressions, classes and class operations
defined at level-0.  A level-1\index{general}{conformance!level-1}
conforming processor must support all the basic expressions, classes,
class operations and modules defined at level-0 and at level-1.

The following two statements govern the conformance of a processor at a given
level.
\begin{enumerate}
    %
    \item A {\em conforming processor\/} must correctly process all programs
    conforming both to the standard at the specified level and the {\em
        implementation-defined} features of the {\em processor}.
    %
    \item A {\em conforming processor\/} should offer a facility to report the
    use of an {\em extension} which is statically determinable solely from
    inspection of a program, without execution.  (It is also considered
    desirable that a facility to report the use of an {\em extension} which is
    only determinable dynamically be offered.)
    %
\end{enumerate}
%
A level-0\index{general}{conformance!level-0} conforming program is one which
observes the syntax and semantics defined for level-0.  A level-0 conforming
program might not conform at level-1.  A {\em strictly-conforming\/} level-0
program is one that also conforms at level-1.  A level-1 conforming program
observes the syntax and semantics defined for level-1.

In addition, a {\em conforming program\/} at any level must not use any {\em
    extensions\/} implemented by a language {\em processor\/}, but it can rely
on {\em implementation-defined\/} features.

\noindent
The documentation of a {\em conforming processor\/} must include:
\begin{enumerate}
    %
    \item A list of all the {\em implementation-defined\/} definitions or
    values.
    %
    \item A list of all the features of the language standard which are
    dependent on the {\em processor\/} and not implemented by this {\em
        processor\/} due to non-support of a particular facility, where such
    non-support is permitted by the standard.
    %
    \item A list of all the features of the language implemented by this {\em
        processor\/} which are {\em extensions\/} to the standard language.
    %
    \item A statement of conformity, giving the complete reference of the
    language standard with which conformity is claimed, and, if appropriate, the
    level of the language supported by this processor.
    %
\end{enumerate}
\end{optDefinition}
%
\clause{Conventions}\index{general}{conventions}
\label{conventions}
\begin{optPrivate}
    Should note error behaviour in reference to constant module binding (RPG).
    Reorganize the naming convention stuff into a list layout to make it easier
    to read---done (JAP).  What is the original value that is restored on exit?
    (RPG).  Various undefined terms are: object, class, metaclass, function,
    control region, closure, continuation, closer lexically, closer dynamically.
    Note: missing sample names for ``c'' and ``q'' suffixes.

    Someone proposed should define notation for rest lists here, but since they
    do not (and cannot?) occur in the definition, I have not done anything about
    it.
\end{optPrivate}
%
\begin{optDefinition}
This section defines the conventions employed in this text, how
definitions will be laid out, the typefaces to be used, the
meta-language used in descriptions and the naming conventions.  A
later section (\ref{sec:definitions}) contains definitions of the
terms used in this text.

A standard function\index{general}{function!standard function} denotes an
immutable top-lexical binding of the defined name.  All the definitions in this
text are bindings in some module except for the special form operators, which
have no definition anywhere.  All bindings and all the special form operators
can be renamed.
%
\begin{note}
    A description making mention of ``an x'' where ``x'' is the name a
    class usually means ``an instance of {\tt <x>}''.
\end{note}
%
Frequently, a class-descriptive name will be used in the argument list of a
function description to indicate a restriction on the domain to which that
argument belongs.  In the case of a function, it is an error to call it with a
value outside the specified domain.  A generic function can be defined with a
particular domain and/or range.  In this case, any new methods must respect the
domain and/or range of the generic function to which they are to be attached.
The use of a class-descriptive name in the context of a generic function
definition defines the intention of the definition, and is not necessarily a
policed restriction.

The {\em result-class} of an operation, except in one case, refers to a
primitive or a defined class described in this definition.  The exception is for
predicates.  Predicates are defined to return either the empty list---written
\nil---representing the boolean value false\index{general}{false}, or any
value other than \nil, representing true\index{general}{true}.  Although the
class containing exactly this set of values is not defined in the language,
notation is abused for convenience and {\em boolean}\index{general}{boolean} is
defined, for the purposes of this report, to mean that set of values.  If the
true value is a useful value, it is specified precisely in the description of
the function and written as \true.
\end{optDefinition}
%
\sclause{Layout and Typography}
\begin{optDefinition}
Both layout and fonts are used to help in the description of \eulisp.  A
language element is defined as an entry with its name as the heading of a
clause, coupled with its classification.  The syntax notation used is based on
that described in \cite{iso9899} with modifications to support the specification
of a return type and to improve clarity.  Syntactic categories (non-terminals)
are indicated by {\it italic} type, and literal words and characters (terminals)
by {\tt constant width} type.  A colon ({\it :}) following a non-terminal
introduces its definition.  Alternative definitions are listed on separate
lines, except when prefaced by the words ``one of''.  An optional symbol is
indicated by the subscript ``{\it opt}'', a list of zero or more occurrences of
a symbol are indicated by the superscript ``{\it *}'', and a list of one or more
occurrences of a symbol are indicated by the superscript ``{\it +}''.  Examples
of several kinds of entry are now given.  Some subsections of entries are
optional and are only given where it is felt necessary.
%
\keyword{a-special-form}
%
\Syntax
\defSyntax{a-special-form}{
\begin{syntaxx}
    \scdef{a-special-form-form}: \ra \sc{result-class} \\
    \>  ( \keywordref{a-special-form} \sc{form-1} ... \scopt{form-n} )
\end{syntaxx}}
\showSyntaxBox{a-special-form}
%
\begin{arguments}
    \item[\sc{form-1}] description of structure and r\^ole of \sc{form-1}.\\
    $\vdots$
    \item[\scopt{form-n}] description of structure and r\^ole of optional
    argument \scopt{form-n}.
\end{arguments}
%
\result%
A description of the result and, possibly, its \sc{result-class}.
%
\remarks%
Any additional information defining the behaviour of {\tt a-special-form} or the
syntax category \scref{a-special-form-form}.
%
\examples
Some examples of use of the special form and the behaviour that should
result.
%
\seealso%
Cross references to related entries.
%
\function{a-function}
%
\Signature
\defSignature{a-function}{
\begin{signature}
    (\functionref{a-function} \sc{argument-1} ... \scopt{argument-n})\\
    \>  \ra \sc{result-class}
\end{signature}}
\showSignatureBox{a-function}
%
\begin{arguments}
    \item[\sc{argument-1}] information about the class or classes of
    \sc{argument-1}.\\
    $\vdots$
    \item[\scopt{argument-n}] information about the class or classes of
    the optional argument \sc{argument-n}.
\end{arguments}
%
\result%
A description of the result and, possibly, its \sc{result-class}.
%
\remarks%
Any additional information about the actions of {\tt a-function}.
%
\examples
Some examples of calling the function with certain arguments and the
result that should be returned.
%
\seealso%
Cross references to related entries.
%
\generic{a-generic}
%
\begin{genericargs}
    \item[argument-a, <class-a>] means that {\em argument-a} of {\tt a-generic}
    must be an instance of {\tt <class-a>} and that {\em argument-a} is one of
    the arguments on which {\tt a-generic} specializes.  Furthermore, each
    method defined on {\tt a-generic} may
    specialize only on a subclass of {\tt <class-a>} for {\em argument-a}.\\
    $\vdots$
    \item[argument-n] means that (i) {\em argument-n} is an instance of {\tt
        <object>}, {\em i.e.} it is unconstrained, (ii) {\tt a-generic} does not
    specialize on {\em argument-n}, (iii) no method on {\tt a-generic} can
    specialize on {\em argument-n}.
\end{genericargs}
%
\result%
A description of the result and, possibly, its class.
%
\remarks%
Any additional information about the actions of {\tt a-generic}.  This
can take two forms depending on the function.  This will probably be
in general terms, since the actual behaviour will be determined by the
methods.
%
\seealso%
Cross references to related entries.
%
\method{a-generic}
%
(A method on {\tt a-generic} with the following specialized arguments.)
%
\begin{specargs}
    \item[argument-a, <class-a>] means that {\em argument-a} of the method must
    be an instance of {\tt <class-a>}.  Of course, this argument must be one
    which was defined in {\tt a-generic} as being open to
    specialization and {\tt <class-a>} must be a subclass of the class.\\
    $\vdots$
    \item[argument-n] means that (i) {\em argument-n} is an instance of {\tt
        <object>}, {\em i.e.} it is unconstrained, because {\tt a-generic} does
    not specialize on {\em argument-n}.
\end{specargs}
%
\result%
A description of the result and, possibly, its class.
%
\remarks%
Any additional information about the actions of this method attached
to {\tt a-generic}.
%
\seealso%
Cross references to related entries.
%
\condition{a-condition}{condition}
%
\begin{initoptions}
    \item[initarg-a, value-a] means that an instance of {\tt <a-condition>} has
    a slot called {\tt initarg-a} which should be initialized to {\em value-a},
    where {\em value-a} is often the name of a class, indicating that {\em
        value-a} should be an instance of that class and a description of the
    information that {\em value-a} is
    supposed to provide about the exceptional situation that has arisen.\\
    $\vdots$
    \item[initarg-n, value-n] As for {\tt initarg-a}.
\end{initoptions}
%
\remarks%
Any additional information about the circumstances in which the
condition will be signalled.
%
\derivedclass{a-class}{class}
%
\begin{initoptions}
    \item[initarg-a, value-a] means that {\tt <a-class>} has an
    initarg whose name is {\tt initarg-a} and the description will usually
    say of what class (or classes) {\em value-a} should be an instance.
    This initarg is required.\\ $\vdots$
    \item[{\tt[}initarg-n, value-n{\tt]}]
    The enclosing square brackets denote that this initarg is optional.
    Otherwise the interpretation of the definition is as for {\tt
        initarg-a}.
\end{initoptions}
%
\remarks%
A description of the r\^ole of {\tt <a-class>}.

\constant{a-constant}{a-class}
%
\remarks%
A description of the constant of type \classref{a-class}.

\constant{<a-class-alias>}{a-class}
%
\remarks%
A description of the r\^ole of the constant binding whose value is
\classref{a-class}, {\em i.e.} the class alias.

\end{optDefinition}
%
\sclause{Naming}
%
\begin{optPrivate}
    Consider adopting Scheme conventions for destructive and predicate
    functions.
\end{optPrivate}
%
\begin{optDefinition}
Naming conventions are applied in the descriptions of primitive and
defined classes as well as in the choice of other function names.
Here is a list of the conventions and some examples of their use.
%
\subdefinition{``{\tt <name>}'' wrapping} By convention, classes have names
which begin with ``{\tt <}'' and end with ``{\tt >}''.
%
\subdefinition{``{\tt binary}'' prefix} The two argument version of a n-ary
argument function.  For example \genericref{binary+} is the two argument
(generic) function corresponding to the n-ary argument \functionref{+} function.
%
% \subdefinition{``{\tt c}'' suffix}
% The version of the function named {\tt foo} that is
% guaranteed to terminate on self-referential structures is usually
% named {\tt fooc}.
%
\subdefinition{``{\tt -class}'' suffix} The name of a metaclass of a set of
related classes.  For example, \classref{function-class}, which is the class of
\classref{simple-function}, \classref{generic-function} and any of their
subclasses.  The exception is \classref{class} itself which is the default
metaclass.  The prefix should describe the general domain of the classes in
question, but not necessarily any particular class in the set.
%
\subdefinition{``{\tt generic-}'' prefix} The generic version of the function
named by the stem.
% ---usually
% required to allow for optional or variable numbers of arguments to
% {\tt abc}.
%
\subdefinition{hyphenation} Function and class names made up of more than one
word are hyphenated, for example: {\tt
    compute-specialized-slot-class}.
%
% \subdefinition{``{\tt make-}'' prefix}
% For most primitive or defined classes there is constructor function,
% which is usually named {\tt make-}{\em class-name}---except where
% historical precedent is strong, for example, {\tt cons} is used in
% preference to {\tt make-pair}.
%
% \subdefinition{``{\tt n}'' prefix}
% The destructive version of the function named {\tt foo} is usually
% named {\tt nfoo}, for example the destructive version of {\tt reverse}
% is named {\tt nreverse}.
%
\subdefinition{``{\tt p}'' suffix} A predicate function is named by a ``{\tt
    p}'' suffix if the function or class name (after removing the enclosing {\tt
    <} and {\tt >}) is not hyphenated, for instance, \functionref{consp}, and is
named by a ``{\tt -p}'' suffix if it is, for instance
\functionref{double-float-p}.
%
% \subdefinition{``{\tt q}'' suffix}
% The version of the function named {\tt foo} that uses {\tt eq} for
% comparison is usually named {\tt fooq}.
%
% \subdefinition{``{\tt -ref}'' suffix}
% A slot reader is named {\em class-name}{\tt{}-ref}---where
% appropriate---and as is the corresponding slot writer {\tt (setter
%     {\em class-name}-ref)}---also where appropriate, for example {\tt
%     table-ref}.  This convention is broken by historical precedent for the
% accessors to slots of pairs, which are called {\tt car} and {\tt cdr}.
%
% \subdefinition{``{\tt -using-}'' infix}
% This function name convention is used when calls are cascaded using
% objects derived from the original object so that users can write
% methods on the relevant classes.  For example, in the case of {\tt
%     slot-value}, the class and slot description corresponding to the
% original object and slot name are retrieved, and the new generic
% functions are called ({\tt slot-value-using-class} and {\tt
%     slot-value-using-slot}).
%
% When an operation depends on a global value, such as the current input
% stream, the value might also be made accessible via a function, for
% example {\tt (a-global-value)} and can be updated using the
% corresponding setter function, for example {\tt ((setter
%     a-global-value) another-value)}.
%
\end{optDefinition}

\clause{Definitions}
\begin{optPrivate}
\end{optPrivate}
\begin{optDefinition}
\label{sec:definitions}
This set of definitions, which are be used throughout this document, is
self-consistent but might not agree with notions accepted in other language
definitions.  The terms are defined in alphabetical rather than dependency order
and where a definition uses a term defined elsewhere in this section it is
written in italics.  Names in {\tt courier} font refer to entities defined in
the language.
%
\begin{definitions}
    \definition{abstract class}\index{general}{abstract class}
    A Class that by definition has no direct instances.

    \definition{applicable method} \index{general}{applicable method}
    \index{general}{method!applicable} A {\em method} is applicable for a
    particular set of arguments if each element in its {\em domain} is a {\em
        superclass} of the {\em class} of the corresponding argument.

    \definition{binding}\index{general}{binding}
    A location containing a value.

    \definition{boolean}\index{general}{boolean} A boolean value is either {\em
        false}\index{general}{false}, which is represented by the empty
    list---written \nil\/ and is also the value of {\tt nil}---or {\em
        true}\index{general}{true}, which is represented by any other value than
    \nil or if specified as \true.

    \definition{class}\index{general}{class} A class is an {\em object} which
    describes the structure and behaviour of a set of {\em objects} which are
    its {\em instances}.  A {\em class} object contains {\em inheritance}
    information and a set of {\em slot descriptions} which define the structure
    of its {\em instances}.  A {\em class} {\em object} is an {\em instance} of
    a {\em metaclass}.  All {\em classes} in \eulisp\ are {\em subclasses} of
    {\classref{object}}, and all {\em instances} of {\classref{class}} are {\em
        classes}.

    \definition{class precedence list} \index{general}{class precedence list}
    Each {\em class} has a linearized list of all its {\em superclasses}, {\em
        direct} and {\em indirect}, beginning with the {\em class} itself and
    ending with the root of the {\em inheritance} {\em graph}, the {\em class}
    {\tt object}.  This list determines the specificity of slot and method {\em
        inheritance}.  A class's class precedence list may be accessed through
    the {\em function} {\tt class-precedence-list}. The rules used to compute
    this list are determined by the {\em class} of the {\em class} through {\em
        methods} of the {\em generic} {\em function} {\tt
        compute-class-precedence-list}.

    \definition{class option}\index{general}{class option} A keyword and its
    associated value applying to a {\em class} appearing in a class definition
    form, for example: the {\tt predicate} keyword and its value, which defines
    a predicate {\em function} for the {\em class} being defined.

    \definition{closure}\index{general}{closure} A first-class function with
    {\em free variables} that are bound in the {\em lexical environment}. Such a
    function is said to be ``closed over'' its free variables.  Example: the
    function returned by the expression {\tt (let ((x 1)) \#'(lambda () x))} is
    a closure since it closes over the free variable {\tt x}.

    \definition{congruent}\index{general}{congruent} A constraint on the form of
    the {\em lambda-list} of a method, which requires it to have the same number
    of elements as the generic function to which it is to be attached.

    \definition{continuation}\index{general}{continuation} A continuation is a
    function of one parameter which represents the rest of the program.  For
    every point in a program there is the remainder of the program coming after
    that point; this can be viewed as a function of one argument awaiting the
    result of that point.  The {\em current continuation} is the continuation
    that would be derived from the current point in a program's execution, see
    \keywordref{let/cc}.

    \definition{converter function}\index{general}{converter function} The
    generic function associated with a class (the target) that is used to
    project an instance of another class (the source) to an instance of the
    target.

    \definition{defining form}\index{general}{defining form} Any form or any
    {\em macro expression} expanding into a form whose operator is one of
    \defformref{defclass}, \defformref{defcondition}, \defformref{defconstant},
    \defformref{defgeneric}, \defformref{deflocal}, \syntaxref{defmacro},
    \syntaxref{defun}, or \defformref{defglobal}.

    \definition{direct instance}\index{general}{direct
        instance}\index{general}{instance!direct} A direct instance of a class
    {\em class$_1$} is any {\em object} whose most specific {\em class} is {\em
        class$_1$}.

    \definition{direct subclass}\index{general}{direct
        subclass}\index{general}{subclass!direct} A {\em class$_1$} is a direct
    {\em subclass} of {\em class$_2$} if {\em class$_1$} is a {\em subclass} of
    {\em class$_2$}, {\em class$_1$} is not identical to {\em class$_2$}, and
    there is no other {\em class$_3$} which is a {\em superclass} of {\em
        class$_1$} and a {\em subclass} of {\em class$_2$}.

    \definition{direct superclass} \index{general}{direct superclass}
    \index{general}{superclass!direct} A {\em direct superclass} of a class {\em
        class$_1$} is any {\em class} for which {\em class$_1$} is a direct {\em
        subclass}.

    \definition{dynamic environment}\index{general}{dynamic environment} The
    {\em inner} and {\em top dynamic} environment, taken together, are referred
    to as the dynamic environment.

    \definition{dynamic extent}\index{general}{dynamic extent} A lifetime
    constraint, such that the entity is created on control entering an
    expression and destroyed when control exits the expression.  Thus the entity
    only exists for the time between control entering and exiting the
    expression.

    \definition{dynamic scope}\index{general}{dynamic scope} An access
    constraint, such that the {\em scope} of the entity is limited to the {\em
        dynamic extent} of the expression that created the entity.

    \definition{extent}\index{general}{extent} That lifetime for which an entity
    exists.  Extent is constrained to be either {\em dynamic} or {\em
        indefinite}.

    \definition{first-class}\index{general}{first-class} First-class entities
    are those which can be passed as parameters, returned from functions, or
    assigned into a variables.

    \definition{function}\index{general}{function} A function is either a {\em
        continuation}, a {\em simple function} or a {\em generic function}.

    \definition{generic function} \index{general}{generic function} Generic
    functions are {\em functions} for which the {\em method} executed depends on
    the {\em class} of its arguments.  A generic function is defined in terms of
    {\em methods} which describe the action of the generic function for a
    specific set of argument classes called the method's {\em domain}.

    \definition{identifier}\index{general}{identifier} An identifier is the
    syntactic representation of a {\em variable}.

    \definition{indefinite extent}\index{general}{indefinite extent} A lifetime
    constraint, such that the entity exists for ever.  In practice, this means
    for as long as the entity is accessible.

    \definition{indirect instance}\index{general}{indirect
        instance}\index{general}{instance!indirect} An indirect instance of a
    class {\em class$_1$} is any {\em object} whose {\em class} is an {\em
        indirect} {\em subclass} of {\em class$_1$}.

    \definition{indirect subclass}\index{general}{indirect
        subclass}\index{general}{subclass!indirect} A {\em class$_1$} is an
    indirect subclass of {\em class$_2$} if {\em class$_1$} is a {\em subclass}
    of {\em class$_2$}, {\em class$_1$} is not identical to {\em class$_2$}, and
    there is at least one other {\em class$_3$} which is a {\em superclass} of
    {\em class$_1$} and a {\em subclass} of {\em class$_2$}.

    \definition{inheritance graph} \index{general}{inheritance graph} A directed
    labelled acyclic graph whose nodes are {\em classes} and whose edges are
    defined by the {\em direct subclass} relations between the nodes.  This
    graph has a distinguished root, the {\em class} \classref{object}, which is
    a {\em superclass} of every {\em class}.

    \definition{inherited slot description}\index{general}{inherited slot
        description} A {\em slot} {\em description} is inherited for a {\em
        class$_1$} if the {\em slot} {\em description} is defined for another
    {\em class$_2$} which is a direct or indirect {\em superclass} of {\em
        class$_1$}.

    \definition{initarg} \index{general}{initarg} A {\em symbol} used as a
    keyword in an {\em initlist} to mark the value of some {\em slot} or
    additional information.  Used in conjunction with \functionref{make} and the
    other {\em object} initialization functions to initialize the object.  An
    initarg may be declared for a {\em slot} in a class definition form using
    the {\tt initarg} {\em slot} {\em option} or the {\tt initargs} {\em class}
    {\tt option}.

    \definition{initform} \index{general}{initform} A form which is evaluated to
    produce a default initial {\em slot} value.  Initforms are closed in their
    {\em lexical} environments and the resulting {\em closure} is called an {\em
        initfunction}.  An initform may be declared for a {\em slot} in a class
    definition form using the {\tt initform} {\em slot} {\em option}.

    \definition{initfunction} \index{general}{initfunction} A {\em function} of
    no arguments whose result is used as the default value of a {\em slot}.
    Initfunctions capture the {\em lexical} environment of an {\em initform}
    declaration in a class definition form.

    \definition{initlist} \index{general}{init-list} A list of alternating
    keywords and values which describes some not-yet instantiated object.
    Generally the keywords correspond to the {\em initargs} of some {\em class}.

    \definition{inner dynamic}\index{general}{inner dynamic} Inner dynamic
    bindings are created by \keywordref{dynamic-let}, referenced by
    \keywordref{dynamic} and modified by \keywordref{dynamic-setq}.  Inner
    dynamic bindings extend---and can shadow---the dynamically enclosing {\em
        dynamic environment}.

    \definition{inner lexical}\index{general}{inner lexical} Inner lexical
    bindings are created by \keywordref{lambda} and \keywordref{let/cc},
    referenced by {\em variables} and modified by \keywordref{setq}.  Inner
    lexical bindings extend---and can shadow---the lexically enclosing {\em
        lexical environment}.  Note that \keywordref{let/cc} creates an
    immutable {\em binding}.

    \definition{instance} \index{general}{instance} Every {\em object} is the
    instance of some {\em class}.  An instance thus describes an {\em object} in
    relation to its {\em class}.  An instance takes on the structure and
    behaviour described by its {\em class}.  An instance can be either {\em
        direct} or {\em indirect}.

    \definition{instantiation graph} \index{general}{instantiation graph} A
    directed graph whose nodes are {\em objects} and whose edges are defined by
    the {\em instance} relations between the {\em objects}.  This graph has only
    one cycle, an edge from \classref{class} to itself.  The instantation graph
    is a tree and \classref{class} is the root.

    \definition{lexical environment}\index{general}{lexical environment} The
    {\em inner} and {\em top lexical} environment of a module are together
    referred to as the lexical environment except when it is necessary to
    distinguish between them.

    \definition{lexical scope}\index{general}{lexical scope} An access
    constraint, such that the {\em scope} of the entity is limited to the
    textual region of the form creating the entity.  See also {\em lexically
        closer} and {\em shadow}.

    \definition{macro}\index{general}{macro} A macro is a function distinguished
    by when it is used: macro functions are only used during the syntax
    expansion of modules to transform expressions.

    \definition{metaclass} \index{general}{metaclass} A metaclass is a {\em
        class} {\em object} whose {\em instances} are themselves {\em classes}.
    All metaclasses in \eulisp\ are {\em instances} of \classref{class}, which
    is an {\em instance} of itself.  All metaclasses are also {\em subclasses}
    of \classref{class}.  \classref{class} is a metaclass.

    \definition{method} \index{general}{method} A method describes the action of
    a {\em generic} {\em function} for a particular list of argument classes
    called the method's {\em domain}.  A {\em method} is thus said to add to the
    behaviour of each of the {\em classes} in its {\em domain}.  Methods are not
    {\em functions} but {\em objects} which contain, among other information, a
    {\em function} representing the method's behaviour.

    \definition{method function} \index{general}{method function} A {\em
        function} which implements the behaviour of a particular {\em method}.
    Method functions have special restrictions which do not apply to all {\em
        functions}: their formal parameter bindings are immutable, and the
    special forms {\tt call-next-method} and {\tt next-method-p} are only valid
    within the lexical scope of a method function.

    \definition{method specificity} \index{general}{method specificity}
    \index{general}{method!specificity} A domain {\em domain$_1$} is more
    specific than another {\em domain$_2$} if the first {\em class} in {\em
        domain$_1$} is a {\em subclass} of the first {\em class} in {\em
        domain$_2$}, or, if they are the same, the rest of {\em domain$_1$} is
    more specific than the rest of {\em domain$_2$}.

    \definition{multi-method} \index{general}{multi-method} A {\em method} which
    specializes on more than one argument.

    \definition{new instance}\index{general}{new instance} A newly allocated
    {\em instance}, which is distinct, but can be isomorphic to other {\em
        instances}.

    \definition{reflective} \index{general}{reflective} A system which can
    examine and modify its own state is said to be {\em reflective}.  \eulisp\
    is reflective to the extent that it has explicit {\em class} objects and
    {\em metaclasses}, and user-extensible operations upon them.

    \definition{scope}\index{general}{scope} That part of the extent in which a
    given {\em variable} is accessible.  Scope is constrained to be {\em
        lexical}, {\em dynamic} or {\em indefinite}.

    \definition{self-instantiated class} \index{general}{self-instantiated
        class} \index{general}{class!self-instantiated} A {\em class} which is
    an {\em instance} of itself.  In \eulisp, \classref{class} is the only
    example of a self-instantiated class.

    \definition{setter function}\index{general}{setter function} The function
    associated with the function that accesses a place in an entity, which
    changes the value stored in that place.

    \definition{simple function}\index{general}{simple function} A function
    comprises at least: an expression, a set of identifiers, which occur in the
    expression, called the parameters and the closure of the expression with
    respect to the {\em lexical environment} in which it occurs, less the
    parameter identifiers.  Note: this is not a definition of the class
    \classref{simple-function}.

    \definition{slot} \index{general}{slot} A named component of an {\em object}
    which can be accessed using the slot's {\em accessor}.  Each {\em slot} of
    an {\em object} is described by a {\em slot description} associated with the
    {\em class} of the {\em object}.  When we refer to the {\em structure} of an
    {\em object}, this usually means its set of {\em slots}.

    \definition{slot description} \index{general}{slot description} A slot
    description describes a {\em slot} in the {\em instances} of a {\em class}.
    This description includes the {\em slot's} name, its logical position in
    {\em instances}, and a way to determine its default value.  A {\em class's}
    slot descriptions may be accessed through the {\em function}
    \functionref{class-slots}.  A slot description can be either
    {\em direct} or {\em inherited}.

    \definition{slot option} \index{general}{slot option} A keyword and its
    associated value applying to one of the slots appearing in a class
    definition form, for example: the {\tt accessor} keyword and its
    value, which defines a function used to read or write the value of a
    particular slot.

    \definition{slot specification} \index{general}{slot specification} A list
    of alternating keywords and values (starting with a keyword) which
    represents a not-yet-created {\em slot} {\em description} during class
    initialization.

    \definition{special form} \index{general}{special form} A special form is a
    semantic primitive of the language.  In consequence, any processor (for
    example, a compiler or a code-walker) need be able to process only the
    special forms of the language and compositions of them.

    \definition{specialize}\index{general}{specialize} A verbal form used to
    describe the creation of a more specific version of some entity.  Normally
    applied to classes, slots and methods.

    \definition{specialize on}\index{general}{specialize on} A verbal form used
    to describe relationship of methods and the classes specified in their
    domains.

    \definition{subclass} \index{general}{subclass} The behaviour and structure
    defined by a class {\em class$_1$} are inherited by a set of {\em classes}
    which are termed {\em subclasses} of {\em class$_1$}.  A {\em subclass} can
    be either {\em direct} or {\em indirect} or itself.

    \definition{superclass} \index{general}{superclass} A {\em class$_1$} is a
    superclass of {\em class$_2$} iff {\em class$_2$} is a subclass of {\em
        class$_1$}.  A {\em superclass} can be either {\em direct} or {\em
        indirect} or itself.

    \definition{top dynamic}\index{general}{top dynamic} Top dynamic bindings
    are created by \defformref{defglobal}, referenced by \keywordref{dynamic}
    and modified by \keywordref{dynamic-setq}.  There is only one {\em top
        dynamic} environment.

    \definition{top lexical}\index{general}{top lexical} Top lexical bindings
    are created in the {\em top} {\em lexical} environment of a module by
    \begin{flushleft} \defformref{defclass}, \defformref{defcondition},
        \defformref{defconstant}, \defformref{defgeneric},
        \syntaxref{defmacro}, \defformref{defclass}, \syntaxref{defun}.
    \end{flushleft}
    All these bindings are immutable.  \defformref{deflocal} creates a mutable
    top-lexical binding.  All such bindings are referenced by {\em variables}
    and those made by \defformref{deflocal} are modified by \keywordref{setq}.
    Each module defines its own distinct {\em top lexical} environment.

\end{definitions}
\end{optDefinition}

\newpage\clause{Lexical Syntax}
\label{syntax}
\index{general}{lexical syntax}
%
\sclause{Character Set}
\label{character-set}
\index{general}{character set}
%
\begin{optDefinition}
%
Case\index{general}{case sensitivity} is distinguished in each of characters,
strings and identifiers, so that {\tt variable-name} and {\tt Variable-name} are
different, but where a character is used in a positional number representation
(\eg \verb+#\x3Ad+) the case is ignored.  Thus, case is also significant
in this definition and, as will be observed later, all the special form and
standard function names are lower case.  In this section, and throughout this
text, the names for individual character glyphs are those used in \cite{iso646}.

The minimal character set \index{general}{character!minimal character set} to
support \eulisp\ is defined in syntax table~\ref{character-set}.  The language
as defined in this text uses only the characters given in this table.  Thus,
left hand sides of the productions in this table define and name groups of
characters which are used later in this definition: \scref{decimal-digit},
\scref{upper-letter}, \scref{lower-letter}, \scref{letter},
\scref{other-character} and \scref{special-character}.  Any character not
specified here is classified under \scref{other-character}, which permits its
use as an initial or a constituent character of an identifier (see
section~\ref{symbol}).
%
\Syntax
\defSyntax{character-set}{
\begin{syntax}
    \scdef{decimal-digit}: one of \\
    \>  0 1 2 3 4 5 6 7 8 9 \\
    \scdef{upper-letter}: one of \\
    \>  A B C D E F G H I J K L M \\
    \>  N O P Q R S T U V W X Y Z \\
    \scdef{lower-letter}: one of \\
    \>  a b c d e f g h i j k l m \\
    \>  n o p q r s t u v w x y z \\
    \scdef{letter}: \\
    \>  \scref{upper-letter} \\
    \>  \scref{lower-letter} \\
    \scdef{normal-other-character}: one of \\
    \>  * / < = > + . \\
    \scdef{other-character}: \\
    \>  \scref{normal-other-character}
    \>  - \\
    \scdef{special-character}: one of \\
    \>  ; ' , \textbackslash{} " \# ( ) ` | @ \\
    \scdef{level-0-character}: \\
    \>  \scref{decimal-digit} \\
    \>  \scref{letter} \\
    \>  \scref{other-character} \\
    \>  \scref{special-character}
\end{syntax}}
\showSyntaxBox{character-set}
%
\end{optDefinition}
%
\sclause{Whitespace and Comments}
\begin{optPrivate}
    \verb+\#tab+ is omitted from whitespace because it is not a standard
    character.

    JWD: Is \verb+#\tab+ whitespace?  Section 2.4.2 does not include it, but the
    formal syntax in section A.8 does.  In Common Lisp, it is a "semi-standard"
    character, which may indicate that it ought to be somewhat of a special
    case.

    Need to be able to allow whitespace as a constituent of numbers.  Leave for
    later!  Define an extended input syntax??
\end{optPrivate}
%
\begin{optDefinition}
Whitespace characters\index{general}{whitespace} are space and
newline.  The newline character is also used to represent end of record for
configurations providing such an input model, thus, a reference to newline in
this definition should also be read as a reference to end of record.  The only
use of whitespace is to improve the legibility of programs for human readers.
Whitespace separates tokens and is only significant in a string or when it
occurs escaped within an identifier.

A comment\index{general}{comment} is introduced by the \scref{comment-begin}
\index{general}{comment!\sc{comment-begin} glyph} glyph, called \sc{semicolon}
(\verb+;+) and continues up to, but does not include, the end of the line:
%
\Syntax
\defSyntax{comment}{
\begin{syntax}
    \scdef{comment}: \\
    \>  \scref{comment-begin} \sc{all subsequent characters} \\
    \>\>  \sc{up to the end of the line} \\
    \scdef{comment-begin}: \\
    \>  ;
\end{syntax}}
\showSyntaxBox{comment}

Hence, a comment cannot occur in the middle of a token because of the whitespace
in the form of the newline.  Thus a comment is equivalent to whitespace.
%
\begin{note}
    There is no notation in \eulisp\ for block comments.
\end{note}
\end{optDefinition}

\sclause{Objects}
\begin{optDefinition}
An object is either a \scref{literal}, a \scref{symbol} or a \scref{list}.  The
syntax of the classes of objects that can be read by \eulisp\ is defined in the
section of this definition corresponding to the class as defined below:
%
\Syntax
\label{object-syntax}
\defSyntax{object}{
\begin{syntax}
    \scdef{object}: \\
    \>  \scref{literal} \\
    \>  \scref{list}      \>\>\>\S\ref{list} \\
    \>  \scref{symbol}    \>\>\>\S\ref{symbol} \\
    \scdef{literal}: \\
    \>  \scref{boolean} \\
    \>  \scref{character} \>\>\>\S\ref{character} \\
    \>  \scref{float}     \>\>\>\S\ref{float} \\
    \>  \scref{integer}   \>\>\>\S\ref{integer} \\
    \>  \scref{string}    \>\>\>\S\ref{string} \\
    \>  \scref{vector}    \>\>\>\S\ref{vector}
\end{syntax}}
\showSyntaxBox{object}

\sclause{Boolean}
\label{boolean}
\index{general}{boolean}
%
A boolean value is either \sc{false} \index{general}{false}, which is
represented by the empty list---written \nil{} and is also the value of
\constantref{nil}---or \sc{true} \index{general}{true}, which is represented by
any other value than \nil{} or if specified as \true{}:
%
\Syntax
\label{object-syntax}
\defSyntax{boolean}{
\begin{syntax}
    \scdef{boolean}: \\
    \>  \scref{true} \\
    \>  \scref{false} \\
    \scdef{true}: \\
    \>  \true \\
    \>  \scref{object} not \nil \\
    \scdef{false}: \\
    \>  \nil \\
    \>  \constantref{nil}
\end{syntax}}
\showSyntaxBox{boolean}

Although the class containing exactly this set of values is not defined in the
language, notation is abused for convenience and \scref{boolean}
\index{general}{boolean} is defined, for the purposes of this definition, to
mean that set of values.

The syntax for \scref{identifier}s \index{general}{identifier!syntax}
\index{general}{syntax!identifier} corresponds to that for \scref{symbol}s.

\end{optDefinition}

\gdef\module{}\newpage\clause{Modules}
\label{sec:modules}
\index{general}{module}
%
\begin{optDefinition}
The \eulisp\ module scheme has several influences:
LeLisp's\index{general}{LeLisp} module system and module compiler (complice),
Haskell\index{general}{Haskell}, ML\index{general}{Standard
    ML}~\bref{ML-modules}, MIT-Scheme's {\tt make-environment} and T's locales.

All bindings of objects in \eulisp\ reside in some module somewhere.  Also, all
programs in \eulisp\ are written as one or more modules.  Almost every module
imports a number of other modules to make its definition meaningful.  These
imports have two purposes, which are separated in \eulisp: firstly the bindings
needed to process the syntax in which the module is written, and secondly the
bindings needed to resolve the free variables in the module after syntax
expansion.  These bindings are made accessible by specifying which modules are
to be imported for which purpose in a directive at the beginning of each module.
The names of modules are bound in a disjoint binding
environment\index{general}{module!name bindings}\index{general}{binding!of
    module names} which is only accessible via the module definition form.  That
is to say, modules are not first-class.  The body of a module definition
comprises a list of directives followed by a sequence of definitions,
expressions and export forms.

The module mechanism provides abstraction and security in a form complementary
to that provided by the object system.  Indeed, although objects do support data
abstraction, they do not support all forms of information hiding and they are
usually conceptually smaller units than modules.  A module defines a mapping
between a set of names and either local or imported bindings of those names.
Most such bindings are immutable.  The exception are those bindings created by
\defformref{deflocal} which may be modified by both the defining and importing
modules.  There are no implicit imports into a module---not even the special
forms are available in a module that imports nothing.  A module exports nothing
by default.  Mutually referential modules are not possible because a module must
be defined before it can be used.  Hence, the importation dependencies form a
directed acyclic graph.
%
\begin{figure*}[t]
\begin{example}
\label{example:module}
\examplecaption{module directives}
\begin{center}
\begin{minipage}[t]{\textwidth}{
\syntax
(defmodule a-module
  (import
    (module-1                                        ;; import everything from module-1
     (except (binding-a) module-2)                   ;; all but binding-a from module-2
     (only (binding-b) module-3)                     ;; only binding-b from module-3
     (rename
      ((binding-c binding-d) (binding-d binding-c))  ;; all of module-4, but exchange
      module-4))                                     ;; the names of binding-c and binding-d

   syntax
    (syntax-module-1                                 ;; all of the module syntax-module-1
     (rename ((syntax-a syntax-b))                   ;; rename the binding of syntax-a
      syntax-module-2)                               ;; of syntax-module-2 as syntax-b
     (rename ((syntax-c syntax-a))                   ;; rename the binding of syntax-c
      syntax-module-3))                              ;; of syntax-module-3 as syntax-a

   expose
    ((except (binding-e) module-5)                   ;; all but binding-e from module-5
     module-6)                                       ;; export all of module-6

   export
    (binding-1 binding-2 binding-3))                 ;; and three bindings from this module
  ...
  (export local-binding-4)                           ;; a fourth binding from this module
  ...
  (export binding-c)                                 ;; the imported binding binding-c
  ...)
\endsyntax
}
\end{minipage}
\end{center}
\end{example}
\end{figure*}
%
The processing of a module definition uses three environments, which are
initially empty.  These are the top-lexical, the external and the syntax
environments of the module.
%
\begin{description}
    \item[top-lexical] The top-lexical environment comprises all the locally
    defined bindings and all the imported bindings.

    \item[external] The external environment comprises all the exposed
    bindings---bindings from modules being exposed by this module but not
    necessarily imported---and all the exported bindings, which are either local
    or imported.  Thus, the external environment might not be a subset of the
    top-lexical environment because, by virtue of an expose directive, it can
    contain bindings from modules which have not been imported.  This is the
    environment received by any module importing this module.

    \item[syntax] The syntax environment comprises all the bindings available
    for the syntax expansion of the module.
\end{description}
%
Each binding is a pair of a local-name and a module-name.  It is a violation
if any two instances of local-name in any one of these environments have
different module-names.  This is called a name clash.  These environments do not
all need to exist at the same time, but it is simpler for the purposes of
definition to describe module processing as if they do.
\end{optDefinition}

\sclause{Directives}
\index{general}{module!directives}
\label{directives}
%
\begin{optDefinition}
The list of module directives is a sequence of keywords and forms, where the
keywords indicate the interpretation of the forms (see syntax
table~\ref{defmodule-syntax}).  This representation allows for the addition of
further keywords at other levels of the definition and also for
implementation-defined keywords\index{general}{implementation-defined!module
    directives}.  For the keywords given here, there is no defined order of
appearance, nor is there any restriction on the number of times that a keyword
can appear.  Multiple occurrences of any of the directives defined here are
treated as if there is a single directive whose form is the combination of each
of the occurrences.  This definition describes the processing of four keywords,
which are now described in detail.  The syntax of all the directives is given in
syntax syntax table~\ref{defmodule-syntax} and an example of their use appears
in example~\ref{example:module}.
\end{optDefinition}
%
\sclause{Export Directive}
%
\index{general}{module!export}
\begin{optDefinition}
This is denoted by the keyword {\tt export} followed by a sequence of
names of top-lexical bindings---these could be either locally-defined or
imported---and has the effect of making those bindings accessible to any module
importing this module by adding them to the external environment of the module.
A name clash can arise in the external environment from interaction with exposed
modules.
\end{optDefinition}
%
\sclause{Import Directive}
%
\index{general}{module!import}
\label{import}
%
\begin{optDefinition}
This is denoted by the keyword {\tt import} followed by a sequence of {\em
    module descriptor}s (see syntax table~\ref{defmodule-syntax}), being module
names or the filters {\tt except}, {\tt only} and {\tt rename}.  This sequence
denotes the union of all the names generated by each element of the sequence.  A
filter can, in turn, be applied to a sequence of module descriptors, and so the
effect of different kinds of filters can be combined by nesting them.  An import
directive specifies either the importation of a module in its entirety or the
selective importation of specified bindings from a module.

The purpose of this directive is to specify the imported bindings which
constitute part of the top-lexical environment of a module.  These are the
explicit run-time dependencies of the module.  Additional run-time dependencies
may arise as a result of syntax expansion.  These are called implicit run-time
dependencies.

In processing import directives, every name should be thought of as a pair of a
{\em module name} and a {\em local-name}.  Intuitively, a namelist of such pairs
is generated by reference to the module name and then filtered by {\tt except},
{\tt only} and {\tt rename}.  In an import directive, when a namelist has been
filtered, the names are regarded as being defined in the top-lexical environment
of the module into which they have been imported.  A name clash can arise in the
top-lexical environment from interaction between different imported modules.
Elements of an import directive are interpreted as follows:
%
\begin{description}
    \item[{\em module name}] All the exported names from {\em module name}.

    \item[{\tt except}] Filters the names from each {\em module descriptor}
    discarding those specified and keeping all other names.  The {\tt except}
    directive is convenient when almost all of the names exported by a module
    are required, since it is only necessary to name those few that are not
    wanted to exclude them.

    \item[{\tt only}] Filters the names from each {\em module descriptor}
    keeping only those names specified and discarding all other names.  The {\tt
        only} directive is convenient when only a few names exported by a module
    are required, since it is only necessary to name those that are wanted to
    include them.

    \item[{\tt rename}] Filters the names from each {\em module descriptor}
    replacing those with {\em old-name} by {\em new-name}.  Any name not
    mentioned in the replacement list is passed unchanged.  Note that once a
    name has been replaced the new-name is not compared against the replacement
    list again.  Thus, a binding can only be renamed once by a single {\tt
        rename} directive.  In consequence, name exchanges are possible.
\end{description}
\end{optDefinition}
%
\sclause{Expose Directive}
\index{general}{module!expose}
\begin{optDefinition}
This is denoted by the keyword {\tt expose} followed by a list of {\em module
    directive\/}s (see syntax table~\ref{defmodule-syntax}).  The purpose of this
directive is to allow a module to export subsets of the external environments of
various modules without importing them itself.  Processing an expose directive
employs the same model as for imports, namely, a pair of a module-name and a
local-name with the same filtering operations.  When the namelist has been
filtered, the names are added to the external environment of the module begin
processed.  A name clash can arise in the external environment from interaction
with exports or between different exposed modules.  As an example of the use of
{\tt expose}, a possible implementation of the {\tt
    eulisp0}\index{general}{level-0 modules!eulisp0} module is shown in
example~\ref{example:exposes}.
%
\begin{example}
\label{example:exposes}
\examplecaption{module using {\tt expose}}{
\syntax
(defmodule eulisp-level-0
  (expose
    (character collection compare condition convert
     copy double-float elementary-functions event
     formatted-io fixed-precision-integer function
     keyword list lock number object-0 stream string
     symbol syntax-0 table thread vector)))
\endsyntax
}
\end{example}
%
It is also meaningful for a module to include itself in an expose
directive.  In this way, it is possible to refer to all the bindings
in the module being defined.  This is convenient, in combination with
{\tt except} (see \S~\ref{import}), as a way of exporting all but
a few bindings in a module, especially if syntax expansion creates
additional bindings whose names are not known, but should be exported.
\end{optDefinition}
%
\sclause{Syntax Directive}
\label{syntax-directive}
\index{general}{module!syntax}
\begin{optDefinition}
This directive is processed in the same way as an import directive,
except that the bindings are added to the syntax environment.  This
environment is used in the second phase of module processing (syntax
expansion).  These constitute the dependencies for the syntax expansion
of the definitions and expressions in the body of the module.  A name
clash can arise in the syntax environment from interaction between
different syntax modules.

It is important to note that special forms are considered part of the
syntax and they may also be renamed.
\end{optDefinition}

\sclause{Definitions and Expressions}
%
\begin{optDefinition}
Definitions in a module only contain unqualified names---that is, {\em
    local-name}s, using the above terminology.  A top-lexical binding is created
exactly once and shared with all modules that import its exported name from the
module that created the binding.  A name clash can arise in the top-lexical
environment from interaction between local definitions and between local
definitions and imported modules.  Only top-lexical bindings created by
\defformref{deflocal} are mutable---both in the defining module and in any
importing module.  It is a violation to modify an immutable binding.
Expressions, that is non-defining forms, are collected and evaluated in order of
appearance at the end of the module definition process when the top-lexical
environment is complete---that is after the creation and initialization of the
top-lexical bindings.  The exception to this is the \keywordref{progn} form,
which is descended and the forms within it are treated as if the
\keywordref{progn} were not present.  Definitions may only appear either at
top-level within a module definition or inside any number of \keywordref{progn}
forms.  This is specified more precisely in the grammar for a module given
below.
%
\Syntax
\savesyntax\defmoduleZeroSyntax\vbox{
\syntax
module syntax
   = '(', 'defmodule', module name,
     module directives, {level 0 module form}, ')';
module name
   = identifier; (* \[\S\ref{symbol}\] *)
module directives
   = '(', {module directive}, ')';
module directive
   = 'export', '(', {identifier}, ')'
   | 'expose', '(', {module descriptor}, ')'
   | 'import', '(', {module descriptor}, ')'
   | 'syntax', '(', {module descriptor}, ')';
level 0 module form
   = '(', 'export', {identifier}, ')'
   | level 0 expression (* \[\S\ref{control-0}\] *)
   | defining form (* \[\S\ref{control-0}\] *)
   | '(', 'progn', {level 0 module form}, ')';
module descriptor
   = module name
   | module filter;
module filter
   = '(', 'except', '(', {identifier}, ')',
     module descriptor, ')'
   | '(', 'only', '(', {identifier}, ')',
     module descriptor, ')'
   | '(', 'rename', '(', {rename pair}, ')',
     module descriptor, ')';
rename pair
   = '(', identifier, identifier, ')';
\endsyntax
}
%
\savesyntax\defmoduleOneSyntax\vbox{
\syntax
module
   = '(', 'defmodule', module name,
     module directives, {level 1 module form}, ')';
level 1 module form
   = level 1 expression (* \[\S\ref{control-1}\] *)
   | level 0 module form;
\endsyntax
}
\label{defmodule-syntax}
\syntaxtable{module}{\defmoduleZeroSyntax}
%
\end{optDefinition}

\sclause{Module Processing}
%
\begin{optPrivate}
    GN is unhappy with the definition before use requirement.  It could be
    cleaned up by requiring the source to exist but not th e processed module.
\end{optPrivate}
%
\begin{optDefinition}
The following steps summarize the module definition process:
%
\begin{description}
    \item[{\bf directive processing}] This is described in detail in
    \S~\ref{directives}--\ref{syntax-directive}.  This step creates and
    initializes the top-lexical, syntax and external environments.

    \item[{\bf syntax expansion}] \index{general}{macros---see also syntax}
    \index{general}{macro expansion---see also syntax} \index{general}{syntax}
    \index{general}{syntax expansion} The body of the module is expanded
    according to the operators defined in the syntax environment constructed
    from the syntax directive.
    \begin{note}
        The semantics of syntax expansion are still under discussion and will be
        described fully in a future version of the \eulisp\ definition.  In
        outline, however, it is intended that the mechanism should provide for
        hygenic expansion of forms in such a way that the programmer need have
        no knowledge of the expansion-time or run-time dependencies of the
        syntax defining module.  Currently syntax expansion is unhygienic to
        allow a simple syntax for syntax macro definition.
    \end{note}

    \item[{\bf static analysis}] The expanded body of the module is analyzed.
    Names referenced in export forms are added to the external environment.
    Names defined by defining forms are added to the top-lexical environment. It
    is a violation, if a free identifier in an expression or defining form
    does not have a binding in the top-lexical environment.
    \begin{note}
        Additional implementation-defined steps may be added here, such as
        compilation.
    \end{note}

    \item[{\bf initialization}] The top-lexical bindings of the module (created
    above) are initialized by evaluating the defining forms in the body of the
    module in the order they appear.
    \begin{note}
        In this sense, a module can be regarded as a generalization of the
        \keywordref{labels} form of this and other Lisp dialects.
    \end{note}

    \item[{\bf expression evaluation}] The expressions in the body of the module
    are evaluated in the order in which they appear.
\end{description}
%
\end{optDefinition}
%
\sclause{Module Definition}
\label{defmodule}
\ttindex{defmodule}
%
\begin{optDefinition}
%
\syntaxform{defmodule}
\Syntax
\noindent
The syntax of a module and its constituents is defined in
syntax table~\ref{defmodule-syntax}.
%
\begin{arguments}
    \item[module name] A symbol used to name the module.
    \item[module directives] A form specifying the exported names, exposed
    modules, imported modules and syntax modules used by this module.
    \item[module form] One of a defining form, an expression or an export
    directive.
\end{arguments}
%
\remarks%
The \syntaxref{defmodule} form defines a module named by {\em module name}
and associates the name {\em module name} with a module object in the
module binding environment\index{general}{binding!of module
names}\index{general}{module!name bindings}.
\begin{note}
    Intentionally, nothing is defined about any relationship between modules and
    files.
\end{note}
%
\examples
An example module definition with explanatory comments is given in
example~\ref{example:module}.
%
\end{optDefinition}


\gdef\module{level-0}\newpage\input{telos-0}
\gdef\module{level-0}\newpage\clause{Level-0 Defining, Special and Function-call Forms}
%
\label{control-0}
\index{general}{level-0}
%
\begin{optDefinition}
\noindent
This section gives the syntax of well-formed expressions and describes the
semantics of the special-forms, functions and macros of the level-0 language.
In the case of level-0 macros, the description includes a set of expansion
rules.  However, these descriptions are not prescriptive of any processor and a
conforming program cannot rely on adherence to these expansions.
\end{optDefinition}

\sclause{Simple Expressions}
%
\begin{optPrivate}
    What happens with vectors, instances etc.  They can be printed out, so why
    can't they be constants too.

    HED wants ids as well as ids and initializers for \specopref{let}.
\end{optPrivate}
%
\begin{optDefinition}
\noindent
%
\syntaxform{constant}
\ttindex{constant}
\noindent
There are two kinds of constants\index{general}{constant!literal}, literal
constants and defined constants\index{general}{constant!defined}.  Only the
first kind are considered here.  A literal constant is a number, a string, a
character, or the empty list.  The result of processing such a literal constant
is the constant itself---that is, it denotes
itself\index{general}{processing!constants}.
%
\examples
\begin{tabular}{ll}
    \verb+()+ & the empty list\\
    \verb+123+ & a fixed precision integer\\
    \verb+#\a+ & a character\\
    \verb+"abc"+ & a string
\end{tabular}

\defop{defconstant}
%
\Syntax
\defSyntax{defconstant}{
\begin{syntax}
    \scdef{defconstant-form}: \ra{} \classref{object} \\
    \>  ( \defopref{defconstant} \scref{constant-name} \scref{form} ) \\
    \scdef{constant-name}: \\
    \>  \scref{identifier}
\end{syntax}}%
\showSyntaxBox{defconstant}
%
\begin{arguments}
    \item[identifier] A symbol naming an immutable top-lexical binding to be
    initialized with the value of \scref{form}.

    \item[form] The \scref{form} whose value will be stored in the binding of
    \scref{identifier}.
\end{arguments}
%
\remarks%
The value of \scref{form} is stored in the top-lexical binding of
\scref{identifier}.  It is a violation to attempt to modify the binding of a
defined constant.

\constant{t}{symbol}
%
\remarks%
This may be used to denote the abstract boolean value \sc{true}, but so may any
other value than \nil{}.

\syntaxform{symbol}
\ttindex{symbol}
\noindent
The current lexical binding of \syntaxref{symbol} is
returned\index{general}{processing!symbols}.  A symbol can also name a defined
constant---that is, an immutable top-lexical binding.

\defop{deflocal}
%
\Syntax
\defSyntax{deflocal}{
\begin{syntax}
    \scdef{deflocal-form}: \ra{} \classref{object} \\
    \>  ( \defopref{deflocal} \scref{local-name} \scref{form} ) \\
    \scdef{local-name}: \\
    \>  \scref{identifier}
\end{syntax}}%
\showSyntaxBox{deflocal}
%
\begin{arguments}
    \item[identifier] A symbol naming a binding containing the value of
    \scref{form}.

    \item[form] The \scref{form} whose value will be stored in the binding of
    \scref{identifier}.
\end{arguments}
%
\remarks%
The value of \scref{form} is stored in the top-lexical binding of
\scref{identifier}.  The binding created by a \defopref{deflocal} form is
mutable.
%
\seealso%
\specopref{setq}.

\specop{quote}
\index{general}{literal}
%
\Syntax
\defSyntax{quote}{
\begin{syntax}
    \scdef{quote-form}: \ra{} \scref{object} \\
    \>  ( \specopref{quote} \scref{object} ) \\
    \>  \syntaxref{'}\scref{object}
\end{syntax}}%
\showSyntaxBox{quote}
%
\begin{arguments}
    \item[object] the \scref{object} to be quoted.
\end{arguments}
%
\result%
The result is \scref{object}.
%
\remarks%
The result of processing the expression {\tt (\specopref{quote} \scref{object})}
is \scref{object}.  The \scref{object} can be any object having an external
representation \index{general}{external representation (see also
    \functionref{prin} and \functionref{write})}.  The special form
\specopref{quote} can be abbreviated using {\em apostrophe} --- graphic
representation~\verb+'+\ttsubindex{quote}{abbreviation with \syntaxref{'}} ---
so that {\tt (\specopref{quote} a)} can be written {\tt \syntaxref{'}a}.  These
two notations are used to incorporate literal constants
\index{general}{literal!quotation} in programs.  It is an error to modify a
literal expression \index{general}{literal!modification of}.

\syntaxform{'}
%
\remarks%
See \specopref{quote}.

\end{optDefinition}

\sclause{Functions: creation, definition and application}
\label{function}
%
\begin{optPrivate}
    Need more detail about \functionref{apply}.

    The stuff about macroexpand is probably redundant now?  Although I
    suppose we need something at top-level??
\end{optPrivate}
%
\begin{optDefinition}
%
\specop{lambda}
%
\Syntax
\label{lambda-syntax-table}
\defSyntax{lambda}{
\begin{syntax}
    \scdef{lambda-form}: \ra{} \classref{function} \\
    \>  ( \specopref{lambda} \scref{lambda-list} \scref{body} ) \\
   \scdef{lambda-list}: \\
   \>  \scref{identifier} \\
   \>  \scref{simple-list} \\
   \>  \scref{rest-list} \\
   \scdef{simple-list}: \\
   \>  ( \scseqref{identifier} ) \\
   \scdef{rest-list}: \\
   \>  ( \scseqref{identifier} . \scref{identifier} ) \\
   \scdef{body}: \\
   \>  \scseqref{form}
\end{syntax}}%
\showSyntaxBox{lambda}
%
\begin{arguments}
    \item[lambda-list] The parameter list of the function conforming to the
    syntax \ref{lambda-syntax-table}.
    \item[form] An expression.
\end{arguments}
%
\result%
A function with the specified \scref{lambda-list} and sequence of \scref{form}s.
%
\remarks%
The function construction operator is \specopref{lambda}.  Access to the lexical
environment of definition is guaranteed.  The syntax of \scref{lambda-list} is
defined in ref{lambda-syntax-table}.

If \scref{lambda-list} is an \scref{identifier}, it is bound to a newly
allocated list of the actual parameters.  This binding \index{general}{scope and
    extent!of \specopref{lambda} bindings} has lexical scope and indefinite
extent.  If \scref{lambda-list} is a \scref{simple-list}, the arguments are
bound to the corresponding \scref{identifier}.  Otherwise, \scref{lambda-list}
must be a \scref{rest-list}.  In this case, each \scref{identifier} preceding
the dot is bound to the corresponding argument and the \scref{identifier}
succeeding the dot is bound to a newly allocated list whose elements are the
remaining arguments.  These bindings have lexical scope and indefinite extent.
It is a violation if the same identifier appears more than once in a
\scref{lambda-list}.  It is an error to modify \scref{rest-list}.

\defop{defmacro}
%
\Syntax
\defSyntax{defmacro}{
\begin{syntax}
    \scdef{defmacro-form}: \ra{} \classref{function} \\
    \>  ( \defopref{defmacro} \scref{macro-name} \scref{lambda-list}
    \scref{body} ) \\
    \scdef{macro-name}: \\
    \> \scref{identifier}
\end{syntax}}%
\showSyntaxBox{defmacro}
%
\begin{arguments}
    \item[macro-name] A symbol naming an immutable top-lexical binding to be
    initialized with a function having the specified \scref{lambda-list} and
    \scref{body}.

    \item[lambda-list] The parameter list of the function conforming to the
    syntax specified under \specopref{lambda}.

    \item[body] A sequence of forms.
\end{arguments}
%
\remarks%
The \defopref{defmacro} form defines a function \index{general}{macro} named by
\scref{macro-name} and stores the definition as the top-lexical binding of
\scref{macro-name} \index{general}{macro!definition by \defopref{defmacro}}
\index{general}{binding!module}.  The interpretation of the \scref{lambda-list}
is as defined for \specopref{lambda} (see \ref{lambda-syntax-table}).
%
\begin{note}
    A macro is automatically exported from the the module which defines it.  A
    macro cannot be used in the module which defines it.
\end{note}
%
\seealso%
\specopref{lambda}.

\defop{defun}
%
\Syntax
\defSyntax{defun}{
\begin{syntax}
    \scdef{defun-form}: \ra{} \classref{function} \\
    \>  \scref{simple-defun} \\
    \>  \scref{setter-defun} \\
   \scdef{simple-defun}: \\
   \>  ( \defopref{defun} \scref{function-name} \scref{lambda-list} \\
   \>\>  \scref{body} ) \\
   \scdef{setter-defun}: \\
   \>  ( \defopref{defun} ( \functionref{setter} \scref{function-name} )
   \scref{lambda-list} \\
   \>\>  \scref{body} ) \\
   \scdef{function-name}: \\
   \>  \scref{identifier}
\end{syntax}}%
\showSyntaxBox{defun}
%
\begin{arguments}
    \item[function-name] A symbol naming an immutable top-lexical binding to be
    initialized with a function having the specified \scref{lambda-list} and
    \scref{body}.

    \item[{\tt (\functionref{setter} \scref{function-name})}] An expression
    denoting the setter function to correspond to \scref{function-name}.

    \item[lambda-list] The parameter list of the function conforming to the
    syntax specified under \specopref{lambda}.

    \item[body] A sequence of forms.
\end{arguments}
%
\remarks%
The \defopref{defun} form defines a function named by \scref{function-name} and
stores the definition (i) as the top-lexical binding of \scref{function-name} or
(ii) as the setter function of \scref{function-name}.  The interpretation of the
\scref{lambda-list} is as defined for \specopref{lambda}.
%
\rewriterules
%
\begin{RewriteTable}{defun}{lll}
\begin{minipage}[t]{\columnwidth}%
    \begin{tabbing}%
        00\= \kill
        (\defopref{defun} \scref{identifier}\\
        \>\scref{lambda-list}\\
        \>\scref{body})
    \end{tabbing}
\end{minipage}
&\rewrite&
\begin{minipage}[t]{\columnwidth}
    \begin{tabbing}
        00\= \kill
        (\defopref{defconstant} \scref{identifier}\\
        \>(\specopref{lambda} \scref{lambda-list} \\
        \>\scref{body}))
    \end{tabbing}
\end{minipage}\\
\\
\begin{minipage}[t]{\columnwidth}
    \begin{tabbing}
        00\= \kill
        (\defopref{defun}\\
        \>(\functionref{setter} \scref{identifier})\\
        \>\scref{lambda-list} \\
        \>\scref{body})
    \end{tabbing}
\end{minipage}
&\rewrite&
\begin{minipage}[t]{\columnwidth}
    \begin{tabbing}
        00\= \kill
        ((\functionref{setter} \functionref{setter}) \\
        \>\scref{identifier}\\
        \>(\specopref{lambda} \scref{lambda-list} \\
        \>\scref{body}))
    \end{tabbing}%
\end{minipage}%
\end{RewriteTable}

\syntaxform{function call}
\index{general}{function!calling}
%
\Syntax
\defSyntax{function-call}{
\begin{syntax}
    \scdef{function-call-form}: \ra{} \classref{object} \\
    \>  ( \scref{operator} \scseqref{operand} ) \\
    \scdef{operator}: \\
    \>  \scref{identifier} \\
    \scdef{operand}: \\
    \>  \scref{identifier} \\
    \>  \scref{literal} \\
    \>  \scref{special-form} \\
    \>  \scref{function-call-form}
\end{syntax}}%
\showSyntaxBox{function-call}
%
\begin{arguments}
    \item[\scref{operator}] This may be a symbol---being either the name of a
    special form, or a lexical variable---or a function call, which must result
    in an instance of \classref{function}.

    An error is signalled (condition class: \conditionref{invalid-operator}
    \indexcondition{invalid-operator}) if the operator is not a function.

    \item[\scseqref{operand}] Each \scref{operand} must be either an
    \scref{identifier}, a \scref{literal}, a \scref{special-form} or a
    \scref{function-call-form}.
\end{arguments}
%
\result%
The result is the value of the application of \scref{operator} to the
evaluation of \scseqref{operand}.
%
\remarks%
The \scref{operand} expressions are evaluated in order from left to
right.  The \scref{operator} expression may be evaluated at any time
before, during or after the evaluation of the operands.
%
\begin{note}
    The above rule for the evaluation of function calls was finally agreed upon
    for this version since it is in line with one strand of common practice, but
    it may be revised in a future version.
\end{note}
%
\seealso%
\syntaxref{constant}, \syntaxref{symbol}, \specopref{quote}.

\condition{invalid-operator}{general-condition}
%
\begin{initoptions}
    \item[invalid-operator, object] The object which was being used as an
    operator.

    \item[operand-list, list] The operands prepared for the operator.
\end{initoptions}
%
\remarks%
Signalled by function call if the operator is not an instance of
\classref{function}.

\function{apply}
%
\Syntax
\defSyntax{apply}{
\begin{syntax}
    \scdef{apply-form}: \ra{} \classref{object} \\
    \>  ( \functionref{apply} \scref{function} \scref{body} ) \\
    \scdef{function}: \\
    \>  \scref{level-0-form}
\end{syntax}}%
\showSyntaxBox{apply}
%
\begin{arguments}
    \item[function] A form which must evaluate to an instance of
    \classref{function}.

    \item[form$_1$ ... form$_{n-1}$] A sequence of expressions, which will be
    evaluated according to the rules given in \scref{function-call-form}.

    \item[form$_n$] An expression which must evaluate to a proper list.  It is
    an error if {\em obj$_n$} is not a proper list.
\end{arguments}
%
\result%
The result is the result of calling \scref{function} with the actual parameter
list created by appending \scref{form}$_n$ to a list of the arguments
\scref{form}$_1$ through \scref{form}$_{n-1}$.  An error is signalled (condition
class: \conditionref{invalid-operator}\indexcondition{invalid-operator}) if the
first argument is not an instance of \classref{function}.
%
\seealso%
\scref{function-call-form}, \conditionref{invalid-operator}.
\end{optDefinition}

\sclause{Destructive Operations}
%
\begin{optPrivate}
    The term {\em closer} is a weak specification---cross ref terminology?
\end{optPrivate}
%
\begin{optDefinition}
\noindent
An assignment operation\index{general}{assignment} modifies the contents of a
binding named by a identifier---that is, a variable.

\specop{setq}
%
\Syntax
\defSyntax{setq}{
\begin{syntax}
    \scdef{setq-form}: \ra{} \classref{object} \\
    \>  ( \specopref{setq} \scref{identifier} \scref{form} )
\end{syntax}}%
\showSyntaxBox{setq}
%
\begin{arguments}
    \item[identifier] The identifier whose lexical binding is to be updated.

    \item[form] An expression whose value is to be stored in the binding of
    \scref{identifier}.
\end{arguments}
%
\result%
The result is the value of \scref{form}.
%
\remarks%
The \scref{form} is evaluated and the result is stored in the closest lexical
binding named by \scref{identifier}.  It is a violation to modify an immutable
binding.

\function{setter}
%
\begin{arguments}
    \item[reader] An expression which must evaluate to an instance of
    \classref{function}.
\end{arguments}
%
\result%
The {\em writer\/} corresponding to {\em reader}.
%
\remarks%
A generalized place update facility is provided by \functionref{setter}.  Given
{\em reader}\index{general}{function!reader}, \functionref{setter} returns the
corresponding update function\index{general}{function!writer}.  If no such
function is known to \functionref{setter}, an error is signalled (condition
class: \conditionref{no-setter}\indexcondition{no-setter}).  Thus {\tt
    (\functionref{setter} \functionref{car})} returns the function to update the
\functionref{car} of a pair.  New update functions can be added by using
{setter}'s update function, which is accessed by the expression {\tt
    (\functionref{setter} \functionref{setter})}.  Thus {\tt
    ((\functionref{setter} \functionref{setter}) a-reader a-writer)} installs
the function which is the value of {\tt a-writer} as the writer of the reader
function which is the value of {\tt a-reader}.  All writer functions in this
definition and user-defined writers have the same immutable status as other
standard functions, such that attempting to redefine such a function, for
example {\tt ((\functionref{setter} \functionref{setter}) \functionref{car}
    a-new-value)}, signals an error (condition class:
\conditionref{cannot-update-setter}\indexcondition{<cannot-update-setter>})
%
\seealso%
\defopref{defgeneric}, \defopref{defmethod}, \defopref{defclass},
\defopref{defun}.

\condition{no-setter}{general-condition}
%
\begin{initoptions}
    \item[object, object] The object given to \functionref{setter}.
\end{initoptions}
%
\remarks%
Signalled by \functionref{setter} if there is no updater for the given
function.

\condition{cannot-update-setter}{general-condition}
%
\begin{initoptions}
    \item[accessor, object$_1$] The given accessor object.

    \item[updater, object$_2$] The given updater object.
\end{initoptions}
%
\remarks%
Signalled by {\tt (\functionref{setter} \functionref{setter})} if the updater of
the given accessor is immutable.
%
\seealso%
\functionref{setter}.
%
\end{optDefinition}

\sclause{Conditional Expressions}
%
\begin{optPrivate}
    \specopref{cond} rules are strange (RPG).

    Added \specopref{when} and \specopref{unless}.

    Moved these here after October '90 meeting.
\end{optPrivate}
%
\begin{optDefinition}
%
\specop{if}
%
\Syntax
\defSyntax{if}{
\begin{syntax}
    \scdef{if-form}: \ra{} \classref{object} \\
    \>  ( \specopref{if} \scref{antecedent} \\
    \>\>\> \scref{consequent} \\
    \>\>  \scref{alternative} ) \\
    \scdef{antecedent}: \\
    \>  \scref{form} \\
    \scdef{consequent}: \\
    \>  \scref{form} \\
    \scdef{alternative}: \\
    \>  \scref{form} \\
\end{syntax}}%
\showSyntaxBox{if}
%
\result%
Either the value of \scref{consequent} or \scref{alternative} depending on the
value of \scref{antecedent}.
%
\remarks%
The \scref{antecedent} is evaluated.  If the result is \true{} the
\scref{consequent} is evaluated, otherwise the \scref{alternative} is evaluated.
Both \scref{consequent} and \scref{alternative} must be specified.  The result
of \specopref{if} is the result of the evaluation of whichever of
\scref{consequent} or \scref{alternative} is chosen.

\specop{cond}
%
\Syntax
\defSyntax{cond}{
    \begin{syntax}
    \scdef{cond-form}: \ra{} \classref{object} \\
    \>  ( \specopref{cond} \\
    \>\>  \scgseq{{\tt(} \scref{antecedent} \scseqref{consequent} {\tt)}} )
\end{syntax}}%
\showSyntaxBox{cond}
%
\remarks%
The \specopref{cond} macro provides a convenient syntax for collections of {\em
    if-then-elseif...else} expressions.
%
\rewriterules
%
\begin{RewriteTable}{cond}{lll}
    (\specopref{cond}) &\rewrite& () \\
    (\specopref{cond} (\scref{antecedent}) \\
    \tts\ldots) &\rewrite&
    (\specopref{or} \scref{antecedent} (\specopref{cond} \ldots)) \\
\begin{minipage}[t]{0.45\columnwidth}
\begin{tabbing}
    00\=00\= \kill
    (\specopref{cond} \\
    \>(\scref{antecedent}$_1$) \\
    \>(\scref{antecedent}$_2$ \scseqref{consequent}) \\
    \>\ldots)
\end{tabbing}
\end{minipage}
&\rewrite&
\begin{minipage}[t]{0.45\columnwidth}
\begin{tabbing}
    (\specopref{or} \= \scref{antecedent}$_1$ \\
    \>00\=00\= \kill
    \>(\specopref{cond} \\
    \>\>(\scref{antecedent}$_2$ \\
    \>\>\>\scseqref{consequent}) \\
    \>\>\ldots))
\end{tabbing}
\end{minipage} \\

\begin{minipage}[t]{0.45\columnwidth}
\begin{tabbing}
    00\=00\= \kill
    (\specopref{cond} \\
    \>(\scref{antecedent}$_1$ \scseqref{consequent}) \\
    \>(\scref{antecedent}$_2$ \scseqref{consequent}) \\
    \>\ldots)
\end{tabbing}
\end{minipage}
&\rewrite&
\begin{minipage}[t]{0.45\columnwidth}
\begin{tabbing}
    (\specopref{if} \=\scref{antecedent}$_1$ \\
    \>(\specopref{progn} \scseqref{consequent}) \\
    \>00\=00\= \kill
    \>(\specopref{cond} \\
    \>\>(\scref{antecedent}$_2$ \\
    \>\>\>\scseqref{consequent}) \\
    \>\>\ldots))
\end{tabbing}%
\end{minipage}%
\end{RewriteTable}

\constant{else}{symbol}
%
\remarks%
This may be used to denote the default clause in \specopref{cond} and
\specopref{case} forms and has the value \constantref{t}, \ie it is an alias for
\constantref{t} introduced to improve readability of the \specopref{cond} and
\specopref{case} forms.

\specop{when}
%
\Syntax
\defSyntax{when}{
    \begin{syntax}
        \scdef{when-form}: \ra{} \classref{object} \\
        \>  ( \specopref{when} \scref{antecedent} \\
        \>\>\> \scref{consequent} ) \\
    \end{syntax}}%
\showSyntaxBox{when}
%
\result%
The \scref{antecedent} is evaluated and if the result is \true{} the
\scref{consequent} is evaluated and returned otherwise \nil{} is returned.
%
\rewriterules
%
\begin{RewriteTable}{when}{lll}
    \begin{minipage}[t]{0.45\columnwidth}
        \begin{tabbing}
            00\=00\= \kill
            (\specopref{when} \scref{antecedent} \\
            \>\scref{consequent})
        \end{tabbing}
    \end{minipage}
    &\rewrite&
    \begin{minipage}[t]{0.45\columnwidth}
        \begin{tabbing}
            00\=00\= \kill
            (\specopref{if} \=\scref{antecedent} \\
            \>\>\scref{consequent} \\
            \>())
        \end{tabbing}%
    \end{minipage}%
\end{RewriteTable}

\specop{unless}
%
\Syntax
\defSyntax{unless}{
    \begin{syntax}
        \scdef{unless-form}: \ra{} \classref{object} \\
        \>  ( \specopref{unless} \scref{antecedent} \\
        \>\>\> \scref{consequent} ) \\
    \end{syntax}}%
\showSyntaxBox{unless}
%
\result%
The \scref{antecedent} is evaluated and if the result is \nil{} the
\scref{consequent} is evaluated and returned otherwise \nil{} is returned.
%
\rewriterules
%
\begin{RewriteTable}{unless}{lll}
    \begin{minipage}[t]{0.45\columnwidth}
        \begin{tabbing}
            00\=00\= \kill
            (\specopref{unless} \scref{antecedent} \\
            \>\scref{consequent})
        \end{tabbing}
    \end{minipage}
    &\rewrite&
    \begin{minipage}[t]{0.45\columnwidth}
        \begin{tabbing}
            00\=00\= \kill
            (\specopref{if} \=\scref{antecedent} \\
            \>\>() \\
            \>\scref{consequent})
        \end{tabbing}%
    \end{minipage}%
\end{RewriteTable}

\specop{and}
%
\Syntax
\defSyntax{and}{
\begin{syntax}
    \scdef{and-form}: \ra{} \classref{object} \\
    \>  ( \specopref{and} \scseqref{consequent} )
\end{syntax}}%
\showSyntaxBox{and}
%
\remarks%
The expansion of an \specopref{and} form leads to the evaluation of the sequence
of \scref{form}s from left to right.  The first \scref{form} in the sequence
that evaluates to \nil{}\/ stops evaluation and none of the \scref{form}s to its
right will be evaluated---that is to say, it is non-strict.  The result of {\tt
    (\specopref{and})} is \true{}.  If none of the \scref{form}s evaluate to
\nil{}, the value of the last \scref{form} is returned.
%
\rewriterules
%
\begin{RewriteTable}{and}{lll}
    (\specopref{and}) &\rewrite& \true{} \\
    (\specopref{and} \scref{form}) &\rewrite& \scref{form} \\
    (\specopref{and} \scref{form}$_1$ \scref{form}$_2$ \ldots) &\rewrite&
\begin{minipage}[t]{0.45\columnwidth}
\begin{tabbing}
    (\specopref{if} \= \scref{form}$_1$\\
    \>(\specopref{and} \scref{form}$_2$ \ldots)\\
    \>())
\end{tabbing}%
\end{minipage}%
\end{RewriteTable}

\specop{or}
%
\Syntax
\defSyntax{or}{
\begin{syntax}
    \scdef{or-form}: \ra{} \classref{object} \\
    \>  ( \specopref{or} \scseqref{form} )
\end{syntax}}%
\showSyntaxBox{or}
%
\remarks%
The expansion of an \specopref{or} form leads to the evaluation of the sequence
of \scref{form}s from left to right.  The value of the first \scref{form} that
evaluates to \true{} is the result of the \specopref{or} form and none of the
\scref{form}s to its right will be evaluated---that is to say, it is non-strict.
If none of the forms evaluate to \true{}, the value of the last \scref{form}
is returned.
%
\rewriterules
%
\begin{RewriteTable}{or}{lll}
    (\specopref{or}) &\rewrite& () \\
    (\specopref{or} \scref{form}) &\rewrite& \scref{form} \\
    (\specopref{or} \scref{form}$_1$ \scref{form}$_2$ \ldots) &\rewrite&
\begin{minipage}[t]{0.45\columnwidth}
\begin{tabbing}
    (\specopref{let} (\=(x \scref{form}$_1$))\\
    \>(\specopref{if} \= x\\
    \>\>x\\
    \>\>(\specopref{or} \scref{form}$_2$ \ldots)))
\end{tabbing}%
\end{minipage}%
\end{RewriteTable}

Note that {\tt x} does not occur free in any of \scref{form}$_2$ \ldots
\scref{form}$_n$.
%
\end{optDefinition}

\sclause{Variable Binding and Sequences}
\label{subsubsec:variable-binding}
%
\begin{optPrivate}
    The term {\em closure} is not defined.  Should expand on syntactic variants
    and precise the definition of the scope and extent of the binding.

    Should make clear that \scref{body} is a sequence of forms.  The term
    continuation is not defined.  RPG finds ``The invocation...invalidated''
    obscure.

    In {\tt make-values} is the object copied or shared?  Rather sparse!

    The phrase ``...object named by variable'' is weak.

    Does {\tt multiple-argument-values} copy or not?

    JAP removed all the multiple argument junk until a better model can be
    proposed.

    JAP would like to weaken the requirement that let/cc ``signal an error'' to
    be ``is an error''.  Done.  Perhaps someone will notice this issue now.

    Should \scref{lambda-list} become \scref{specialized-lambda-list}?
\end{optPrivate}
%
\begin{optDefinition}

\specop{let/cc}
\ttsubindex{let/cc}{see also \specopref{block} and \specopref{return-from}}
%
\Syntax
\defSyntax{let/cc}{
\begin{syntax}
    \scdef{let/cc-form}: \ra{} \classref{object} \\
    \>  ( \specopref{let/cc} \scref{identifier} \scref{body} )
\end{syntax}}%
\showSyntaxBox{let/cc}
%
\begin{arguments}
    \item[\scref{identifier}] To be bound to the continuation of the
    \specopref{let/cc} form.
    \item[\scref{body}] A sequence of forms to evaluate.
\end{arguments}
%
\result%
The result of evaluating the last form in \scref{body} or the value of
the argument given to the continuation bound to \scref{identifier}.
%
\remarks%
The \scref{identifier} is bound to a new location, which is initialized with the
continuation of the \specopref{let/cc} form.  This binding is immutable and has
lexical scope and indefinite extent\index{general}{scope and extent!of
    \specopref{let/cc} binding}.  Each form in \scref{body} is evaluated in order
in the environment extended by the above binding.  It is an error to call the
continuation outside the dynamic extent of the \specopref{let/cc} form that
created it.  The continuation is a function of one argument.  Calling the
continuation causes the restoration of the lexical environment and dynamic
environment that existed before entering the \specopref{let/cc} form.
%
\examples%
An example of the use of \specopref{let/cc} is given in
example~\ref{example:pathopen}.  The function {\tt path-open} takes a list of
paths, the name of a file and list of options to pass to {\tt open}.  It tries
to open the file by appending the name to each path in turn.  Each time {\tt
    open} fails, it signals a condition that the file was not found which is
trapped by the handler function.  That calls the continuation bound to fail to
cause it to try the next path in the list.  When {\tt open} does find a file,
the continuation bound to {\tt succeed} is called with the stream as its
argument, which is subsequently returned to the caller of {\tt path-open}.  If
the path list is exhausted, \genericref{map} (section~\ref{collection})
terminates and an error (condition class: \conditionref{cannot-open-path}) is
signalled.
%
\begin{example}
\label{example:pathopen}
\examplecaption{using \specopref{let/cc}}
{\codeExample
(defun path-open (pathlist name . options)
  (let/cc succeed
    (map
      (lambda (path)
        (let/cc fail
          (with-handler
            (lambda (condition resume) (fail ()))
            (succeed
              (apply open
                (format () "~a/~a" path name)
                options)))))
      pathlist)
    (error
      (format ()
        "Cannot open stream for (~a) ~a"
        pathlist name)
      <cannot-open-path>)))
\endCodeExample}
\end{example}
%
\seealso%
\specopref{block}, \specopref{return-from}.

\specop{block}
%
\ttsubindex{block}{see also \specopref{let/cc}}
\Syntax
\defSyntax{block}{
\begin{syntax}
    \scdef{block-form}: \ra{} \classref{object} \\
    \>  ( \specopref{block} \scref{identifier} \scref{body} )
\end{syntax}}%
\showSyntaxBox{block}
%
\remarks%
The block expression is used to establish a statically scoped binding of an
escape function.  The block \scref{identifier} is bound to the
continuation\index{general}{continuation} of the block.  The continuation can be
invoked anywhere within the block by using \specopref{return-from}.  The
\scref{form}s are evaluated in sequence and the value of the last one is
returned as the value of the block form.  See also \specopref{let/cc}.
%
\rewriterules
%
\begin{RewriteTable}{block}{lll}
    (\specopref{block} \scref{identifier}) &\rewrite& ()\\
    (\specopref{block} \scref{identifier}  &\rewrite&
    (\specopref{let/cc} \scref{identifier} \\
    \tts\scref{body})                    && \tts\scref{body})
\end{RewriteTable}

The rewrite for \specopref{block} does not prevent the \specopref{block} being
exited from anywhere in its dynamic extent, since the function bound
to \scref{identifier} is a first-class item and can be passed as an
argument like other values.
%
\seealso%
\specopref{return-from}.

\specop{return-from}
\ttsubindex{return-from}{see also \specopref{let/cc}}
%
\Syntax
\defSyntax{return-from}{
\begin{syntax}
    \scdef{return-from-form}: \ra{} \classref{object} \\
    \>  ( \specopref{return-from} \scref{identifier} \scoptref{form} )
\end{syntax}}%
\showSyntaxBox{return-from}
%
\remarks%
In \specopref{return-from}, the \scref{identifier} names the continuation of
the (lexical) \specopref{block} from which to return.
%An error is signalled
%(condition class: {\tt
%<invalid-return-continuation>}\indexcondition{invalid-return-continuation})
%if the value of the variable named by \scref{identifier} is not a
%continuation.
\specopref{return-from} is the invocation of the continuation of the block
named by \scref{identifier}.  The \scref{form} is evaluated and the value
is returned as the value of the block named by \scref{identifier}.
%
\rewriterules
%
\begin{RewriteTable}{return-from}{lll}
    (\specopref{return-from} \scref{identifier}) &\rewrite& (\scref{identifier} ())\\
\begin{minipage}[t]{0.45\columnwidth}
\begin{tabbing}
    00\= \kill
    (\specopref{return-from} \\
    \>\scref{identifier} \scref{form})
\end{tabbing}
\end{minipage}
&\rewrite& (\scref{identifier} \scref{form})
\end{RewriteTable}
%
\seealso%
\specopref{block}.

\specop{labels}
%
\Syntax
\defSyntax{labels}{
\begin{syntax}
    \scdef{labels-form}: \ra{} \classref{object} \\
    \>  ( \specopref{labels} \\
    \>\>  ( \scseqref{function-definition} ) \\
    \>\>  \scref{labels-body} ) \\
    \scdef{function-definition}: \\
    \>  ( \scref{identifier} \scref{lambda-list} \scref{body} ) \\
    \scdef{labels-body}: \\
    \>  \scseqref{form}
\end{syntax}}%
\showSyntaxBox{labels}
%
\begin{arguments}
    \item[identifier] A symbol naming a new inner-lexical binding to be
    initialized with the function having the \scref{lambda-list} and \scref{body}
    specified.

    \item[lambda-list] The parameter list of the function conforming to the
    syntax specified below.

    \item[body] A sequence of forms.

    \item[labels-body] A sequence of forms.
\end{arguments}
%
\result%
The \specopref{labels} operator provides for local mutually recursive function
creation.  Each \scref{identifier} is bound to a new inner-lexical binding
initialized with the function constructed from \scref{lambda-list} and
\scref{body}.  The scope of the \scref{identifier}s is the entire
\specopref{labels} form\index{general}{scope and extent!in \specopref{labels}
    expressions}.  The \scref{lambda-list} is either a single variable or a list
of variables---see \specopref{lambda}.  Each form in \scref{labels-body} is
evaluated in order in the lexical environment extended with the bindings of the
\scref{identifier}s.  The result of evaluating the last form in
\scref{labels-body} is returned as the result of the \specopref{labels} form.

\specop{let}
%
\Syntax
\defSyntax{let}{
\begin{syntax}
    \scdef{let-form}: \ra{} \classref{object} \\
    \>  ( \specopref{let} \scoptref{identifier} ( \scseqref{binding} ) \\
    \>\>  \scref{body} ) \\
    \scdef{binding}: \\
    \>  \scref{variable} \\
    \> ( \scref{variable} \scref{form} ) \\
    \scdef{variable}: \\
    \> \scref{identifier}
    \scdef{var}: \\
    \> \scref{variable}
\end{syntax}}%
\showSyntaxBox{let}
%
\remarks%
The optional \scref{identifier}\/ denotes that the let form can be called from
within its \scref{body}.  This is an abbreviation for \specopref{labels} form in
which \scref{identifier} is bound to a function whose parameters are the
identifiers of the \scref{binding}s of the \specopref{let}, whose body is that of
the \specopref{let} and whose initial call passes the values of the initializing
form of the \scref{binding}s.  A binding is specified by either an identifier or
a two element list of an identifier and an initializing form.  All the
initializing forms are evaluated in order from left to right in the current
environment and the variables named by the identifiers in the \scref{binding}s
are bound to new locations holding the results.  Each form in \scref{body} is
evaluated in order in the environment extended by the above bindings.  The
result of evaluating the last form in \scref{body} is returned as the result of
the \specopref{let} form.
%
\rewriterules
%
\begin{RewriteTable}{let}{lll}
    (\specopref{let} () \scref{body}) &\rewrite& (\specopref{progn} \scref{body}) \\
\begin{minipage}[t]{\columnwidth}
\begin{tabbing}
    (\specopref{let} (\=(\scref{var}$_1$ \scref{form}$_1$) \\
    \>(\scref{var}$_2$ \scref{form}$_2$) \\
    \>\scref{var}$_3$ \\
    \>\ldots)\\
    00\= \kill
    \>\scref{body})
\end{tabbing}
\end{minipage}
&\rewrite&
\begin{minipage}[t]{\columnwidth}
\begin{tabbing}
    00\=00\=00\= \kill
    ((\specopref{lambda} (\scref{var}$_1$ \scref{var}$_2$ \scref{var}$_3$ \ldots) \\
    \>\>\scref{body}) \\
    \>\scref{form}$_1$ \scref{form}$_2$ () \ldots) \\
\end{tabbing}
\end{minipage}\\
\begin{minipage}[t]{\columnwidth}
\begin{tabbing}
    (\specopref{let} \= \scref{var}$_0$ \\
    \>(\=(\scref{var}$_1$ \scref{form}$_1$) \\
    \>\>\scref{var}$_2$ \\
    \>\>\ldots)\\
    00\= \kill
    \>\scref{body})
\end{tabbing}
\end{minipage}
&\rewrite&
\begin{minipage}[t]{\columnwidth}
\begin{tabbing}
    00\= \kill
    (\specopref{labels}\\
    \>(\=(\scref{var}$_0$ (\scref{var}$_1$ \scref{var}$_2$ \ldots) \\
    \>\>\scref{body})) \\
    \>(\scref{var}$_0$ \scref{form}$_1$ () \ldots))
\end{tabbing}%
\end{minipage}%
\end{RewriteTable}

\specop{let*}
%
\Syntax
\defSyntax{let*}{
\begin{syntax}
    \scdef{let-star-form}: \ra{} \classref{object} \\
    \>  ( \specopref{let*} ( \scseqref{binding} ) \\
    \>\>  \scref{body} )
\end{syntax}}%
\showSyntaxBox{let*}
%
\remarks%
A \scref{binding} is specified by a two element list of a variable and an
initializing form.  The first initializing form is evaluated in the current
environment and the corresponding variable is bound to a new location containing
that result.  Subsequent bindings are processed in turn, evaluating the
initializing form in the environment extended by the previous binding.  Each
form in \scref{body} is evaluated in order in the environment extended by the
above bindings.  The result of evaluating the last form is returned as the
result of the \specopref{let*} form.
%
\rewriterules
%
\begin{RewriteTable}{let*}{lll}
    (\specopref{let*} () \scref{body}) &\rewrite& (progn \scref{body}) \\
\begin{minipage}[t]{\columnwidth}
\begin{tabbing}
    (\specopref{let*} (\=(\scref{var}$_1$ \scref{form}$_1$) \\
    \>(\scref{var}$_2$ \scref{form}$_2$) \\
    \>\scref{var}$_3$\\
    \>\ldots) \\
    00\= \kill
    \>\scref{body})
\end{tabbing}%
\end{minipage}%
&\rewrite&
\begin{minipage}[t]{\columnwidth}%
\begin{tabbing}%
    00\= \kill
    (\specopref{let} ((\scref{var}$_1$ \scref{form}$_1$)) \\
    \>(\specopref{let*} (\=(\scref{var}$_2$ \scref{form}$_2$) \\
    \>\>\scref{var}$_3$\\
    \>\>\ldots) \\
    00\=00\= \kill
    \>\>\scref{body}))
\end{tabbing}%
\end{minipage}%
\end{RewriteTable}

\specop{progn}
%
\Syntax
\defSyntax{progn}{
\begin{syntax}
    \scdef{progn-form}: \ra{} \classref{object} \\
    \>  ( \specopref{progn} \scref{body} )
\end{syntax}}%
\showSyntaxBox{progn}
%
\begin{arguments}
    \item[form\/$^*$] A sequence of forms and in certain circumstances, defining
    forms.
\end{arguments}
%
\result%
The sequence of \scref{form}s is evaluated from left to right, returning the
value of the last one as the result of the \specopref{progn} form.  If the
sequence of forms is empty, \specopref{progn} returns \nil{}.
%
\remarks%
If the \specopref{progn} form occurs enclosed only by \specopref{progn} forms
and a \syntaxref{defmodule} form, then the \scref{form}s within the
\specopref{progn} can be defining forms, since they appear in the top-lexical
environment.  It is a violation for defining forms to appear in inner-lexical
environments.

% \specop{multiple-value-bind}{lambda-list form body}{obj}
% \gdef\multipleValueBind{\verb+(multiple-value-bind+ lambda-list form
%     body\verb+)+} The \scref{form} is evaluated and the elements of
% \scref{lambda-list} are bound to the resulting values.  The cardinality of the
% resulting values might not agree with the arity of the {\tt
%     multiple-value-bind} \scref{lambda-list}.  If the number of values is less
% than the arity, the excess formal parameters receive the value \nil{}.  It is
% an error if the number of values is greater than the arity.  A conforming
% processor must support a minimum {\tt multiple-value-bind} arity of 15.  If
% \scref{lambda-list} is an \scref{identifier}, it is bound to a newly allocated
% list of the multiple values.  If \scref{lambda-list} is a \scref{rest-list},
% each \scref{identifier} preceding the dot is bound to its corresponding value
% in the multiple value and the \scref{identifier} succeeding the dot is bound
% to a newly allocated list of the remaining elements of the multiple value.
% These bindings have lexical scope and extent.  Each form in \scref{body} is
% evaluated in order in the environment extended by the above bindings.
%
% \standard{values}{obj$_1$ ... obj$_n$}{multiple-value(obj)} The values {\em
%     obj$_1$} to {\em obj$_n$} are passed to the dynamically closest enclosing
% instance of {\tt multiple-value-bind}.  If there is no such enclosing {\tt
%     multiple-value-bind}, the excess values are discarded and only {\em
%     obj$_1$} is returned.

\specop{unwind-protect}
%
\Syntax
\defSyntax{unwind-protect}{
\begin{syntax}
    \scdef{unwind-protect-form}: \ra{} \classref{object} \\
    \>  (\specopref{unwind-protect} \scref{protected-form} \\
    \>\>  \scseqref{after-form} ) \\
    \scdef{protected-form}: \\
    \> \scref{form} \\
    \scdef{after-form}: \\
    \> \scref{form}
 \end{syntax}}%
\showSyntaxBox{unwind-protect}
%
\begin{arguments}
    \item[\scdef{protected-form}] A form.
    \item[\scseqref{after-form}] A sequence of forms.
\end{arguments}
%
\result%
The value of \scref{protected-form}.
%
\remarks%
The normal action of \specopref{unwind-protect} is to process
\scref{protected-form} and then each of \scref{after-form}s in order, returning
the value of \scref{protected-form} as the result of \specopref{unwind-protect}.
A non-local exit from the dynamic extent of \scref{protected-form}, which can be
caused by processing a non-local exit form, will cause each of
\scref{after-form}s to be processed before control goes to the continuation
specified in the non-local exit form.  The \scref{after-form}s are not protected
in any way by the current \specopref{unwind-protect}.  Should any kind of
non-local exit occur during the processing of the \scref{after-form}s, the
\scref{after-form}s being processed are not reentered.  Instead, control is
transferred to wherever specified by the new non-local exit but the
\scref{after-form}s of any intervening \specopref{unwind-protect}s between the
dynamic extent of the target of control transfer and the current
\specopref{unwind-protect} are evaluated in increasing order of dynamic extent.
%
\examples
%
\begin{example}
\label{example:unwind-loop}
\examplecaption{Interaction of \specopref{unwind-protect} with non-local exits}
{\codeExample
(progn
  (let/cc k1
    (labels
      ((loop
         (let/cc k2
           (unwind-protect (k1 10) (k2 99))
         ;; continuation bound to k2
         (loop))))
      (loop)))
  ;; continuation bound to k1
  ...)
\endCodeExample}
\end{example}
%
The code fragment in example~\ref{example:unwind-loop} illustrates both the use
of \specopref{unwind-protect} and of a difference between the semantics of
\eulisp\ and some other Lisps.  Stepping through the evaluation of this form:
{\tt k1} is bound to the continuation of its \specopref{let/cc} form; a
recursive function named {\tt loop} is constructed, {\tt loop} is called from
the body of the \specopref{labels} form; {\tt k2} is bound to the continuation
of its \specopref{let/cc} form; \specopref{unwind-protect} calls {\tt k1}; the
after forms of \specopref{unwind-protect} are evaluated in order; {\tt k2} is
called; {\tt loop} is called; etc..  This program loops indefinitely.
%
\end{optDefinition}

\sclause{Quasiquotation Expressions}
\gdef\module{syntax-0}
\label{backquote}
\index{general}{quasiquotation}
\index{general}{backquoting}
%
\begin{optPrivate}
    Should it be an error to splice an improper list even if its the last item?
\end{optPrivate}
%
\begin{optDefinition}

\specop{quasiquote}
%
\Syntax
\defSyntax{quasiquote}{
\begin{syntax}
    \scdef{quasiquote-form}: \ra{} \classref{object} \\
    \> ( \specopref{quasiquote} \scref{skeleton} ) \\
    \> \syntaxref{`}\scref{skeleton} \\
    \scdef{skeleton}: \\
    \>  \scref{form}
\end{syntax}}%
\showSyntaxBox{quasiquote}
%
\remarks%
Quasiquotation is also known as backquoting.  A \specopref{quasiquote}d
expression is a convenient way of building a structure.  The \scref{skeleton}
describes the shape and, generally, many of the entries in the structure but
some holes remain to be filled.  The \specopref{quasiquote} macro can be
abbreviated by using the glyph called {\em grave accent} (\syntaxref{`})
\ttsubindex{quasiquote}{abbreviation with \syntaxref{`}}, so that {\tt
    (\specopref{quasiquote} \scref{skeleton})} can be written
\syntaxref{`}\scref{skeleton}.

\syntaxform{`}
%
\remarks%
See \specopref{quasiquote}.

\specop{unquote}
\Syntax
\defSyntax{unquote}{
\begin{syntax}
    \scdef{unquote-form}: \ra{} \classref{object} \\
    \>  ( \specopref{unquote} \scref{form} ) \\
    \>  \syntaxref{,}\scref{form}
\end{syntax}}%
\showSyntaxBox{unquote}
%
\remarks%
See \specopref{unquote-splicing}.

\syntaxform{,}
%
\remarks%
See \specopref{unquote}.

\specop{unquote-splicing}
\Syntax
\defSyntax{unquote-splicing}{
\begin{syntax}
    \scdef{unquote-splicing-form}: \ra{} \classref{object} \\
    \>  ( \specopref{unquote-splicing} \scref{form} ) \\
    \> \syntaxref{,@}\scref{form}
\end{syntax}}%
\showSyntaxBox{unquote-splicing}
%
\remarks%
The holes in a \specopref{quasiquote}d expression are identified by unquote
expressions of which there are two kinds---forms whose value is to be inserted
at that location in the structure and forms whose value is to be spliced into
the structure at that location.  The former is indicated by an
\specopref{unquote} expression and the latter by an \specopref{unquote-splicing}
expression.  In \specopref{unquote-splicing} the \scref{form} must result in a
proper list.  The insertion of the result of an unquote-splice expression is as
if the opening and closing parentheses of the list are removed and all the
elements of the list are appended in place of the unquote-splice expression.
%An
%error is signalled (condition class: {\tt
%<unquote-no-context>}\indexcondition{unquote-no-context}) if either of
%these syntaxes occurs outside the scope of a \specopref{quasiquote}
%expression.

The syntax forms \specopref{unquote} and \specopref{unquote-splicing} can be
abbreviated respectively by using the glyph called {\em comma} (\syntaxref{,})
\ttsubindex{unquote}{abbreviation with \syntaxref{,}} preceding an expression
and by using the diphthong {\em comma} followed by the glyph called {\em
    commercial at} (\syntaxref{,@}) \ttindex{unquote-splicing!abbreviation to
    ,@} preceding a form.  Thus, {\tt (\specopref{unquote} a)} may be written
{\tt \syntaxref{,}a} and {\tt (\specopref{unquote-splicing} a)} can be written
{\tt \syntaxref{,@}a}.
%
\examples
{\tt
\begin{tabular}{lll}
    `(a ,(list 1 2) b) & $\rightarrow$ & (a (1 2) b)\\
    `(a ,@(list 1 2) b) & $\rightarrow$ & (a 1 2 b)
\end{tabular}}
%
\end{optDefinition}

\syntaxform{,@}
%
\remarks%
See \specopref{unquote-splicing}.

\sclause{Summary of Level-0 Defining, Special and Function-call Forms}
%
\begin{optDefinition}
%
\raggedbottom
%
This section gives the syntax of the character-set, comments and all level-0
forms starting with modules.  The syntax of data objects is given in the section
pertaining to the class and is summarized in
section~\ref{object-syntax-summary}.

\showSyntax{character-set}
\showSyntax{comment}

\ssclause{Syntax of Level-0 modules}
%
\showSyntax{defmodule-0}

\ssclause{Syntax of Level-0 defining forms}
%
\showSyntax{defining-0-forms}
\showSyntax{defclass}
\showSyntax{defgeneric}
\showSyntax{defmethod}
\showSyntax{defconstant}
\showSyntax{deflocal}
\showSyntax{defmacro}
\showSyntax{defun}
\showSyntax{defcondition}

\ssclause{Syntax of Level-0 special forms}
%
\showSyntax{special-0-forms}
\showSyntax{generic-lambda}
\showSyntax{lambda}
\showSyntax{quote}
\showSyntax{setq}
\showSyntax{if}
\showSyntax{cond}
\showSyntax{when}
\showSyntax{unless}
\showSyntax{and}
\showSyntax{or}
\showSyntax{let/cc}
\showSyntax{labels}
\showSyntax{progn}
\showSyntax{unwind-protect}
\showSyntax{apply}
\showSyntax{call-next-handler}
\showSyntax{with-handler}
\showSyntax{block}
\showSyntax{return-from}
\showSyntax{let}
\showSyntax{let*}
\showSyntax{quasiquote}
\showSyntax{unquote}
\showSyntax{unquote-splicing}
\showSyntax{with-input-file}
\showSyntax{with-output-file}
\showSyntax{with-source}
\showSyntax{with-sink}

\ssclause{Syntax of Level-0 function calls}
%
\showSyntax{function-call}
%
\flushbottom
%
\end{optDefinition}

\gdef\module{level-0}\newpage\input{condition}
\gdef\module{level-0}\newpage\input{thread}

% -----------------------------------------------------------------------
%%% Level-0 Module Library
\clause{Level-0 Module Library}
\label{section:level-0}
This section describes the classes required at level-0 and the operations
defined on instances of those classes.  The
section is organized by module in alphabetical order.  These sub-sections
contain information about the predefined classes in \eulisp\ that are necessary
to make the language usable.

\gdef\module{level-0}\sclause{Characters}
\label{character}
\index{general}{character}
\index{general}{level-0 modules!character}
\index{general}{character!module}
\begin{optDefinition}
%
The defined name of this module is {\tt character}.

\syntaxform{character}
%
Character literals\index{general}{literal!character} are denoted by the {\em
    extension\/} glyph, called {\em hash} (\verb+#+), followed by the {\em
    character-extension\/}\index{general}{character!character-extension glyph}
glyph, called {\em reverse solidus\/} (\verb+\+), followed by the name of the
character.  The syntax for the external representation of characters is defined
in syntax table~\ref{character-syntax}.  For most characters, their name is the
same as the glyph associated with the character, for example: the character
``a'' has the name ``a'' and has the external representation \verb+#\a+.
Certain characters in the group named {\em special\/} (see
table~\ref{character-set} and also syntax table~\ref{character-syntax}) form the
syntax category \scref{special-character-token} and are referred to using the
digrams defined in table~\ref{character-digrams}.
%
\begin{table}[h]
\label{character-digrams}
\caption{Character digrams}%
\begin{center}
\begin{tabular}{|ll|}\hline
    Operation & Digram \\
    \hline
    alert & \verb+\a+ \\
    backspace & \verb+\b+ \\
    delete & \verb+\d+ \\
    formfeed & \verb+\f+ \\
    linefeed & \verb+\l+ \\
    newline & \verb+\n+ \\
    return & \verb+\r+ \\
    tab & \verb+\t+ \\
    vertical tab & \verb+\v+ \\
    hex-insertion & \verb+\x+ \\
    string delimiter & \verb+\"+ \\
    string escape & \verb+\\+ \\
    \hline
\end{tabular}
\end{center}
\end{table}
%
Any character which does not have an external representation dealt with by cases
described so far is represented by the digram \verb+#\x+ (see
table~\ref{character-digrams}) followed four hexadecimal digits.  The value of
the hexadecimal number represents the position of the character in the current
character set.  Examples of such character literals are \verb+#\x0000+ and
\verb+#\xabcd+, which denote, respectively, the characters at position 0 and at
position 43981 in the character set current at the time of reading or writing.
The syntax for the external representation of characters is defined in syntax
table~\ref{character-syntax} below:
%
\Syntax
\label{character-syntax}
\defSyntax{character}{
\begin{syntax}
    \scdef{character}: \\
    \>  \scref{literal-character-token} \\
    \>  \scref{special-character-token} \\
    \>  \scref{numeric-character-token} \\
    \scdef{literal-character-token}: \\
    \>  \#\textbackslash{}\scref{letter} \\
    \>  \#\textbackslash{}\scref{decimal-digit} \\
    \>  \#\textbackslash{}\scref{other-character} \\
    \>  \#\textbackslash{}\scref{special-character} \\
    \scdef{special-character-token}: \\
    \>  \#\textbackslash{}\textbackslash{}a \\
    \>  \#\textbackslash{}\textbackslash{}b \\
    \>  \#\textbackslash{}\textbackslash{}d \\
    \>  \#\textbackslash{}\textbackslash{}f \\
    \>  \#\textbackslash{}\textbackslash{}l \\
    \>  \#\textbackslash{}\textbackslash{}n \\
    \>  \#\textbackslash{}\textbackslash{}r \\
    \>  \#\textbackslash{}\textbackslash{}t \\
    \>  \#\textbackslash{}\textbackslash{}v \\
    \>  \#\textbackslash{}\textbackslash{}" \\
    \>  \#\textbackslash{}\textbackslash{}\textbackslash{} \\
    \scdef{numeric-character-token}: \\
    \>  \#\textbackslash{}\textbackslash{}x
        \scref{hexadecimal-digit} \scref{hexadecimal-digit} \\
    \>\>  \scref{hexadecimal-digit} \scref{hexadecimal-digit}
\end{syntax}}%
\showSyntaxBox{character}

%
\begin{note}
    This text refers to the ``current character set'' but defines no means of
    selecting alternative character sets.  This is to allow for future
    extensions and implementation-defined extensions which support more than one
    character set.
\end{note}

\derivedclass{character}{object}
\index{general}{level-0 classes!\theclass{character}}
%
The class of all characters.

\function{characterp}
%
\begin{arguments}
    \item[{object}] Object to examine.
\end{arguments}
%
\result%
Returns {\em object\/} if it is a character, otherwise \nil{}.

\method{equal}
%
\begin{specargs}
    \item[character$_1$, \classref{character}] A character.
    \item[character$_2$, \classref{character}] A character.
\end{specargs}
%
\result%
If {\em character$_1$\/} is the same character as {\em character$_2$\/} the
result is {\em character$_1$}, otherwise the result is \nil{}.

\method{binary<}
%
\begin{specargs}
    \item[character$_1$, \classref{character}] A character.
    \item[character$_2$, \classref{character}] A character.
\end{specargs}
%
\result%
If both characters denote uppercase alphabetic or both denote lowercase
alphabetic, the result is defined by alphabetical order.  If both characters
denote a digit, the result is defined by numerical order.  In these three cases,
if the comparison is true, the result is {\em character$_1$}, otherwise it is
\nil{}.  Any other comparison is an error and the result of such comparisons is
undefined.
%
\examples
\begin{tabular}{lcl}
    \verb+(binary< #\A #\Z)+ & \Ra & \verb+#\A+\\
    \verb+(binary< #\a #\z)+ & \Ra & \verb+#\a+\\
    \verb+(binary< #\0 #\9)+ & \Ra & \verb+#\0+\\
    \verb+(binary< #\A #\a)+ & \Ra & {\em undefined}\\
    \verb+(binary< #\A #\0)+ & \Ra & {\em undefined}\\
    \verb+(binary< #\a #\0)+ & \Ra & {\em undefined}\\
\end{tabular}
%
\seealso%
Method on \genericref{binary<} for \hyperref[string]{strings}.

%\method{binary=}
%
%\begin{specargs}
%\item[c$_1$, \classref{character}] A character.
%\item[c$_2$, \classref{character}] A character.
%\end{specargs}
%
%\result%
%If {\em c$_1$} and {\em c$_2$} denote the same character, the result
%is {\em c$_1$}, otherwise it is \nil{}.

\generic{as-lowercase}
%
\begin{genericargs}
    \item[object, \classref{object}] An object to convert to lower case.
\end{genericargs}
%
\result%
An instance of the same class as {\em object\/} converted to lower case
according to the actions of the appropriate method for the class of {\em
    object}.
%
\seealso%
Another method is defined on \genericref{as-lowercase} for
\hyperref[string]{strings}.

\method{as-lowercase}
%
\begin{specargs}
    \item[character, \classref{character}] A character.
\end{specargs}
%
\result%
If {\em character\/} denotes an upper case character, a character denoting its
lower case counterpart is returned.  Otherwise the result is the argument.

\generic{as-uppercase}
%
\begin{genericargs}
    \item[object, \classref{object}] An object to convert to upper case.
\end{genericargs}
%
\result%
An instance of the same class as {\em object\/} converted to upper case
according to the actions of the appropriate method for the class of {\em
    object}.
%
\seealso%
Another method is defined on \genericref{as-uppercase} for
\hyperref[string]{strings}.

\method{as-uppercase}
%
\begin{specargs}
    \item[character, \classref{character}] A character.
\end{specargs}
%
\result%
If {\em character\/} denotes an lower case character, a character denoting its
upper case counterpart is returned.  Otherwise the result is the argument.

\method{generic-prin}
%
\begin{specargs}
    \item[character, \classref{character}] Character to be ouptut on {\em
        stream}.
    \item[stream, \classref{stream}] Stream on which {\em character\/} is to be
    ouptut.
\end{specargs}
%
\result%
The character {\em character}.
%
\remarks%
Output the interpretation of {\em character\/} on {\em stream}.

\method{generic-write}
%
\begin{specargs}
    \item[character, \classref{character}] Character to be ouptut on {\em
        stream}.
    \item[stream, \classref{stream}] Stream on which {\em character\/} is to be
    ouptut.
\end{specargs}
%
\result%
The character {\em character}.
%
\remarks%
Output external representation of {\em character\/} on {\em stream\/} in the
format \verb+#\+{\em{}name\/} as described at the beginning of this section.
%
\end{optDefinition}

\gdef\module{level-0}\newpage\input{collection}
\gdef\module{level-0}\newpage\sclause{Comparison}
\label{compare}
\index{general}{level-0 modules!compare}
\index{general}{compare!module}
%
\begin{optDefinition}
%
The defined name of this module is {\tt compare}.  There are three binary
functions for comparing objects for equality, \functionref{eq},
\functionref{eql}, \genericref{binary=} and the n-ary \functionref{=} which uses
\genericref{binary=}.  The three binary functions are related in the following
way:
%
\begin{center}
\begin{tabular}{rcccl}
    {\tt (eq {\em a} {\em b})} & $\Rightarrow$ & {\tt (eql {\em a} {\em
            b})} & $\Rightarrow$ & {\tt (binary= {\em a} {\em b})}\\
    {\tt (eq {\em a} {\em b})} & $\not\Leftarrow$ & {\tt (eql {\em a} {\em
            b})} & $\not\Leftarrow$ & {\tt (binary= {\em a} {\em b})}\\
\end{tabular}
\end{center}
%
There are four n-ary function for comparing objects by order, \functionref{<}
and \functionref{>} which are implemented by the generic function
\genericref{binary<}, \functionref{<=} and \functionref{>=} which are
implemented by the generic functions \genericref{binary<} and
\genericref{binary=}.  There is also one binary function for comparing objects
for inequality, \functionref{!=}.  A summary of the comparison functions and the
classes for which they have defined behaviour is given below:

\framebox[\linewidth]
{
\begin{tabular*}{\linewidth}{ll}
    \functionref{eq}: & \classref{object}$\times$\classref{object} \\
    \hline
    \functionref{eql}: & \classref{object}$\times$\classref{object}
    \Ra \functionref{eq} \\
    & \classref{character}$\times$\classref{character} \\
    & \classref{fixed-precision-integer}$\times$ \\
    & \classref{fixed-precision-integer} \\
    & \classref{double-float}$\times$\classref{double-float}
    \Ra \genericref{binary=}\\
    \hline
    \genericref{binary=}: & \classref{object}$\times$\classref{object} \\
    & \classref{character}$\times$\classref{character} \\
    & \classref{null}$\times$\classref{null} \\
    & \classref{number}$\times$\classref{number}
    \Ra \functionref{eql}\\
    & \classref{fixed-precision-integer}$\times$ \\
    & \classref{fixed-precision-integer} \\
    & \classref{double-float}$\times$\classref{double-float} \\
    & \classref{double-float}$\times$\classref{fixed-precision-integer} \\
    & \classref{fixed-precision-integer}$\times$\classref{double-float} \\
    & \classref{cons}$\times$\classref{cons} \\
    & \classref{string}$\times$\classref{string} \\
    & \classref{vector}$\times$\classref{vector} \\
    \hline
    \genericref{binary<}: & \classref{character}$\times$\classref{character} \\
    & \classref{symbol}$\times$\classref{symbol} \\
    & \classref{fixed-precision-integer}$\times$ \\
    & \classref{fixed-precision-integer} \\
    & \classref{double-float}$\times$\classref{double-float} \\
    & \classref{string}$\times$\classref{string} \\
    \hline
    \functionref{=}: & \classref{object}$\times$\classref{object}
    \Ra \genericref{binary=} \\
    \functionref{!=}: & \classref{object}$\times$\classref{object} \\
    \functionref{<}: & \classref{object}$\times$\classref{object}
    \Ra \genericref{binary<} \\
    \functionref{>}: & \classref{object}$\times$\classref{object} \\
    \functionref{<=}: & \classref{object}$\times$\classref{object} \\
    \functionref{>=}: & \classref{object}$\times$\classref{object}
\end{tabular*}
}

\function{eq}
%
\begin{arguments}
    \item[object$_1$] An object.
    \item[object$_2$] An object.
\end{arguments}
%
\result%
Compares {\em object$_1$} and {\em object$_2$} and returns \true{} if they are
the {\em same\/} object, otherwise \nil{}.  {\em Same\/} in this context means
``identifies the same memory location''.
%
\remarks%
In the case of numbers and characters the behaviour of \functionref{eq} might
differ between processors because of implementation choices about internal
representations.  Therefore, \functionref{eq} might return \true{} or \nil{} for
numbers which are \functionref{=} and similarly for characters which are
\functionref{eql}, depending on the implementation
\ttsubindex{eq}{implementation-defined behaviour}
\index{general}{implementation-defined!behaviour of \functionref{eq}}.
%
\examples%
\begin{tabular}{lcl}
    \verb+(eq () nil)+ & \Ra & \verb+t+\\
    \verb+(eq '() 'nil)+ & \Ra & \verb+t+\\
    \verb+(eq 'a 'a)+ & \Ra & \verb+t+\\
    \verb+(eq 'a 'b)+ & \Ra & \verb+()+\\
    \verb+(eq #\a #\a)+ & \Ra & \verb+t+ or \verb+()+\\
    \verb+(eq 3 3)+ & \Ra & \verb+t+ or \verb+()+\\
    \verb+(eq 3 3.0)+ & \Ra & \verb+()+\\
    \verb+(eq 3.0 3.0)+ & \Ra & \verb+t+ or \verb+()+\\
    \verb+(eq (cons 'a 'b) (cons 'a 'c))+ & \Ra & \verb+()+\\
    \verb+(eq (cons 'a 'b) (cons 'a 'b))+ & \Ra & \verb+()+\\
    \verb+(eq '(a . b) '(a . b))+ & \Ra & \verb+t+ or \verb+()+\\
    \verb+(let ((x (cons 'a 'b))) (eq x x))+ & \Ra & \verb+t+\\
    \verb+(let ((x '(a . b))) (eq x x))+ & \Ra & \verb+t+\\
    \verb+(eq "string" "string")+ & \Ra & \verb+t+ or \verb+()+\\
    \verb+(eq #('a 'b) #('a 'b))+ & \Ra & \verb+t+ or \verb+()+\\
    \verb+(let ((x #('a 'b))) (eq x x))+ & \Ra & \verb+t+\\
\end{tabular}

\function{eql}
%
\begin{arguments}
    \item[object$_1$] An object.
    \item[object$_2$] An object.
\end{arguments}
%
\result%
If the class of {\em object$_1$} and of {\em object$_2$} is the same and is a
subclass of \classref{character} or \classref{number}, the result is that of
comparing them under \methodref{binary=}{character} or
\methodref{binary=}{number} respectively.  Otherwise the result is that of
comparing them under \functionref{eq}.
%
\examples%
Given the same set of examples as for \functionref{eq}, the same result is
obtained except in the following cases:

\begin{tabular}{lcl}
    \verb+(eql #\a #\a)+ & \Ra & \verb+t+\\
    \verb+(eql 3 3)+ & \Ra & \verb+t+\\
    \verb+(eql 3.0 3.0)+ & \Ra & \verb+t+\\
\end{tabular}

\generic{binary=}
%
\begin{arguments}
    \item[object$_1$, \classref{object}] An object.
    \item[object$_2$, \classref{object}] An object.
\end{arguments}
%
\result%
Returns \true{} or \nil{} according to the method for the class(es) of {\em
    object$_1$} and {\em object$_2$}. It is an error if either or both of the
arguments is self-referential.
%
\seealso%
Class specific methods on \genericref{binary=} are defined for
\classref{character}, \classref{list}, \classref{number} (with specialisations
for \classref{fixed-precision-integer} and \classref{double-float}),
\classref{string}, \classref{vectors}.  All other cases are handled by the
default method defined for \classref{object}:

\method{binary=}{object}
%
\begin{specargs}
    \item[object$_1$, \classref{object}] An object.
    \item[object$_2$, \classref{object}] An object.
\end{specargs}
%
\result%
The result is as if \functionref{eql} had been called with the arguments
supplied.

\generic{binary<}
%
\begin{genericargs}
    \item[object$_1$, \classref{object}] An object.
    \item[object$_2$, \classref{object}] An object.
\end{genericargs}
%
\result%
The first argument if it is less than the second, according to the method for
the class of the arguments, otherwise \nil{}.
%
\seealso%
Class specific methods on \genericref{binary<} are defined for
\classref{character}, \classref{string}, \classref{fixed-precision-integer} and
\classref{double-float}.

\function{=}
%
\begin{arguments}
    \item[{number$_1$ \ldots}] A non-empty sequence of numbers.
\end{arguments}
%
\result%
Given one argument the result is \true{}.  Given more than one argument the
result is determined by \genericref{binary=}, returning \true{} if all the
arguments are the same, otherwise \nil{}.

\function{!=}
%
\begin{arguments}
    \item[{number$_1$ \ldots}] A non-empty sequence of numbers.
\end{arguments}
%
\result%
Given one argument the result is \nil{}.  Given more than one argument the
result is determined by \genericref{binary=}, returning \nil{} if all the
arguments are the same, otherwise \true{}.

\function{<}
%
\begin{arguments}
    \item[object$_1$ \ldots] A non-empty sequence of objects.
\end{arguments}
%
\result%
Given one argument the result is \true{}.  Given more than one argument the
result is \true{} if the sequence of objects {\em object$_1$} up to {\em
    object$_n$} is strictly increasing according to the generic function
\genericref{binary<}.  Otherwise, the result is \nil{}.

\function{>}
%
\begin{arguments}
    \item[object$_1$ \ldots] A non-empty sequence of objects.
\end{arguments}
%
\result%
Given one argument the result is \true{}.  Given more than one argument the
result is \true{} if the sequence of objects {\em object$_1$} up to {\em
    object$_n$} is strictly decreasing according to the generic function
\genericref{binary<} applied to the arguments in reverse order.  Otherwise, the
result is \nil{}.

\function{<=}
%
\begin{arguments}
    \item[object$_1$ \ldots] A non-empty sequence of objects.
\end{arguments}
%
\result%
Given one argument the result is \true{}.  Given more than one argument the
result is \true{} if the sequence of objects {\em object$_1$} up to {\em
    object$_n$} is strictly increasing according to the generic function
\genericref{binary<} and \genericref{binary=}.  Otherwise, the result is \nil{}.

\function{>=}
%
\begin{arguments}
    \item[object$_1$ \ldots] A non-empty sequence of objects.
\end{arguments}
%
\result%
Given one argument the result is \true{}.  Given more than one argument the
result is \true{} if the sequence of objects {\em object$_1$} up to {\em
    object$_n$} is strictly decreasing according to the generic function
\genericref{binary<} and \genericref{binary=} applied to the arguments in
reverse order.  Otherwise, the result is \nil{}.

\function{max}
%
\begin{arguments}
    \item[object$_1$ \ldots] A non-empty sequence of objects.
\end{arguments}
%
\result%
The maximal element of the sequence of objects {\em object$_1$} up to {\em
    object$_n$} using the generic function \genericref{binary<}.  Zero arguments
is an error.  One argument returns {\em object$_1$}.

\function{min}
%
\begin{arguments}
    \item[object$_1$ \ldots] A non-empty sequence of objects.
\end{arguments}
%
\result%
The minimal element of the sequence of objects {\em object$_1$} up to {\em
    object$_n$} using the generic function \genericref{binary<}.  Zero arguments
is an error.  One argument returns {\em object$_1$}.
%
\end{optDefinition}

\gdef\module{level-0}\newpage\sclause{Conversion}
\label{convert}
\index{general}{level-0 modules!convert}
\index{general}{convert!module}
%
\begin{optDefinition}
The defined name of this module is {\tt convert}.

The mechanism for the conversion of an instance of one class to an instance of
another is defined by a user-extensible framework which has some similarity to
the {\tt setter} mechanism.

To the user, the interface to conversion is via the function {\tt convert},
which takes an object and some class to which the object is to be converted.
The target class is used to access an associated {\em converter\/} function, in
fact, a generic function, which is applied to the source instance, dispatching
on its class to select the method which implements the appropriate conversion.
Thus, having defined a new class to which it may be desirable to convert
instances of other classes, the programmer defines a generic function:

{\codeExample
(defgeneric (converter \[{\em new-class}\]) (instance))
\endCodeExample}

Hereafter, new converter methods may be defined for {\em new-class\/}
using a similar extended syntax for {\tt defmethod}:

{\codeExample
(defmethod (converter \[{\em new-class}\])
           ((instance \[{\em other-class}\])))
\endCodeExample}

The conversion is implemented by defining methods on the converter for {\em
    new-class\/} which specialize on the source class.  This is also how methods
are documented in this text: by an entry for a method on the converter function
for the target class.  In general, the method for a given source class is
defined in the section about that class, for example, converters from one kind
of collection to another are defined in section~\ref{collection}, converters
from string in section~\ref{string}, etc..

\function{convert}
%
\begin{arguments}
    \item[object] An instance of some class to be converted to an instance of
    {\em class}.
    %
    \item[class] The class to which {\em object\/} is to be converted.
\end{arguments}
%
\result%
Returns an instance of {\em class\/} which is equivalent in some class-specific
sense to {\em object\/}, which may be an instance of any type.  Calls the
converter function associated with {\em class\/} to carry out the conversion
operation.  An error is signalled (condition: \conditionref{no-converter}
\indexcondition{no-converter}) if there is no associated function.  An error is
signalled (condition: \conditionref{no-applicable-method}
\indexcondition{no-applicable-method}) if there is no method to convert an
instance of the class of {\em object\/} to an instance of {\em class\/}.

\condition{conversion-condition}{condition}
%
This is the general condition class for all conditions arising from conversion
operations.
%
\begin{initoptions}
    \item[source, \classref{object}] The object to be converted into an instance
    of {\em target-class}.
    %
    \item[target-class, \classref{class}] The target class for the conversion
    operation.
\end{initoptions}
%
\remarks%
Should be signalled by {\tt convert} or a converter method.

\condition{no-converter}{conversion-condition}
%
\begin{initoptions}
    \item[source, \classref{object}] The object to be converted into an instance
    of {\em target-class}.
    %
    \item[target-class, \classref{class}] The target class for the conversion
    operation.
\end{initoptions}
%
\remarks%
Should be signalled by {\tt convert} if there is no associated function.

\function{converter}
%
\begin{arguments}
    \item[target-class] The class whose set of conversion methods is required.
\end{arguments}
%
\result%
The accessor returns the converter function for the class {\em
    target-class}.  The converter is a generic-function with methods specialized
on the class of the object to be converted.

\setter{converter}
%
\begin{arguments}
    \item[target-class] The class whose converter function is to be replaced.
    %
    \item[generic-function] The new converter function.
\end{arguments}
%
\result%
The new converter function.  The setter function replaces the converter
function for the class {\em target-class\/} by {\em generic-function}.  The new
converter function must be an instance of {\tt <generic-function>}.
%
\remarks%
Converter methods from one class to another are defined in the section
pertaining to the source class.
%
\seealso%
Converter methods are defined for collections (\ref{collection}), double
float (\ref{double-float}), fixed precision integer (\ref{fpi}),
string (\ref{string}), symbol (\ref{symbol}), vector (\ref{vector}).
%
\end{optDefinition}

\gdef\module{level-0}\newpage\sclause{Copying}
\label{copy}
\index{general}{level-0 modules!copy}
\index{general}{copy!module}
%
\begin{optPrivate}
    I have a cryptic note in my hardcopy which says {\tt (find-method copy
        integer)}, but I don't know why.
\end{optPrivate}
%
\begin{optDefinition}
\noindent
The defined name of this module is {\tt copy}.

\generic{deep-copy}
%
\begin{genericargs}
    \item[object] An object to be copied.
\end{genericargs}
%
\result%
Constructs and returns a copy of the source which is the same (under some class
specific predicate) as the source and whose slots contain copies of the objects
stored in the corresponding slots of the source, and so on.  The exact behaviour
for each class of {\em object\/} is defined by the most applicable method for
{\em object}.
%
\seealso%
Class specific sections which define methods on \genericref{deep-copy}:
list~(\ref{list}), string~(\ref{string}), table~(\ref{table}) and
vector~(\ref{vector}).

\method{deep-copy}{object}
%
\begin{specargs}
    \item[object, \classref{object}] An object.
\end{specargs}
%
\result%
Returns {\em object}.

\method{deep-copy}{class}
%
\begin{specargs}
    \item[class, \classref{class}] A class.
\end{specargs}
%
\result%
Constructs and returns a new structure whose slots are initialized with copies
(using \genericref{deep-copy}) of the contents of the slots of {\em class}.

\generic{shallow-copy}
%
\begin{genericargs}
    \item[object] An object to be copied.
\end{genericargs}
%
\result%
Constructs and returns a copy of the source which is the same (under
some class specific predicate) as the source.  The exact behaviour for
each class of {\em object\/} is defined by the most applicable method
for {\em object}.
%
\seealso%
Class specific sections which define methods on \genericref{shallow-copy}:
pair~(\ref{pair}), string~(\ref{string}), table~(\ref{table}) and
vector~(\ref{vector}).

\method{shallow-copy}{object}
%
\begin{specargs}
    \item[object, \classref{object}] An object.
\end{specargs}
%
\result%
Returns {\em object}.
%
\method{shallow-copy}{class}
%
\begin{specargs}
    \item[class, \classref{class}] A class.
\end{specargs}
%
\result
Constructs and returns a new structure whose slots are initialized
with the contents of the correpsonding slots of {\em struct}.
%
\end{optDefinition}

\gdef\module{level-0}\newpage\sclause{Double Precision Floats}
\index{general}{double float}
\label{double-float}
\index{general}{level-0 modules!double-float}
\index{general}{double float!module}
%
\begin{optDefinition}
\noindent

The defined name of this module is {\tt double}.  Arithmetic operations for
\classref{double-float}\ are defined by methods on the generic functions defined
in the number module (\ref{number}):
%
\begin{flushleft}
    \genericref{binary+},\ttindex{binary+}\indexmeth{binary+}
    \genericref{binary-},\ttindex{binary-}\indexmeth{binary-}
    \genericref{binary*},\ttindex{binary*}\indexmeth{binary*}
    \genericref{binary/},\ttindex{binary/}\indexmeth{binary/}
    \genericref{binary<},\ttindex{binary<}\indexmeth{binary<}
    \genericref{binary=},\ttindex{binary=}\indexmeth{binary=}
    \genericref{binary-mod},\ttindex{binary-mod}\indexmeth{binary-mod}
    \genericref{negate},\ttindex{negate}\indexmeth{negate}
    \genericref{zerop}\ttindex{zerop}\indexmeth{zerop}
\end{flushleft}
%
\noindent
the float module (\ref{float}):
\begin{flushleft}
\genericref{ceiling},\ttindex{ceiling}\indexmeth{ceiling}
\genericref{floor},\ttindex{floor}\indexmeth{floor}
\genericref{round},\ttindex{round}\indexmeth{round}
\genericref{truncate}\ttindex{truncate}\indexmeth{truncate}
\end{flushleft}
%
\noindent
and the elementary functions module (\ref{elementary-functions}):
\begin{flushleft}
\genericref{acos},\ttindex{acos}\indexmeth{acos}
\genericref{asin},\ttindex{asin}\indexmeth{asin}
\genericref{atan},\ttindex{atan}\indexmeth{atan}
\genericref{atan2},\ttindex{atan2}\indexmeth{atan2}
\genericref{cos},\ttindex{cos}\indexmeth{cos}
\genericref{sin},\ttindex{sin}\indexmeth{sin}
\genericref{tan},\ttindex{tan}\indexmeth{tan}
\genericref{cosh},\ttindex{cosh}\indexmeth{cosh}
\genericref{sinh},\ttindex{sinh}\indexmeth{sinh}
\genericref{tanh},\ttindex{tanh}\indexmeth{tanh}
\genericref{exp},\ttindex{exp}\indexmeth{exp}
\genericref{log},\ttindex{log}\indexmeth{log}
\genericref{log10},\ttindex{log10}\indexmeth{log10}
\genericref{pow},\ttindex{pow}\indexmeth{pow}
\genericref{sqrt}\ttindex{sqrt}\indexmeth{sqrt}
\end{flushleft}
%
\noindent
The behaviour of these functions is defined in the modules noted
above.

\derivedclass{double-float}{float}
\index{general}{level-0 classes!\theclass{double-float}}
%
The class of all double precision floating point numbers.

The syntax for the exponent of a double precision floating point is given below:

\defSyntax{double-float}{
\begin{syntax}
    \scdef{double-exponent}: \\
    \>  d \scoptref{sign} \scref{decimal-integer} \\
    \>  D \scoptref{sign} \scref{decimal-integer}
\end{syntax}}%
\showSyntaxBox{double-float}

The general syntax for floating point numbers is given in syntax
table~\ref{float-syntax}.

\function{double-float-p}
%
\begin{arguments}
    \item[object] Object to examine.
\end{arguments}
%
\result%
Returns {\em object\/} if it is a double float, otherwise \nil{}.
%
\seealso%
\functionref{floatp} (\ref{number}).

\constant{most-positive-double-float}{double-float}
\index{general}{implementation-defined!most positive double precision float}
\index{general}{conformity-clause!most positive double precision float}
%
\remarks%
The value of \constantref{most-positive-double-float} is that positive double
precision floating point number closest in value to (but not equal to) positive
infinity that the processor provides.

\constant{least-positive-double-float}{double-float}
\index{general}{implementation-defined!least positive double precision float}
\index{general}{conformity-clause!least positive double precision float}
%
\remarks%
The value of \constantref{least-positive-double-float} is that positive double
precision floating point number closest in value to (but not equal to) zero that
the processor provides.

\constant{least-negative-double-float}{double-float}
\index{general}{implementation-defined!least negative double precision float}
\index{general}{conformity-clause!least negative double precision float}
%
\remarks%
The value of \constantref{least-negative-double-float} is that negative double
precision floating point number closest in value to (but not equal to) zero that
the processor provides.  Even if the processor provide negative zero, this value
must not be negative zero.

\constant{most-negative-double-float}{double-float}
\index{general}{implementation-defined!most negative double precision float}
\index{general}{conformity-clause!most negative double precision float}
%
\remarks%
The value of \constantref{most-negative-double-float} is that negative double
precision floating point number closest in value to (but not equal to) negative
infinity that the processor provides.

\method{equal}
%
\begin{specargs}
    \item[double$_1$, \classref{double-float}] A double precision float.
    \item[double$_2$, \classref{double-float}] A double precision float.
\end{specargs}
%
\result%
The result of calling \genericref{binary=} on {\em double$_1$} and {\em
    double$_2$}.

\converter{string}
%
\begin{specargs}
    \item[x, \classref{double-float}] A double precision float.
\end{specargs}
%
\result%
Constructs and returns a string, the characters of which correspond to the
external representation of {\em x\/} as produced by {\tt generic-prin}, namely
that specified in the syntax as {\em {\tt[}sign{\tt]}float format 3}.

\converter{fixed-precision-integer}
%
\begin{specargs}
    \item[x, \classref{double-float}] A double precision float.
\end{specargs}
%
\result%
A fixed precision integer.

\remarks%
This function is the same as \genericref{round}.  It is defined for the
sake of symmetry.

\method{generic-prin}
%
\begin{specargs}
    \item[double, \classref{double-float}]%
    The double float to be output on {\em stream}.
    \item[stream, \classref{stream}]%
    The stream on which the representation is to be output.
\end{specargs}
%
\result%
The double float supplied as the first argument.
%
\remarks%
Outputs the external representation of {\em double\/} on {\em stream}, as an
optional sign preceding the syntax defined by {\em float format 3}.  Finer
control over the format of the output of floating point numbers is provided by
some of the formatting specifications of {\tt format} (see
section~\ref{formatted-io}).

\method{generic-write}
%
\begin{specargs}
    \item[double, \classref{double-float}]%
    The double float to be output on {\em stream}.
    \item[stream, \classref{stream}]%
    The stream on which the representation is to be output.
\end{specargs}
%
\result%
The double float supplied as the first argument.
%
\remarks%
Outputs the external representation of {\em double\/} on {\em stream}, as an
optional sign preceding the syntax defined by {\em float format 3}.  Finer
control over the format of the output of floating point numbers is provided by
some of the formatting specifications of {\tt format} (see
section~\ref{formatted-io}).
%
\end{optDefinition}

\gdef\module{level-0}\newpage\sclause{Floating Point Numbers}
\index{general}{float}
\index{general}{float!module}
\label{float}
\index{general}{level-0 modules!float}
%
\begin{optDefinition}
\noindent
The defined name of this module is {\tt float}.  This module defines the
abstract class \classref{float}\ and the behaviour of some generic functions on
floating point numbers.  Further operations on numbers are defined in the
numbers module (\ref{number}) and further operations on floating point numbers
are defined in the elementary functions module (\ref{elementary-functions}).  A
concrete float class is defined in the double float module (\ref{double-float}).

\syntaxform{float}
%
The syntax for the external representation of floating point literals is defined
in syntax table~\ref{float-syntax}.  The representation used by
\functionref{write} and \functionref{prin} is that of a sign, a whole part and a
fractional part without an exponent, namely that defined by {\em float format
    3}.  Finer control over the format of the output of floating point numbers
is provided by some of the formatting specifications of \functionref{format}
(section~\ref{formatted-io}).
%
\Syntax
\label{float-syntax}
\defSyntax{float}{
\begin{syntax}
    \scdef{float}: \\
    \>  \scoptref{sign} \scref{unsigned-float} \scoptref{exponent} \\
   \scdef{unsigned-float}: \\
   \>  \scref{float-format-1} \\
   \>  \scref{float-format-2} \\
   \>  \scref{float-format-3} \\
   \scdef{float-format-1}: \\
   \>  \scref{decimal-integer} . \\
   \scdef{float-format-2}: \\
   \>  . \scref{decimal-integer} \\
   \scdef{float-format-3}: \\
   \>  \scref{float-format-1} \scref{decimal-integer} \\
   \scdef{exponent}: \\
   \>  \scref{double-exponent}
\end{syntax}}%
\showSyntaxBox{float}

A floating point number\index{general}{external representation!floating point}
has six forms of external representation depending on whether either or both the
whole and the fractional part are specified and on whether an exponent is
specified.  In addition, a positive floating point number is optionally preceded
by a plus sign and a negative floating point number is preceded by a minus sign.
For example:
%
\verb|+123.| ({\em float format 1\/}),
\verb|-.456| ({\em float format 2\/}),
\verb|123.456| ({\em float format 3\/}); and with exponents:
\verb|+123456.D-3|,
\verb|1.23455D2|,
\verb|-.123456D3|.

\derivedclass{float}{number}
\index{general}{level-0 classes!\theclass{float}}
%
The abstract class which is the superclass of all floating point
numbers.
%
\function{floatp}
\begin{arguments}
    \item[objext] Object to examine.
\end{arguments}
%
\result%
Returns {\em object\/} if it is a floating point number, otherwise \nil{}.

\generic{ceiling}
%
\begin{genericargs}
    \item[float, \classref{float}] A floating point number.
\end{genericargs}
%
\result%
Returns the smallest integral value not less than {\em float\/} expressed as a
float of the same class as the argument.

\generic{floor}
%
\begin{genericargs}
    \item[float, \classref{float}] A floating point number.
\end{genericargs}
%
\result%
Returns the largest integral value not greater than {\em float\/} expressed as a
float of the same class as the argument.

\generic{round}
%
\begin{arguments}
    \item[float] A floating point number.
\end{arguments}
%
\result%
Returns the integer whose value is closest to {\em float}, except in the case
when {\em float\/} is exactly half-way between two integers, when it is rounded
to the one that is even.

\generic{truncate}
%
\begin{arguments}
    \item[float] A floating point number.
\end{arguments}
%
\result%
Returns the greatest integer value whose magnitude is less than or equal to {\em
    float}.
%
\end{optDefinition}

\gdef\module{level-0}\newpage\input{formatted-io}
\gdef\module{level-0}\newpage\input{fpi}
\gdef\module{level-0}\newpage\sclause{Integers}
\index{general}{integer}
\index{general}{integer}{module}
\label{integer}
\index{general}{level-0 modules!integer}
%
\begin{optDefinition}
The defined name of this module is {\tt integer}.  This module defines
the abstract class \classref{integer}\ and the behaviour of some generic
functions on integers.  Further operations on numbers are defined in
the numbers module (\ref{number}).  A concrete integer class is
defined in the fixed precision integer module (\ref{fpi}).

\syntaxform{integer}

A positive integer\index{general}{external representation!integer} is has its
external representation as a sequence of digits optionally preceded by a plus
sign.  A negative integer\index{general}{external representation!integer} is
written as a sequence of digits preceded by a minus sign.  For example,
\verb+1234567890+, \verb+-456+, \verb-+1959-.

Integer literals have an external representation in any base up to base
36\index{general}{base}\index{general}{base!limitation on input}.  For
convenience, base 2\index{general}{binary
    literals}\index{general}{literal!binary}, base 8\index{general}{octal
    literals}\index{general}{literal!octal} and base
16\index{general}{hexadecimal literals}\index{general}{literal!hexadecimal} have
distinguished notations---\verb+#b+, \verb+#o+ and \verb+#x+, respectively.  For
example: \verb+1234+, \verb+#b10011010010+, \verb+#o2322+ and \verb+#x4d2+ all
denote the same value.

The general notation\index{general}{literal!arbitrary
    base}\index{general}{base!arbitrary base literals} for an arbitrary base is
\verb+#+{\em base\/}\verb+r+, where {\em base\/} is an unsigned decimal number.
Thus, the above examples may also be written: \verb+#10r1234+,
\verb+#2r10011010010+, \verb+#8r2322+, \verb+#16r4d2+ or \verb+#36rya+.  The
reading of any number is terminated on encountering a character which cannot be
a constituent of that number.  The syntax for the external representation of
integer literals is defined below.

\Syntax
\label{integer-syntax}
\savesyntax\integerSyntax\vbox{\small\syntax
integer
   = [sign] unsigned integer;
sign
   = '+' | '-';
unsigned integer
   = binary integer
   | octal integer
   | decimal integer
   | hexadecimal integer
   | specified base integer
binary integer
   = '#b', binary digit, {binary digit};
binary digit
   = '0' | '1';
octal integer
   = '#o', octal digit, {octal digit};
octal digit
   = '0' | '1' | '2' | '3' | '4' | '5' | '6' | '7';
decimal integer
   = decimal digit, {decimal digit};
hexadecimal integer
   = '#x', hexadecimal digit, {hexadecimal digit};
hexadecimal digit
   = decimal digit
   | hex lower letter
   | hex upper letter;
hex lower letter
   = 'a' | 'b' | 'c' | 'd' | 'e' | 'f';
hex upper letter
   = 'A' | 'B' | 'C' | 'D' | 'E' | 'F';
specified base integer
   = '#', base specification, 'r',
     specified base digit,
     {specified base digit};
base specification
   = decimal digit - ('0' | '1')
   | ( '1' | '2' ), decimal digit
   | '3', ('0' | '1' | '2' | '3' | '4' | '5' | '6');
specified base digit
   = decimal digit | letter;
\endsyntax}
\syntaxtable{integer}{\integerSyntax}

\begin{note}
    At present this text does not define a class integer with variable
    precision.  It is planned this should appear in a future version at level-1
    of the language.  The class will be named
    \classref{variable-precision-integer}.  The syntax given here is applicable
    to both fixed and variable precision integers.
\end{note}

\derivedclass{integer}{number}
\index{general}{level-0 classes!\theclass{integer}}
%
The abstract class which is the superclass of all integer numbers.

\function{integerp}
%
\begin{arguments}
    \item[object] Object to examine.
\end{arguments}
%
\result
Returns {\em object} if it is an integer, otherwise \nil.

\generic{evenp}
%
\begin{arguments}
    \item[integer, \classref{integer}] An integer.
\end{arguments}
%
\result
Returns \true\/ if two divides {\em integer}, otherwise \nil.

\function{oddp}
%
\begin{arguments}
    \item[integer] An integer.
\end{arguments}
%
\result
Returns the equivalent of the logical negation of \genericref{evenp} applied to
{\em integer}.
%
\end{optDefinition}

\gdef\module{level-0}\newpage\sclause{Keywords}
\label{keyword}
\index{general}{keyword}
\index{general}{keyword!module}
\index{general}{level-0 modules!keyword}
%
\begin{optDefinition}
The defined name of this module is {\tt keyword}.
%
\syntaxform{keyword}
\index{general}{keyword}
\index{general}{keyword!definition of}
%
The syntax of keywords is very similar to that of identifiers and of symbols,
including all the escape conventions, but are distinguished by a colon {\tt (:)}
suffix:
%
\Syntax
\defSyntax{keyword}{
\begin{syntax}
    \scdef{keyword}: \\
    \>  \scref{identifier}:
\end{syntax}}%
\showSyntaxBox{keyword}%

It is an error to use a keyword where an identifier is expected, such as, for
example, in lambda parameter lists or in let binding forms.

{\em The matter of keywords appering in lambda parameter lists, for example,
    {\tt rest:}, instead of the dot notation, is currently an open issue.}

Operationally, the most important aspect of keywords is that each is unique, or,
stated the other way around: the result of processing every syntactic token
comprising the same sequence of characters which denote a keyword is the same
object. Or, more briefly, every keyword with the same name denotes the same
keyword. A consequence of this guarantee is that keywords may be compared using
\functionref{eq}.

\derivedclass{keyword}{name}
\index{general}{level-0 classes!\theclass{keyword}}
%
The class of all instance of \classref{keyword}.
%
\begin{initoptions}
    \item[string, string] The string containing the characters to be used to
    name the keyword. The default value for string is the empty string, thus
    resulting in the keyword with no name, written \verb+|:|+.
\end{initoptions}
%
{\em What is the defined behaviour if the last character of string is colon?}

\function{keywordp}
%
\begin{arguments}
    \item[object] Object to examine.
\end{arguments}
%
\result%
Returns {\em object\/} if it is a keyword.

\function{keyword-name}
%
\begin{arguments}
    \item[keyword] A keyword.
\end{arguments}
%
\result%
Returns a {\em string\/} which is \methodref{binary=}{string} to that given as
the argument to the call to \functionref{make} which created {\em keyword}. It
is an error to modify this string.

\function{keyword-exists-p}
%
\begin{arguments}
    \item[string] A string containing the characters to be used to determine the
    existence of a keyword with that name.
\end{arguments}
%
\result%
Returns the keyword whose name is {\em string\/} if that keyword has already
been constructed by \functionref{make}. Otherwise, returns \nil{}.

\method{generic-prin}{keyword}
%
\begin{specargs}
    \item[keyword, \classref{keyword}] The keyword to be output on {\em stream}.
    %
    \item[stream, \classref{stream}] The stream on which the representation is to be
    output.
\end{specargs}
%
\result%
The keyword supplied as the first argument.
%
\remarks%
Outputs the external representation of {\em keyword\/} on {\em stream\/} as
described in the section on symbols, interpreting each of the characters in the
name.

\method{generic-write}{keyword}
%
\begin{specargs}
    \item[keyword, \classref{keyword}] The keyword to be output on {\em stream}.
    %
    \item[stream, \classref{stream}] The stream on which the representation is to be
    output.
\end{specargs}
%
\result%
The keyword supplied as the first argument.
%
\remarks%
Outputs the external representation of {\em keyword\/} on {\em stream\/} as
described in the section on symbols. If any characters in the name would not
normally be legal constituents of a keyword, the output is preceded and
succeeded by multiple-escape characters.
%
\examples
\begin{tabular}{lcl}
    \verb|(write (make <keyword> 'string "abc"))| &\Ra& \verb+abc:+\\
    \verb|(write (make <keyword> 'string "a c"))| &\Ra& \verb+|a c:|+\\
    \verb|(write (make <keyword> 'string ").("))| &\Ra& \verb+|).(:|+\\
\end{tabular}

\converter{string}
%
\begin{specargs}
    \item[keyword, \classref{keyword}] A keyword to be converted to a string.
\end{specargs}
%
\result%
A string.
%
\remarks%
This function is the same as \functionref{keyword-name}.  It is defined for the
sake of symmetry.
%
\end{optDefinition}

\gdef\module{level-0}\newpage\sclause{Lists}
\index{general}{null}
\label{null}
\index{general}{list}
\index{general}{list!module}
\label{list}
\index{general}{pair}
\label{pair}
%
\begin{optDefinition}
The name of this module is {\tt list}.  The class \classref{list}\ is an
abstract class and has two subclasses: \classref{null}\ and \classref{cons}.
The only instance of \classref{null}\ is the empty list.  The combination of
these two classes allows the creation of proper lists, since a proper list is
one whose last pair contains the empty list in its \functionref{cdr} field.  See
also section~\ref{collection} (collections) for further operations on lists.

% The class \syntaxref{pair}\index{general}{pair} (also known as a {\em dotted
%     pair}) is a 2-tuple, whose fields are called, for historical reasons, car
% and cdr.  Pairs are created by the function \functionref{cons} and the fields
% are accessed by the functions \functionref{car} and \functionref{cdr}.  The
% major use of pairs is in the construction of (proper) lists.  A (proper)
% list\index{general}{list} is defined as either the empty list (denoted by {\tt
%     '()}) or a pair whose \functionref{cdr} is a proper list.  An improper
% list is one containing a \functionref{cdr} which is not a list.

\derivedclass{list}{collection}
\index{general}{level-0 classes!\theclass{list}}
%
The class of all lists.

\syntaxform{()}
%
\remarks%
The empty list\index{general}{external representation!null (empty list)},
which is the only instance of the class \classref{null}, has as its
external representation an open parenthesis followed by a close
parenthesis.  The empty list is also used to denote the boolean value
{\em false}.

\derivedclass{null}{list}
\index{general}{level-0 classes!\theclass{null}}

The class whose only instance is the empty list, denoted \nil{}.

\function{null}
%
\begin{arguments}
    \item[object] Object to examine.
\end{arguments}
%
\result%
Returns \true\/ if {\em object} is the empty list, otherwise \nil{}.

\method{generic-prin}
%
\begin{specargs}
    \item[null] The empty list.
    \item[stream] The stream on which the representation is to be output.
\end{specargs}
%
\result%
The empty list.
%
\remarks%
Output the external representation of the empty list on {\em stream\/}
as described above.

\method{generic-write}
%
\begin{specargs}
    \item[null] The empty list.
    \item[stream] The stream on which the representation is to be output.
\end{specargs}
%
\result%
The empty list.
%
\remarks%
Output the external representation of the empty list on {\em stream\/}
as described above.

\syntaxform{pair}
%
A pair\index{general}{external representation!pair} is written as \verb+(+{\em
    object}$_1$ \verb+.+ {\em object}$_2$\verb+)+, where {\em object}$_1$ is
called the \functionref{car} and {\em object}$_2$ is called the
\functionref{cdr}.  There are two special cases in the external representation
of pair.  If {\em object}$_2$ is the empty list, then the pair is written as
\verb+(+{\em object}$_1$\verb+)+.  If {\em object$_2$} is an instance of
\syntaxref{pair}, then the pair is written as \verb+(+{\em object}$_1$ {\em
    object}$_3$ . {\em object$_4$}\verb+)+, where {\em object$_3$} is the
\functionref{car} of {\em object$_2$} and {\em object$_4$} is the
\functionref{cdr} with the above rule for the empty list applying.  By
induction, a list of length $n$ is written as\index{general}{external
    representation!list} \verb+(+{\em object}$_1$ \ldots {\em object$_{n-1}$}
. {\em object}$_n$\verb+)+, with the above rule for the empty list applying.
The representations of {\em object$_1$} and {\em object$_2$} are determined by
the external representations defined in other sections of this definition (see
\classref{character}\ (\ref{character}), \classref{double-float}\
(\ref{double-float}),
% \classref{null}\ (\ref{null}),
\classref{fixed-precision-integer}\ (\ref{spint}),
\classref{string}\ (\ref{string}),
\classref{symbol}\ (\ref{symbol}) and
\classref{vector}\ (\ref{vector}), as well as instances of \classref{cons}\
itself.  The syntax for the external representation of pairs and lists
is defined in syntax table~\ref{pair-syntax}.

\Syntax
\label{pair-syntax}
\defSyntax{list}{
\begin{syntaxx}
    \scdef{null}: \\
    \>  () \\
    \scdef{pair}: \\
    \>  ( \scref{object} . \scref{object} ) \\
    \scdef{list}: \\
    \>  \scref{empty-list} \\
    \>  \scref{proper-list} \\
    \>  \scref{improper-list} \\
    \scdef{empty-list}: \\
    \>  () \\
    \scdef{proper-list}: \\
    \>  ( \scSeqref{object} ) \\
    \scdef{improper-list}: \\
    \>  ( \scSeqref{object} . \scref{object} )
\end{syntaxx}}%
\showSyntaxBox{list}
%
\examples
\begin{tabular}{ll}
    \nil{} & the empty list\\
    {\tt (1)} & a list whose \functionref{car} is {\tt 1} and \functionref{cdr} is \nil{}\\
    {\tt (1 . 2)} & a pair whose \functionref{car} is {\tt 1} and \functionref{cdr} is
    {\tt 2}\\
    {\tt (1 2)} & a list whose \functionref{car} is {\tt 1} and \functionref{cdr} is {\tt
        (2)}
\end{tabular}

\derivedclass{cons}{list}
\index{general}{level-0 classes!\theclass{pair}}

The class of all instances of \classref{cons}.  An instance of the class
\classref{cons}\ (also known informally as a {\em dotted pair\/} or a {\em
    pair\/}) is a 2-tuple, whose slots are called, for historical reasons,
\functionref{car} and \functionref{cdr}.  Pairs are created by the function
\functionref{cons} and the slots are accessed by the functions \functionref{car}
and \functionref{cdr}.  The major use of pairs is in the construction of
(proper) lists.  A (proper) list\index{general}{list} is defined as either the
empty list (denoted by \nil{}) or a pair whose \functionref{cdr} is a proper list.
An improper list is one containing a \functionref{cdr} which is not a list (see
syntax table~\ref{pair-syntax}).

It is an error to apply \functionref{car} or \functionref{cdr} or their
\functionref{setter} functions to anything other than a pair.  The empty list is
not a pair and {\tt (car ())} or {\tt (cdr ())} is an error.

\function{consp}
%
\begin{arguments}
    \item[object] Object to examine.
\end{arguments}
%
\result%
Returns {\em object\/} if it is a pair, otherwise \nil{}.

\function{atom}
%
\begin{arguments}
    \item[object] Object to examine.
\end{arguments}
%
\result%
Returns {\em object\/} if it is not a pair, otherwise \nil{}.

\function{cons}
%
\begin{arguments}
    \item[object$_1$] An object.  pair.
    \item[object$_2$] An object.  pair.
\end{arguments}
%
\result%
Allocates a new pair whose slots are initialized with {\em object$_1$} in the
\functionref{car} and {\em object$_2$} in the \functionref{cdr}.

\function{car}
%
\begin{arguments}
    \item[pair] A pair.
\end{arguments}
%
\result%
Given a pair, such as the result of {\tt (cons {\em object$_1$} {\em
        object$_2$})}, then the function \functionref{car} returns {\em
    object$_1$}.

\function{cdr}
%
\begin{arguments}
    \item[pair] A pair.
\end{arguments}
%
\result%
Given a pair, such as the result of {\tt (cons {\em object$_1$} {\em
        object$_2$})}, then the function \functionref{cdr} returns {\em
    object$_2$}.

\setter{car}
%
\begin{arguments}
    \item[pair] A pair.
    \item[object] An object.
\end{arguments}
%
\result%
Given a pair, such as the result of {\tt (cons {\em object$_1$} {\em
        object$_2$})}, then the function {\tt (setter car)} replaces {\em
    object$_1$} with {\em object}.  The result is {\em object}.

\setter{cdr}
%
\begin{arguments}
    \item[pair] A pair.
    \item[object] An object.
\end{arguments}
%
\result%
Given a pair, such as the result of {\tt (cons {\em object$_1$} {\em
        object$_2$})}, then the function {\tt (setter cdr)} replaces {\em
    object$_2$} with {\em object}.  The result is {\em object}.
%
\remarks%
Note that if {\em object\/} is not a proper list, then the use of {\tt
(setter cdr)} might change {\em pair\/} into an improper list.

\method{equal}
%
\begin{specargs}
    \item[pair$_1$] A pair.
    \item[pair$_2$] A pair.
\end{specargs}
%
\result%
The result is the conjunction of the pairwise application of \genericref{equal}
to the \functionref{car} fields and the \functionref{cdr} fields of the
arguments.

\method{deep-copy}
%
\begin{specargs}
    \item[pair, \classref{cons}] A pair.
\end{specargs}
%
\result%
Constructs and returns a copy of the list starting at {\em pair\/} copying both
the \functionref{car} and the \functionref{cdr} slots of the list.  The list can
be proper or improper.  Treatment of the objects stored in the \functionref{car}
slot (and the \functionref{cdr} slot in the case of the final pair of an
improper list) is determined by the \genericref{deep-copy} method for the class
of the object.

\method{shallow-copy}
%
\begin{specargs}
    \item[pair, \classref{cons}] A pair.
\end{specargs}
%
\result%
Constructs and returns a copy of the list starting at {\em pair\/} but copying
only the \functionref{cdr} slots of the list, terminating when a pair is
encountered whose \functionref{cdr} slot is not a pair.  The list beginning at
{\em pair\/} can be proper or improper.

\function{list}
%
\begin{arguments}
    \item[{\optional{object$_1$ ... object$_n$}}] A sequence of objects.
\end{arguments}
%
\result%
Allocates a set of pairs each of which has been initialized with {\em
object$_i$} in the \functionref{car} field and the pair whose \functionref{car} field
contains {\em object$_{i+1}$} in the \functionref{cdr} field.  Returns the pair
whose \functionref{car} field contains {\em object$_1$}.
%
\examples
\begin{tabular}{lcl}
    \verb|(list)| &\Ra& \verb|()|\\
    \verb|(list 1 2 3)| &\Ra& \verb|(1 2 3)|
\end{tabular}

\method{generic-prin}
%
\begin{specargs}
    \item[pair, \classref{cons}] The pair to be output on {\em stream}.
    \item[stream, \classref{stream}] The stream on which the representation is
    to be output.
\end{specargs}
%
\result%
The pair supplied as the first argument.
%
\remarks%
Output the external representation of {\em pair\/} on {\em stream\/} as
described at the beginning of this section.  Uses \genericref{generic-prin} to
produce the external representation of the contents of the \functionref{car} and
\functionref{cdr} slots of {\em pair}.

\method{generic-write}
%
\begin{specargs}
    \item[pair, \classref{cons}] The pair to be output on {\em stream}.
    \item[stream, \classref{stream}] The stream on which the representation is
    to be output.
\end{specargs}
%
\result%
The pair supplied as the first argument.
%
\remarks%
Output the external representation of {\em pair\/} on {\em stream\/} as
described at the beginning of this section.  Uses \genericref{generic-write} to
produce the external representation of the contents of the \functionref{car} and
\functionref{cdr} slots of {\em pair}.

\end{optDefinition}

\gdef\module{level-0}\newpage\defModule{mathlib}{Elementary Functions}
%
\begin{optPrivate}
APL is probably the source for these definitions, which have been
subsequently adapted for Lisp.  More detail is required.

910520--Improved matters some by simplifying some of the stuff from
CLtl2.  Also checked out Fortran-90 but that is under-specified for my
liking.
\end{optPrivate}
\begin{optDefinition}
The defined name of this module is {\tt mathlib}.  The functionality defined for
this module is intentionally precisely that of the trigonmetric functions,
hyperbolic functions, exponential and logarithmic functions and power functions
defined for {\tt <math.h>} in ISO/IEC 9899~:~1990 with the exceptions of {\tt
    frexp}, {\tt ldexp} and {\tt modf}.

\constant{pi}{double-float}
%
\remarks%
The value of \constantref{pi} is the ratio the circumference of a circle to its
diameter stored to double precision floating point accuracy.

\generic{acos}
%
\begin{genericargs}
    \item[float, \classref{float}] A floating point number.
\end{genericargs}
%
\result%
Computes the principal value of the arc cosine of {\em float} which is a value
in the range $[0,\pi]$ radians.  An error is signalled (condition-class:
\conditionref{domain-condition}\indexcondition{domain-condition}) if {\em float}
is not in the range $[-1,+1]$.

\generic{asin}
%
\begin{genericargs}
    \item[float, \classref{float}] A floating point number.
\end{genericargs}
%
\result%
Computes the principal value of the arc sine of {\em float} which is a value in
the range $[-\pi/2,+\pi/2]$ radians.  An error is signalled (condition-class:
\conditionref{domain-condition}\indexcondition{domain-condition}) if {\em float}
is not in the range $[-1,+1]$.

\generic{atan}
%
\begin{genericargs}
    \item[float, \classref{float}] A floating point number.
\end{genericargs}
%
\result%
Computes the principal value of the arc tangent of {\em float}
which is a value in the range $[-\pi/2,+\pi/2]$ radians.

\generic{atan2}
%
\begin{genericargs}
    \item[float$_1$, \classref{float}] A floating point number.
    \item[float$_2$, \classref{float}] A floating point number.
\end{genericargs}
%
\result%
Computes the principal value of the arc tangent of {\em float$_1$}/{\em
    float$_2$}, which is a value in the range $[-\pi,+\pi]$ radians, using the
signs of both arguments to determine the quadrant of the result.  An error might
be signalled (condition-class:
\conditionref{domain-condition}\indexcondition{domain-condition}) if either {\em
    float$_1$} or {\em float$_2$} is zero.

\generic{cos}
%
\begin{genericargs}
    \item[float, \classref{float}] A floating point number.
\end{genericargs}
%
\result%
Computes the cosine of {\em float} (measured in radians).

\generic{sin}
%
\begin{genericargs}
    \item[float, \classref{float}] A floating point number.
\end{genericargs}
%
\result%
Computes the sine of {\em float} (measured in radians).

\generic{tan}
%
\begin{genericargs}
    \item[float, \classref{float}] A floating point number.
\end{genericargs}
%
\result%
Computes the tangent of {\em float} (measured in radians).

\generic{cosh}
%
\begin{genericargs}
    \item[float, \classref{float}] A floating point number.
\end{genericargs}
%
\result%
Computes the hyperbolic cosine of {\em float}.  An error might be signalled
(condition class:
\conditionref{range-condition}\indexcondition{range-condition}) if the magnitude
of {\em float} is too large.

\generic{sinh}
%
\begin{genericargs}
    \item[float, \classref{float}] A floating point number.
\end{genericargs}
%
\result%
Computes the hyperbolic sine of {\em float}.  An error might be signalled
(condition class:
\conditionref{range-condition}\indexcondition{range-condition}) if the magnitude
of {\em float} is too large.

\generic{tanh}
%
\begin{genericargs}
    \item[float, \classref{float}] A floating point number.
\end{genericargs}
%
\result%
Computes the hyperbolic tangent of {\em float}.

\generic{exp}
%
\begin{genericargs}
    \item[float, \classref{float}] A floating point number.
\end{genericargs}
%
\result%
Computes the exponential function of {\em float}.  An error might be signalled
(condition class:
\conditionref{range-condition}\indexcondition{range-condition}) if the magnitude
of {\em float} is too large.

\generic{log}
%
\begin{genericargs}
    \item[float, \classref{float}] A floating point number.
\end{genericargs}
%
\result%
Computes the natural logarithm of {\em float}.  An error is signalled (condition
class: \conditionref{domain-condition}\indexcondition{domain-condition}) if {\em
    float} is negative. An error might be signalled (condition class:
\conditionref{range-condition}\indexcondition{range-condition}) if {\em float}
is zero.

\generic{log10}
%
\begin{genericargs}
    \item[float, \classref{float}] A floating point number.
\end{genericargs}
%
\result%
Computes the base-ten logarithm of {\em float}.  An error is signalled
(condition class:
\conditionref{domain-condition}\indexcondition{domain-condition}) if {\em float}
is negative. An error might be signalled (condition class:
\conditionref{range-condition}\indexcondition{range-condition}) if {\em float}
is zero.

\generic{pow}
%
\begin{genericargs}
    \item[float$_1$, \classref{float}] A floating point number.
    \item[float$_2$, \classref{float}] A floating point number.
\end{genericargs}
%
\result%
Computes {\em float$_1$} raised to the power {\em float$_2$}.  An error is
signalled (condition class:
\conditionref{domain-condition}\indexcondition{domain-condition}) if {\em
    float$_1$} is negative and {\em float$_2$} is not integral.  An error is
signalled (condition class: \conditionref{domain-condition}) if the result
cannot be represented when {\em float$_1$} is zero and {\em float$_2$} is less
than or equal to zero.  An error might be signalled (condition class:
\conditionref{range-condition}\index{general}{range-condition}) if the result
cannot be represented.

\generic{sqrt}
%
\begin{genericargs}
    \item[float, \classref{float}] A floating point number.
\end{genericargs}
%
\result%
Computes the non-negative square root of {\em float}.  An error is signalled
(condition class:
\conditionref{domain-condition}\indexcondition{domain-condition}) if {\em float}
is negative.
%
\end{optDefinition}

\gdef\module{level-0}\newpage\sclause{Numbers}
\index{general}{number}
\index{general}{number!module}
\label{number}
\index{general}{level-0 modules!number}
\index{general}{generic arithmetic}
%
\begin{optPrivate}
    Need to discuss issue of ordering and how that fits in with gaussian
    extensions.  Is it simply enough that there is no applicable method in each
    case?  Where is class number in this hierachy?  Perhaps number is ring?

    Ring defines 0, 1, +, -, * and possibly **.  Euclidean domain adds div,
    factor etc.  Field adds /, etc.
\end{optPrivate}
%
\begin{optRationale}
    In keeping with Lisp tradition, many of the arithmetic operators are n-ary.
    However, because \eulisp\ uses its generic function facility to describe the
    arithmetic operations---so that the user can add new arithmetics and
    integrate them with the rest of the system---and because n-argument
    discrimination cannot be satisfactorily defined, the actual generic
    arithmetic functions are all binary.  The meaning of each n-ary function in
    terms of the binary operator is defined by saying the binary operation is
    left-associative.  Thus, {\tt (+ a b c d)} must be evaluated as {\tt
        (binary+ (binary+ (binary+ a b) c) d)}.  The operators affected by this
    requirement are: \functionref{+}, \functionref{-}, \functionref{*},
    \functionref{/}, \functionref{max} and \functionref{min}.  The arithmetic
    comparison functions are also n-ary, so that {\tt (< a b c d)} must be
    evaluated in the same way.  The operators affected are: \functionref{=} and
    \functionref{<}.
\end{optRationale}
%
\begin{optDefinition}
\noindent
The defined name of this module is {\tt number}.

Numbers can take on many forms with unusual properties, specialized for
different tasks, but two classes of number suffice for the majority of needs,
namely integers (\classref{integer}, \classref{int}) and floating point
numbers (\classref{float}, \classref{double-float}).  Thus, these only are
defined at level-0.

Table~\ref{level-0-number-class-hierarchy} shows the initial number class
hierarchy at level-0.  The inheritance relationships by this diagram are part of
this definition, but it is not defined whether they are direct or not.  For
example, \classref{integer}\ and \classref{float}\ are not necessarily direct
subclasses of \classref{number}, while the class of each number class might be a
subclass of \classref{number}.  Since there are only two concrete number
classes at level-0,
coercion\index{general}{coercion}\index{general}{number!coercion} is simple,
namely from \classref{int}\ to \classref{double-float}.  Any
level-0 version of a library module, for example, {\em elementary-functions}
\ref{elementary-functions}, need only define methods for these two classes.
Mathematically, the reals are regarded as a superset of the integers and for the
purposes of this definition we regard \classref{float}\ as a superset of
\classref{integer}\ (even though this will cause representation problems when
variable precision integers are introduced).  Hence, \classref{float}\ is
referred to as being {\em higher} that \classref{integer}\ and arithmetic
involving instances of both classes will cause integers to be converted to an
equivalent floating point value, before the calculation proceeds\footnote{This
    behaviour is popularly referred to as {\em floating point contagion}} (see
in particular \genericref{binary/},
\genericreflabel{binary\protect\%}{binarypercent} and \genericref{binary-mod}).

\begin{table}[h]
\caption{Level-0 number class hierarchy}
\label{level-0-number-class-hierarchy}
\begin{center}
{\tt\begin{tabbing}
<num\=ber> \\
    \><flo\=at> \\
    \>    \><double-float> \\
    \><integer> \\
    \>    \><int>
\end{tabbing}}
\end{center}
\end{table}

\derivedclass{number}{object}
%
The abstract class which is the superclass of all number classes.

\function{number?}
%
\begin{arguments}
    \item[object] Object to examine.
\end{arguments}
%
\result%
Returns {\em object\/} if it is a number, otherwise \nil{}.

\condition{arithmetic-condition}{condition}

\begin{initoptions}
    \item[operator, object] The operator which signalled the condition.
    \item[operand-list, list] The operands passed to the operator.
\end{initoptions}
%
\remarks%
This is the general condition class for conditions arising from arithmetic
operations.

\condition{division-by-zero}{arithmetic-condition}
%
Signalled by any of \genericref{binary/},
\genericreflabel{binary\protect\%}{binarypercent} and \genericref{binary-mod} if
their second argument is zero.

\function{+}
%
\begin{arguments}
    \item[{\optional{number$_1$ number$_2$ ...}}] A sequence of numbers.
\end{arguments}
%
\result%
Computes the sum of the arguments using the generic function
\genericref{binary+}.  Given zero arguments, \functionref{+} returns {\tt 0} of
class \classref{integer}.  One argument returns that argument.  The arguments
are combined left-associatively.

\function{-}
%
\begin{arguments}
    \item[{number$_1$ \optional{number$_2$ ...}}] A non-empty sequence of
    numbers.
\end{arguments}
%
\result%
Computes the result of subtracting successive arguments---from the second to the
last---from the first using the generic function \genericref{binary-}.  Zero
arguments is an error.  One argument returns the negation of the argument, using
the generic function \genericref{negate}.  The arguments are combined
left-associatively.

\function{*}
%
\begin{arguments}
    \item[{\optional{number$_1$ number$_2$ ...}}] A sequence of numbers.
\end{arguments}
%
\result%
Computes the product of the arguments using the generic function
\genericref{binary*}.  Given zero arguments, \functionref{*} returns {\tt 1} of
class \classref{integer}.  One argument returns that argument.  The arguments
are combined left-associatively.

\function{/}
%
\begin{arguments}
    \item[{number$_1$ \optional{number$_2$ ...}}] A non-empty sequence of
    numbers.
\end{arguments}
%
\result%
Computes the result of dividing the first argument by its succeeding arguments
using the generic function \genericref{binary/}.  Zero arguments is an
error.  One argument computes the reciprocal of the argument.  It is an error in
the single argument case, if the argument is zero.

\functionlabeled{\protect\%}{percent}
%
\begin{arguments}
    \item[{number$_1$ \optional{number$_2$ ...}}] A non-empty sequence of
    numbers.
\end{arguments}
%
\result%
Computes the result of taking the remainder of dividing the first argument by
its succeeding arguments using the generic function
\genericreflabel{binary\protect\%}{binarypercent}.  Zero arguments is an error.
One argument returns that argument.

\function{mod}
%
\begin{arguments}
    \item[{number$_1$ \optional{number$_2$ ...}}] A non-empty sequence of
    numbers.
\end{arguments}
%
\result%
Computes the largest integral value not greater than the result of dividing the
first argument by its succeeding arguments using the generic function
\genericref{binary-mod}.  Zero arguments is an error.  One argument returns {\em
    number$_1$}.

\function{gcd}
%
\begin{arguments}
    \item[{number$_1$ \optional{number$_2$ ...}}] A non-empty sequence of
    numbers.
\end{arguments}
%
\result%
Computes the greatest common divisor of {\em number$_1$} up to {\em number$_n$}
using the generic function \genericref{binary-gcd}.  Zero arguments is an error.  One
argument returns {\em number$_1$}.

\function{lcm}
%
\begin{arguments}
    \item[{number$_1$ \optional{number$_2$ ...}}] A non-empty sequence of
    numbers.
\end{arguments}
%
\result%
Computes the least common multiple of {\em number$_1$} up to {\em number$_n$}
using the generic function \genericref{binary-lcm}.  Zero arguments is an error.  One
argument returns {\em number$_1$}.

\function{abs}
%
\begin{arguments}
    \item[number] A number.
\end{arguments}
%
\result%
Computes the absolute value of {\em number}.

\generic{zero?}
%
\begin{genericargs}
    \item[number] A number.
\end{genericargs}
%
\result%
Compares {\em number\/} with the zero element of the class of {\em number\/}
using the generic function \genericref{binary=}.

\generic{negate}
%
\begin{genericargs}
    \item[number, \classref{number}] A number.
\end{genericargs}
%
\result%
Computes the additive inverse of {\em number}.

\function{signum}
%
\begin{arguments}
    \item[number] A number.
\end{arguments}
%
\result%
Returns {\em number\/} if \genericref{zero?} applied to {\em number\/} is
\true{}.  Otherwise returns the result of converting $\pm$1 to the class of {\em
    number\/} with the sign of {\em number}.

\function{positive?}
%
\begin{arguments}
    \item[number] A number.
\end{arguments}
%
\result%
Compares {\em number\/} against the zero element of the class of {\em number\/}
using the generic function \genericref{binary<}.

\function{negative?}
%
\begin{arguments}
    \item[number] A number.
\end{arguments}
%
\result%
Compares {\em number\/} against the zero element of the class of {\em number\/}
using the generic function \genericref{binary<}.

\method{binary=}{number}
%
\begin{genericargs}
    \item[number$_1$, \classref{number}] A number.
    \item[number$_2$, \classref{number}] A number.
\end{genericargs}
%
\result%
Returns \true{} if {\em number$_1$} and {\em number$_2$} are numerically equal
otherwise \nil{};

\generic{binary+}
%
\begin{genericargs}
    \item[number$_1$, \classref{number}] A number.
    \item[number$_2$, \classref{number}] A number.
\end{genericargs}
%
\result%
Computes the sum of {\em number$_1$} and {\em number$_2$}.

\generic{binary-}
\begin{genericargs}
    \item[number$_1$, \classref{number}] A number.
    \item[number$_2$, \classref{number}] A number.
\end{genericargs}
%
\result%
Computes the difference of {\em number$_1$} and {\em number$_2$}.

\generic{binary*}
%
\begin{genericargs}
    \item[number$_1$, \classref{number}] A number.
    \item[number$_2$, \classref{number}] A number.
\end{genericargs}
%
\result%
Computes the product of {\em number$_1$} and {\em number$_2$}.

\generic{binary/}
%
\begin{genericargs}
    \item[number$_1$, \classref{number}] A number.
    \item[number$_2$, \classref{number}] A number.
\end{genericargs}
%
\result%
Computes the division of {\em number$_1$} by {\em number$_2$} expressed as a
number of the class of the higher of the classes of the two arguments.  The sign
of the result is positive if the signs the arguments are the same.  If the signs
are different, the sign of the result is negative.  If the second argument is
zero, the result might be zero or an error might be signalled (condition class:
\conditionref{division-by-zero}\indexcondition{division-by-zero}).

\genericlabeled{binary\protect\%}{binarypercent}
%
\begin{genericargs}
    \item[number$_1$, \classref{number}] A number.
    \item[number$_2$, \classref{number}] A number.
\end{genericargs}
%
\result%
Computes the value of {\em{}number$_1$}$-i*${\em{}number$_2$} expressed as a
number of the class of the higher of the classes of the two arguments, for some
integer $i$ such that, if {\em number$_2$} is non-zero, the result has the same
sign as {\em number$_1$} and magnitude less then the magnitude of {\em
    number$_2$}.  If the second argument is zero, the result might be zero or an
error might be signalled (condition class:
\conditionref{division-by-zero}\indexcondition{division-by-zero}).

\generic{binary-mod}
%
\begin{genericargs}
    \item[number$_1$, \classref{number}] A number.
    \item[number$_2$, \classref{number}] A number.
\end{genericargs}
%
\result%
Computes the largest integral value not greater than {\em
    number$_1$}$\over${\em{}number$_2$} expressed as a number of the class of
the higher of the classes of the two arguments, such that if {\em number$_2$} is
non-zero, the result has the same sign as {\em number$_2$} and magnitude less
than {\em number$_2$}.  If the second argument is zero, the result might be zero
or an error might be signalled (condition class:
\conditionref{division-by-zero}\indexcondition{division-by-zero}).

\generic{binary-gcd}
%
\begin{genericargs}
    \item[number$_1$, \classref{number}] A number.
    \item[number$_2$, \classref{number}] A number.
\end{genericargs}
%
\result%
Computes the greatest common divisor of {\em number$_1$} and {\em number$_2$}.

\generic{binary-lcm}
%
\begin{genericargs}
    \item[number$_1$, \classref{number}] A number.
    \item[number$_2$, \classref{number}] A number.
\end{genericargs}
%
\result%
Computes the lowest common multiple of {\em number$_1$} and {\em number$_2$}.
%
\end{optDefinition}

\gdef\module{level-0}\newpage%%% ----------------------------------------------------------------------------
%%% Streams
\sclause{Streams}
\label{stream}
\index{general}{stream}
\index{general}{stream!module}
\index{general}{level-0 modules!stream}
%
\begin{optDefinition}
The defined name of this module is {\tt stream}.

The aim of the stream design presented here is an open architecture for
programming with streams, which should be applicable when the interface to some
object can be characterized by either serial access to, or delivery of,
objects.

The two specific objectives are: (i) transfer of objects between a process and
disk storage; (ii) transfer of objects between one process and another.

The fundamental purpose of a stream object in the scheme presented here is to
provide an interface between two objects through the two functions
\functionref{read}, for streams from which objects are received, and
\functionref{write}, for streams to which objects are sent.

%%%  ---------------------------------------------------------------------------
%%%  Stream classes
\ssclause{Stream classes}

% ------------------------------------------------------------------------------
\derivedclass{stream}{object}
% ------------------------------------------------------------------------------
\index{general}{level-0 classes!\classref{stream}}
%
This is the root of the stream class hierarchy and also defines the basic
stream class.
%
\begin{initoptions}
    \item[read-action, \classref{function}] A function which is called by the
    \classref{stream} \methodref{generic-read}{stream} method. The accessor for
    this slot is called {\tt stream-read-action}.
    %
    \item[write-action, \classref{function}] A function which is called by the
    \classref{stream} \methodref{generic-write}{stream} method. The accessor for
    this slot is called {\tt stream-write-action}.
\end{initoptions}
%
The following accessor functions are defined for \classref{stream}
%
\begin{functions}
    \item[stream-lock] A lock, to be used to allow exclusive access to a stream.
    %
    \item[stream-source] An object to which the stream is connected and from
    which input is read.
    %
    \item[stream-sink] An object to which the stream is connected and to which
    ouptut is written.
    %
    \item[stream-buffer] An object which is used to buffer data by some
    subclasses of \classref{stream}. Its default value is \nil{}.
    %
    \item[stream-buffer-size] The maximum number of objects that can be stored
    in {\em stream-buffer}. Its default value is 0.
\end{functions}
%
\noindent
The transaction unit of \classref{stream} is \classref{object}.

% ------------------------------------------------------------------------------
\function{stream?}
% ------------------------------------------------------------------------------
\begin{arguments}
  \item[object, \classref{object}] The object to be examined.
\end{arguments}
%
\result%
Returns {\em object\/} if it is a stream, otherwise \nil{}.

% ------------------------------------------------------------------------------
\function{from-stream}
% ------------------------------------------------------------------------------
A constructor function of one argument for \classref{stream} which
returns a stream whose {\tt stream-read-action} is the given argument.

% ------------------------------------------------------------------------------
\function{to-stream}
% ------------------------------------------------------------------------------
A constructor function of one argument for \classref{stream} which returns a
stream whose {\tt stream-write-action} is the given argument.

% ------------------------------------------------------------------------------
\derivedclass{buffered-stream}{stream}
% ------------------------------------------------------------------------------
\index{general}{level-0 classes!\theclass{buffered-stream}}
%
This class specializes \classref{stream} by the use of a buffer which may grow
arbitrarily large. The transaction unit of \classref{buffered-stream} is
\classref{object}.

% ------------------------------------------------------------------------------
\derivedclass{fixed-buffered-stream}{buffered-stream}
% ------------------------------------------------------------------------------
\index{general}{level-0 classes!\theclass{fixed-buffered-stream}}
%
This class specializes \classref{buffered-stream} by placing a bound on the
growth of the buffer. The transaction unit of \classref{fixed-buffered-stream}
is \classref{object}.

% ------------------------------------------------------------------------------
\derivedclass{file-stream}{fixed-buffered-stream}
% ------------------------------------------------------------------------------
\index{general}{level-0 classes!\theclass{file-stream}}
%
This class specializes \classref{fixed-buffered-stream} by providing an
interface to data stored in files on disk. The transaction unit of
\classref{file-stream} is \classref{character}. The following additional
accessor functions are defined for \classref{file-stream}:
%
\begin{functions}
    \item[file-stream-filename] The path identifying the file system object
    associated with the stream.
    %
    \item[file-stream-mode] The mode of the connection between the stream and
    the file system object (usually either read or write).
    %
    \item[file-stream-buffer-position] A key identifying the current position in
    the stream's buffer.
\end{functions}

% ------------------------------------------------------------------------------
\function{file-stream?}
% ------------------------------------------------------------------------------
\begin{arguments}
  \item[object, \classref{object}] The object to be examined.
\end{arguments}
%
\result%
Returns {\em object\/} if it is a \classref{file-stream} otherwise \nil{}.

% ------------------------------------------------------------------------------
\derivedclass{string-stream}{buffered-stream}
% ------------------------------------------------------------------------------
\index{general}{level-0 classes!\theclass{string-stream}}
%
The class of the default string stream.

% ------------------------------------------------------------------------------
\function{string-stream?}
% ------------------------------------------------------------------------------
\begin{arguments}
    \item[object, \classref{object}] The object to be examined.
\end{arguments}
%
\result%
Returns {\em object\/} if it is a \classref{string-stream} otherwise \nil{}.

%%%  ---------------------------------------------------------------------------
%%%  Stream operators
\ssclause{Stream operators}

% ------------------------------------------------------------------------------
\function{connect}
% ------------------------------------------------------------------------------
\begin{arguments}
    \item[source] The source object from which the stream will read data.
    \item[sink] The sink object to which the stream will write data.
    \item[\optional{options}] An optional argument for specifying
    implementation-defined options.
\end{arguments}
%
\result%
The return value is \nil{}.
%
\remarks%
Connects {\em source\/} to {\em sink\/} according to the class-specific
behaviours of \genericref{generic-connect}.

% ------------------------------------------------------------------------------
\generic{generic-connect}
% ------------------------------------------------------------------------------
\begin{genericargs}
    \item[source, \classref{object}] The source object from which the stream
    will read data.
    \item[sink, \classref{object}] The sink object to which the stream will
    write data.
    \item[\optional{options}, \classref{list}] An optional argument for
    specifying implementation-defined options.
\end{genericargs}
%
\remarks%
Generic form of \functionref{connect}.

% ------------------------------------------------------------------------------
\method{generic-connect}{stream}
% ------------------------------------------------------------------------------
\begin{specargs}
    \item[source, \classref{stream}] The stream which is to be the source of
    {\em sink}.
    \item[sink, \classref{stream}] The stream which is to be the sink of {\em
        source}.
    \item[options, \classref{list}] A list of implementation-defined options.
\end{specargs}
%
\result%
The return value is \nil{}.
%
\remarks%
Connects the source of {\em sink\/} to {\em source\/} and the sink of
{\em source\/} to {\em sink}.

% ------------------------------------------------------------------------------
\method{generic-connect}{path}
% ------------------------------------------------------------------------------
\begin{specargs}
    \item[source, \theclass{path}] A path name.
    \item[sink, \classref{file-stream}] The stream via which data will be
    received from the file named by \scref{path}.
    \item[options, \classref{list}] A list of implementation-defined options.
\end{specargs}
%
\result%
The return value is \nil{}.
%
\remarks%
Opens the object identified by the path {\em source\/} for reading and
connects {\em sink\/} to it. Hereafter, {\em sink\/} may be used for reading
data from {\em sink\/}, until {\em sink\/} is disconnected or
reconnected. Implementation-defined options for the opening of files may be
specified using the third argument.
%
\seealso%
\functionref{open-input-file}.

% ------------------------------------------------------------------------------
\method{generic-connect}{file-stream}
% ------------------------------------------------------------------------------
\begin{specargs}
    \item[source, \classref{file-stream}] The stream via which data will be sent
    to the file named by \scref{path}.
    \item[sink, \theclass{path}] A path name.
    \item[options, \classref{list}] A list of implementation-defined options.
\end{specargs}
%
\result%
The return value is \nil{}.
%
\remarks%
Opens the object identifed by the path {\em sink\/} for writing and
connects {\em source\/} to it. Hereafter, {\em source\/} may be used for writing
data to {\em sink}, until {\em source\/} is disconnected or
reconnected. Implementation-defined options for the opening of files may be
specified using the third argument.
%
\seealso%
\functionref{open-output-file}.

% ------------------------------------------------------------------------------
\generic{reconnect}
% ------------------------------------------------------------------------------
\begin{genericargs}
    \item[s1, \classref{stream}] A stream.
    \item[s2, \classref{stream}] A stream.
\end{genericargs}
%
\result%
The return value is \nil{}.
%
\remarks%
Transfers the source and sink connections of {\em s1\/} to {\em s2},
leaving {\em s1\/} disconnected.

% ------------------------------------------------------------------------------
\method{reconnect}{stream}
% ------------------------------------------------------------------------------
\begin{specargs}
    \item[s1, \classref{stream}] A stream.
    \item[s2, \classref{stream}] A stream.
\end{specargs}
%
\result%
The return value is \nil{}.
%
\remarks%
Implements the \genericref{reconnect} operation for objects of class
\classref{stream}.

% ------------------------------------------------------------------------------
\generic{disconnect}
% ------------------------------------------------------------------------------
\begin{genericargs}
    \item[s, \classref{stream}] A stream.
\end{genericargs}
%
\result%
The return value is \nil{}.
%
\remarks%
Disconnects the stream {\em s\/} from its source and/or its sink.

% ------------------------------------------------------------------------------
\method{disconnect}{stream}
% ------------------------------------------------------------------------------
\begin{specargs}
    \item[s, \classref{stream}] A stream.
\end{specargs}
%
\result%
The return value is \nil{}.
%
\remarks%
Implements the diconnect operation for objects of class \classref{stream}.

% ------------------------------------------------------------------------------
\method{disconnect}{file-stream}
% ------------------------------------------------------------------------------
\begin{specargs}
    \item[s, \classref{file-stream}] A file stream.
\end{specargs}
%
\result%
The return value is \nil{}.
%
\remarks%
Implements the diconnect operation for objects of class
\classref{file-stream}. In particular, this involves closing the file associated
with the stream {\em s}.

%%%  ---------------------------------------------------------------------------
%%%  Stream objects
\ssclause{Stream objects}

% ------------------------------------------------------------------------------
\instance{stdin}{file-stream}
% ------------------------------------------------------------------------------
\remarks%
The standard input stream, which is a file-stream and whose transaction unit is
therefore character. In Posix compliant configurations, this object is
initialized from the Posix \instanceref{stdin} object. Note that although
\instanceref{stdin} itself is a constant binding, it may be connected to
different files by the \genericref{reconnect} operation.

% ------------------------------------------------------------------------------
\instance{lispin}{stream}
% ------------------------------------------------------------------------------
\remarks%
The standard lisp input stream, and its transaction unit is object. This stream
is initially connected to \instanceref{stdin} (although not necessarily
directly), thus a \functionref{read} operation on \instanceref{lispin} will case
characters to be read from \instanceref{stdin} and construct and return an
object corresponding to the next lisp expression. Note that although
\instanceref{lispin} itself is a constant binding, it may be connected to
different source streams by the \genericref{reconnect} operation.

% ------------------------------------------------------------------------------
\instance{stdout}{file-stream}
% ------------------------------------------------------------------------------
\remarks%
The standard output stream, which is a file-stream and whose transaction unit is
therefore character. In Posix compliant configurations, this object is
initialized from the Posix \instanceref{stdout} object. Note that although
\instanceref{stdout} itself is a constant binding, it may be connected to
different files by the \genericref{reconnect} operation.

% ------------------------------------------------------------------------------
\instance{stderr}{file-stream}
% ------------------------------------------------------------------------------
\remarks%
The standard error stream, which is a file-stream and whose transaction unit is
therefore character. In Posix compliant configurations, this object is
initialized from the Posix \instanceref{stderr} object. Note that although
\instanceref{stderr} itself is a constant binding, it may be connected to
different files by the \genericref{reconnect} operation.

%%%  ---------------------------------------------------------------------------
%%%  Buffer management
\ssclause{Buffer management}
\label{Buffer-management}

% ------------------------------------------------------------------------------
\generic{fill-buffer}
% ------------------------------------------------------------------------------
\begin{genericargs}
    \item[stream, \classref{buffered-stream}] A stream.
\end{genericargs}
%
\result%
The buffer associated with {\em stream\/} is refilled from its {\em source\/}.
Returns a count of the number of items read.
%
\remarks%
This function is guaranteed to be called when an attempt is made to read from a
buffered stream whose buffer is either empty, or from which all the items have
been read.

% ------------------------------------------------------------------------------
\method{fill-buffer}{buffered-stream}
% ------------------------------------------------------------------------------
\begin{specargs}
    \item[stream, \classref{buffered-stream}] A stream.
\end{specargs}

% ------------------------------------------------------------------------------
\method{fill-buffer}{file-stream}
% ------------------------------------------------------------------------------
\begin{specargs}
    \item[stream, \classref{file-stream}] A stream.
\end{specargs}

% ------------------------------------------------------------------------------
\generic{flush-buffer}
% ------------------------------------------------------------------------------
\begin{genericargs}
    \item[stream, \classref{buffered-stream}] A stream.
\end{genericargs}
%
\result%
The contents of the buffer associated with {\em stream} is flushed to
its sink. If this operation succeeds, a \true{} value is returned, otherwise the
result is \nil{}.
%
\remarks%
This function is guaranteed to be called when an attempt is made to
write to a buffered stream whose buffer is full.

% ------------------------------------------------------------------------------
\method{flush-buffer}{buffered-stream}
% ------------------------------------------------------------------------------
\begin{specargs}
    \item[stream, \classref{buffered-stream}] A stream.
\end{specargs}
%
\result%
The contents of the buffer associated with {\em stream} is flushed to
its sink. If this operation succeeds, a \true{} value is returned, otherwise the
result is \nil{}.
%
\remarks%
Implements the \genericref{flush-buffer} operation for objects of class
\classref{buffered-stream}.

% ------------------------------------------------------------------------------
\method{flush-buffer}{file-stream}
% ------------------------------------------------------------------------------
\begin{specargs}
    \item[stream, \classref{file-stream}] A stream.
\end{specargs}
%
\result%
The contents of the buffer associated with {\em stream} is flushed to
its sink. If this operation succeeds, a \true{} value is returned, otherwise the
result is \nil{}.
%
\remarks%
Implements the \genericref{flush-buffer} operation for objects of
\classref{file-stream}. This method is called both when the buffer is full and
after a newline character is written to the buffer.

% ------------------------------------------------------------------------------
\condition{end-of-stream}{stream-condition}
% ------------------------------------------------------------------------------
\begin{initoptions}
    \item[stream, \classref{stream}] A stream.
\end{initoptions}
%
\remarks%
Signalled by the default end of stream action, as a consequence of a
read operation on {\em stream\/}, when it is at end of stream.
%
\seealso%
\genericref{generic-read}.

% ------------------------------------------------------------------------------
\generic{end-of-stream}
% ------------------------------------------------------------------------------
\begin{genericargs}
    \item[stream, \classref{buffered-stream}] A stream.
\end{genericargs}
%
\remarks%
This function is guaranteed to be called when a read operation
encounters the end of {\em stream\/} and the {\tt eos-error?} argument to read
has a non-\nil{}\/ value.

% ------------------------------------------------------------------------------
\method{end-of-stream}{buffered-stream}
% ------------------------------------------------------------------------------
\begin{specargs}
    \item[stream, \classref{buffered-stream}] A stream.
\end{specargs}
%
\remarks%
Signals the end of stream condition.

% ------------------------------------------------------------------------------
\method{end-of-stream}{file-stream}
% ------------------------------------------------------------------------------
\begin{specargs}
    \item[stream, \classref{file-stream}] A stream.
\end{specargs}
%
\remarks%
Disconnects {\em stream\/} and signals the end of stream condition.

%%%  ---------------------------------------------------------------------------
%%%  Reading from streams
\ssclause{Reading from streams}

% ------------------------------------------------------------------------------
\condition{read-error}{condition}
% ------------------------------------------------------------------------------
%
\begin{genericargs}
    \item[stream, \classref{stream}] A stream.
\end{genericargs}
%
\remarks%
Signalled by a \functionref{read} operation which fails in some manner other
than when it is at end of stream.

% ------------------------------------------------------------------------------
\function{read}
% ------------------------------------------------------------------------------
\begin{arguments}
    \item[\optional{stream}] A stream.
    \item[\optional{eos-error?}] A boolean.
    \item[\optional{eos-value}] Value to be returned to indicate end of stream.
\end{arguments}
%
\result%
That of calling \genericref{generic-read} with the arguments supplied or
defaulted as described.
%
\remarks%
The {\em stream\/} defaults to \instanceref{lispin}, {\em
    eos-error?\/} defaults to \nil{}\/ and {\em eos-value\/} defaults to {\tt
    eos-default-value}.

% ------------------------------------------------------------------------------
\generic{generic-read}
% ------------------------------------------------------------------------------
\begin{genericargs}
    \item[stream, \classref{stream}] A stream.
    \item[eos-error?, \classref{object}] A boolean.
    \item[eos-value, \classref{object}] Value to be returned to indicate end of
    stream.
\end{genericargs}
%
\result%
The next transaction unit from {\em stream}.
%
\remarks%
If the end of {\em stream\/} is encountered and the value of {\em eos-error?}
is \nil{}, the result is {\em eos-value\/}. If the end of {\em stream} is
encountered and the value of {\tt eos-error?} is non-\nil{}, the function
\methodref{end-of-stream}{stream} is called with the argument {\em stream}.

% ------------------------------------------------------------------------------
\method{generic-read}{stream}
% ------------------------------------------------------------------------------
\begin{specargs}
    \item[stream, \classref{stream}] A stream.
    \item[eos-error?, \classref{object}] A boolean.
    \item[eos-value, \classref{object}] Value to be returned to indicate end of
    stream.
\end{specargs}
%
\result%
That of calling the {\em read-action\/} of stream with the arguments {\em
    stream\/}, {\em eos-error?\/} and {\em eos-value}.  Returns \true.
%
\remarks%
Implements the \genericref{generic-read} operation for objects of class
\classref{stream}.

% ------------------------------------------------------------------------------
\method{generic-read}{buffered-stream}
% ------------------------------------------------------------------------------
\begin{specargs}
    \item[stream, \classref{buffered-stream}] A buffered stream.
    \item[eos-error?, \classref{object}] A boolean.
    \item[eos-value, \classref{object}] Value to be returned to indicate end of
    stream.
\end{specargs}
%
\result%
The next object stored in the stream buffer.  If the buffer is empty,
the function \genericref{fill-buffer} is called. If the refilling operation did
not succeed, the end of stream action is carried out as described under
\genericref{generic-read}.  Returns \true.
%
\remarks%
Implements the \genericref{generic-read} operation for objects of class
\classref{buffered-stream}.

% ------------------------------------------------------------------------------
\method{generic-read}{file-stream}
% ------------------------------------------------------------------------------
\begin{specargs}
    \item[stream, \classref{file-stream}] A file stream.
    \item[eos-error?, \classref{object}] A boolean.
    \item[eos-value, \classref{object}] Value to be returned to indicate end of
    stream.
\end{specargs}
%
\result%
The next object stored in the stream buffer.  If the buffer is empty,
the function \genericref{fill-buffer} is called. If the refilling operation did
not succeed, the end of stream action is carried out as described under
\genericref{generic-read}.  Returns \true.
%
\remarks%
Implements the \genericref{generic-read} operation for objects of class
\classref{file-stream}.

%%%  ---------------------------------------------------------------------------
%%%  Writing to streams
\ssclause{Writing to streams}

% ------------------------------------------------------------------------------
\generic{generic-write}
% ------------------------------------------------------------------------------
\begin{genericargs}
    \item[object, \classref{object}] An object to be written to {\em stream}.
    \item[stream, \classref{stream}] Stream to which {\em object\/} is to be
    written.
\end{genericargs}
%
\result%
Returns {\em object}.
%
\remarks%
Outputs the external representation of {\em object\/} on the output stream
{\em stream}.
%
\seealso%
The following \genericref{generic-write} methods are defined:
\methodref{generic-write}{character}, \methodref{generic-write}{symbol},
\methodref{generic-write}{keyword},
\methodref{generic-write}{fixed-precision-integer},
\methodref{generic-write}{double-float}, \methodref{generic-write}{null},
\methodref{generic-write}{cons}, \methodref{generic-write}{list},
\methodref{generic-write}{string}, \methodref{generic-write}{vector},
\methodref{generic-write}{stream}, \methodref{generic-write}{buffered-stream}
and \methodref{generic-write}{file-stream}.

% ------------------------------------------------------------------------------
\method{generic-write}{stream}
% ------------------------------------------------------------------------------
\begin{specargs}
    \item[object, \classref{object}] An object to be written to {\em stream}.
    \item[stream, \classref{stream}] Stream to which {\em object\/} is to be
    written.
\end{specargs}

% ------------------------------------------------------------------------------
\method{generic-write}{buffered-stream}
% ------------------------------------------------------------------------------
\begin{specargs}
    \item[object, \classref{object}] An object to be written to {\em stream}.
    \item[stream, \classref{buffered-stream}] Stream to which {\em object\/} is
    to be written.
\end{specargs}

% ------------------------------------------------------------------------------
\method{generic-write}{file-stream}
% ------------------------------------------------------------------------------
\begin{specargs}
    \item[object, \classref{object}] An object to be written to {\em stream}.
    \item[stream, \classref{file-stream}] Stream to which {\em object\/} is
    to be written.
\end{specargs}

% ------------------------------------------------------------------------------
\function{swrite}
% ------------------------------------------------------------------------------
\begin{arguments}
    \item[stream] Stream to which {\em object\/} is to be written.
    \item[object] An object to be written to {\em stream}.
\end{arguments}
%
\result%
Returns {\em stream}.
%
\remarks%
Outputs the external representation of {\em object\/} on the output
stream {\em stream\/} using \genericref{generic-write}.
%
\seealso%
\genericref{generic-write}.

% ------------------------------------------------------------------------------
\function{write}
% ------------------------------------------------------------------------------
\begin{arguments}
    \item[object] An object to be written to {\em stream}.
\end{arguments}
%
\result%
Returns \instanceref{stdout}.
%
\remarks%
Outputs the external representation of {\em object\/} on \instanceref{stdout}
using \genericref{generic-write}.
%
\seealso%
\functionref{swrite}, \genericref{generic-write}.

%%%  ---------------------------------------------------------------------------
%%%  Additional functions
\ssclause{Additional functions}

% ------------------------------------------------------------------------------
\function{read-line}
% ------------------------------------------------------------------------------
\begin{arguments}
    \item[stream] A stream.
    \item[\optional{eos-error?}] A boolean.
    \item[\optional{eos-value}] Value to be returned to indicate end of stream.
\end{arguments}
%
\result%
A string.
%
\remarks%
Reads a line (terminated by a newline character or the end of the
stream) from the stream of characters which is {\em stream}.  Returns
the line as a string, discarding the terminating newline, if any.  If
the stream is already at end of stream, then the stream action is
called: the default stream action is to signal an error: (condition
class: \conditionref{end-of-stream}\indexcondition{end-of-stream}).

% ------------------------------------------------------------------------------
\generic{generic-prin}
% ------------------------------------------------------------------------------
\begin{genericargs}
    \item[object, \classref{object}] An object to be output on {\em stream}.
    \item[stream, \classref{stream}] A character stream on which {\em
        object\/} is to be output.
\end{genericargs}
%
\result%
Returns {\em object}.
%
\remarks%
Outputs the external representation of {\em object\/} on the output stream {\em
    stream\/}.
%
\seealso%
\functionref{prin}.  The following \genericref{generic-write} methods are
defined: \methodref{generic-write}{character},
\methodref{generic-write}{symbol}, \methodref{generic-write}{keyword},
\methodref{generic-write}{fixed-precision-integer},
\methodref{generic-write}{double-float}, \methodref{generic-write}{null},
\methodref{generic-write}{cons}, \methodref{generic-write}{list},
\methodref{generic-write}{string} and \methodref{generic-write}{vector}.

% ------------------------------------------------------------------------------
\function{sprin}
% ------------------------------------------------------------------------------
\begin{arguments}
    \item[stream] A character stream on which {\em object\/} is to
    be output.
    \item[{\optional{object$_1$ object$_2$ ...}}] A sequence of objects to be
    output on {\em stream}.
\end{arguments}
%
\result%
Returns {\em stream}.
%
\remarks%
Outputs the external representation of {\em object$_1$ object$_2$ ...} on the
output stream {\em stream\/} using \genericref{generic-prin} for each object.
%
\seealso%
\genericref{generic-prin}.

% ------------------------------------------------------------------------------
\function{prin}
% ------------------------------------------------------------------------------
\begin{arguments}
    \item[{\optional{object$_1$ object$_2$ ...}}] A sequence of objects to be
    output on \instanceref{stdout}.
\end{arguments}
%
\result%
Returns \instanceref{stdout}.
%
\remarks%
Outputs the external representation of {\em object$_1$ object$_2$ ...} on the
output stream \instanceref{stdout} using \genericref{sprin} for each object.
%
\seealso%
\functionref{sprin}, \genericref{sprin}.

% ------------------------------------------------------------------------------
\function{sprint}
% ------------------------------------------------------------------------------
\begin{arguments}
    \item[stream] A character stream on which {\em object\/} is
    to be output.
    \item[{\optional{object$_1$ object$_2$ ...}}] A sequence of objects to be
    output on {\em stream}.
\end{arguments}
%
\result%
Returns {\em stream}.
%
\remarks%
Outputs the external representation of {\em object$_1$ object$_2$ ...} on the
output stream {\em stream\/} using \genericref{generic-prin} for each object,
followed by a newline character.
%
\seealso%
\genericref{generic-prin}.

% ------------------------------------------------------------------------------
\function{print}
% ------------------------------------------------------------------------------
\begin{arguments}
    \item[{\optional{object$_1$ object$_2$ ...}}] A sequence of objects to be
    output on \instanceref{stdout}.
\end{arguments}
%
\result%
Returns \instanceref{stdout}.
%
\remarks%
Outputs the external representation of {\em object$_1$ object$_2$ ...} on the
output stream \instanceref{stdout} using \genericref{sprint} for each object,
followed by a newline character.
%
\seealso%
\functionref{sprint}, \genericref{sprint}.

% ------------------------------------------------------------------------------
\function{snewline}
% ------------------------------------------------------------------------------
\begin{arguments}
    \item[stream] A stream on which the newline is to be output.
\end{arguments}
%
\result%
Returns {\em stream}.
%
\remarks%
Outputs a newline character to {\em stream}.

% ------------------------------------------------------------------------------
\function{newline}
% ------------------------------------------------------------------------------
%
\result%
Returns \instanceref{stdout}.
%
\remarks%
Outputs a newline character to \instanceref{stdout}.
%
\seealso%
\functionref{snewline}.

% ------------------------------------------------------------------------------
\function{sflush}
% ------------------------------------------------------------------------------
\begin{arguments}
    \item[stream] A stream to flush.
\end{arguments}
%
\result%
Returns {\em stream}.
%
\remarks%
{\tt sflush} causes any buffered data for the stream to be written to
the stream. The stream remains open.

% ------------------------------------------------------------------------------
\function{flush}
% ------------------------------------------------------------------------------
%
\result%
Returns \instanceref{stdout}.
%
\remarks%
{\tt flush} causes any buffered data for \instanceref{stdout} to be written to
\instanceref{stdout}.
%
\seealso%
\functionref{sflush}.

% ------------------------------------------------------------------------------
\function{sprin-char}
% ------------------------------------------------------------------------------
\begin{arguments}
    \item[stream] A stream.
    \item[char] Character to be written to {\em stream}.
    \item[\optional{times}] Integer count.
\end{arguments}
%
\result%
Outputs {\em char} on {\em stream}.  The optional count {\em times\/} defaults
to 1.

% ------------------------------------------------------------------------------
\function{prin-char}
% ------------------------------------------------------------------------------
\begin{arguments}
    \item[char] Character to be written to \instanceref{stdout}.
    \item[\optional{times}] Integer count.
\end{arguments}
%
\result%
Outputs {\em char} on \instanceref{stdout}. The optional count {\em times\/}
defaults to 1.
%
\seealso%
\functionref{sprin-char}.

% ------------------------------------------------------------------------------
\function{sread}
% ------------------------------------------------------------------------------
\begin{arguments}
    \item[stream] A stream.
    \item[\optional{eos-error?}] A boolean.
    \item[\optional{eos-value}] Value to be returned to indicate end of stream.
\end{arguments}

%%% ---------------------------------------------------------------------------
%%%  Convenience forms
\ssclause{Convenience forms}

% ------------------------------------------------------------------------------
\function{open-input-file}
% ------------------------------------------------------------------------------
\begin{arguments}
    \item[path] A path identifying a file system object.
\end{arguments}
%
\result%
Allocates and returns a new \classref{file-stream} object whose source
is connected to the file system object identified by \scref{path}.

% ------------------------------------------------------------------------------
\function{open-output-file}
% ------------------------------------------------------------------------------
\begin{arguments}
    \item[path] A path identifying a file system object.
\end{arguments}
%
\result%
Allocates and returns a new \classref{file-stream} object whose sink is
connected to the file system object identified by \scref{path}.

% ------------------------------------------------------------------------------
\specop{with-input-file}
% ------------------------------------------------------------------------------
\Syntax
\defSyntax{with-input-file}{
\begin{syntax}
    \scdef{with-input-file-form}: \ra{} \classref{object} \\
    \>  ( \specopref{with-input-file} \scref{path} \\
    \>\>  \scref{body} ) \\
    \scdef{path}: \\
    \>  \scref{string}
\end{syntax}}%
\showSyntaxBox{with-input-file}

% ------------------------------------------------------------------------------
\specop{with-output-file}
% ------------------------------------------------------------------------------
\Syntax
\defSyntax{with-output-file}{
\begin{syntax}
    \scdef{with-output-file-form}: \ra{} \classref{object} \\
    \>  ( \specopref{with-output-file} \scref{path} \\
    \>\>  \scref{body} )
\end{syntax}}%
\showSyntaxBox{with-output-file}

% ------------------------------------------------------------------------------
\specop{with-source}
% ------------------------------------------------------------------------------
\Syntax
\defSyntax{with-source}{
\begin{syntax}
    \scdef{with-source-form}: \ra{} \classref{object} \\
    \>  ( \specopref{with-source} ( \scref{identifier} \scref{form} ) \\
    \>\>  \scref{body} )
\end{syntax}}%
\showSyntaxBox{with-source}

% ------------------------------------------------------------------------------
\specop{with-sink}
% ------------------------------------------------------------------------------
\Syntax
\defSyntax{with-sink}{
\begin{syntax}
    \scdef{with-sink-form}: \ra{} \classref{object} \\
    \>  ( \specopref{with-sink} ( \scref{identifier} \scref{form} ) \\
    \>\>  \scref{body} )
\end{syntax}}%
\showSyntaxBox{with-sink}

\end{optDefinition}
%
\begin{optPrivate}
%
%%%  ---------------------------------------------------------------------------
%%%  Posix bindings
\ssclause{Posix bindings}

% ------------------------------------------------------------------------------
\function{fopen}
% ------------------------------------------------------------------------------

Arguments

stream
A stream to connect to a file.
path
A path identifying a file system object.
mode
How the file is to be opened.

Connects stream to the file system object identified by path according to the
specified mode. The mode is specified by one of the following symbols:

r
Opens file for reading.
w
Opens file for writing, creating the file if it does not exist.
a
Opens file for append output, creating the file if it does not exist. If
the file exists, the stream position points to the end of the file, so any
output adds to the file.
r+
Opens the file for update. Both input and output can be performed on the
stream. Any existing data in the file is preserved.
w+
Opens the file for update. Both input and output can be performed on the
stream. Any existing data in the file is eliminated; the file is re-created
if it exists.
a+
Opens the file for update. The file can be read at any location, but any
output to the file occurs starting from its current end.

The result is stream.

% ------------------------------------------------------------------------------
\function{fclose}
% ------------------------------------------------------------------------------

Arguments

stream
A stream.

Closes the file stream stream and returns (). The stream is disconnected from
the file system object with which it was associated and may no longer be used
for read or write operations. The stream buffer is flushed to the file system
before the file is closed.

% ------------------------------------------------------------------------------
\function{fcntl}
% ------------------------------------------------------------------------------

Arguments

stream
A stream.
[cmds]
fcntl commands as defined below.

This function returns or sets information about an open file stream. The symbol
command determines the action of fcntl and in some cases, the third argument
command-values must also be supplied. The command can be any one of the
following symbols, which are a subset of those defined in 1990:??? (POSIX C
API):

F_GETFL
This is the default value. In this case, fcntl returns a list of symbols or
integers, in no particular order, describing the stream's attributes. These
values can be among the following:

O_APPEND
The file is open in append mode.
O_NONBLOCK
Waiting for data does not cause blocking.
O_RDONLY
The file is open for read only.
O_RDWR
The file is open for read and write.
O_WRONLY
The file is open for write only.
Other implementation-defined values specified as integers.

F_SETFL
In this case the the third argument command-values must be a list of
attributes from the set described above for F_GETFL. However, according to
1990:???, only the values O_APPEND and O_NONBLOCK can be modified.
\end{optPrivate}

%%% ----------------------------------------------------------------------------

\gdef\module{level-0}\newpage\defModule{string}{Strings}
%
\begin{optDefinition}
The defined name of this module is {\tt string}.  See also
section~\ref{collection} (collections) for further operations on
strings.
%
\syntaxform{string}
%
String literals\index{general}{external representation!string} are
delimited by the glyph called {\em quotation mark\/} (\verb+"+).  For
example, \verb+"abcd"+.

Sometimes it might be desirable to include string delimiter characters in
strings.  The aim of escaping in strings \index{general}{string!escaping in} is
to fulfill this need.  The \sc{string-escape}
\index{general}{string!string-escape glyph} glyph is defined as {\em reverse
    solidus\/} (\verb+\+).  String escaping can also be used to include certain
other characters that would otherwise be difficult to denote.  The set of named
special characters (see \S~\ref{character-set} and \S~\ref{character}) are
included in strings using the character digrams defined in
table~\ref{character-digrams}.  To allow arbitrary characters to appear in
strings, the hex-insertion digram is followed by an integer denoting the
position of the character in the current character set as for characters (see
\S~\ref{character-set}).  The syntax for the external representation of strings
is defined in syntax table~\ref{string-syntax} below:
%
\Syntax
\label{string-syntax}
\defSyntax{string}{
\begin{syntax}
    \scdef{string}: \\
    \>  " \scseqref{string-constituent} " \\
    \scdef{string-constituent}: \\
    \>  \scref{normal-string-constituent} \\
    \>  \scref{digram-string-constituent} \\
    \>  \scref{numeric string constituent} \\
   \scdef{normal-string-constituent}: \\
   \>  \scref{level-0-character} other than " or \textbackslash{} \\
   \scdef{digram-string-constituent}: one of \\
   \>  \textbackslash{}a \textbackslash{}b \textbackslash{}d \textbackslash{}f
   \textbackslash{}l \textbackslash{}n \textbackslash{}r \textbackslash{}t
   \textbackslash{}v \textbackslash{}" \textbackslash{}\textbackslash{} \\
   \scdef{numeric-string-constituent}: \\
   \>  \textbackslash{}x \scref{hexadecimal-digit} \\
   \>  \textbackslash{}x \scref{hexadecimal-digit} \scref{hexadecimal-digit} \\
   \>  \textbackslash{}x \scref{hexadecimal-digit} \scref{hexadecimal-digit} \\
   \>\>  \scref{hexadecimal-digit} \\
   \>  \textbackslash{}x \scref{hexadecimal-digit} \scref{hexadecimal-digit} \\
   \>\>  \scref{hexadecimal-digit} \scref{hexadecimal-digit}
\end{syntax}}%
\showSyntaxBox{string}

Some examples of string literals appear in table~\ref{example:string-literal}.
%
\begin{example}
\label{example:string-literal}
\examplecaption{Examples of string literals}
\begin{center}
\begin{tabular}{|ll|}\hline
    Example & Contents\\\hline
    \verb+"a\nb"+ & \verb+#\a+, \verb+#\n+ and \verb+#\b+\\
    \verb+"c\\"+ & \verb+#\c+ and \verb+#\\+\\
    \verb+"\x1 "+ & \verb+#\x1+ followed by \verb+#\space+\\
    \verb+"\xabcde"+ & \verb+#\xabcd+ followed by \verb+#\e+\\
    \verb+"\x1\x2"+ & \verb+#\x1+ followed by \verb+#\x2+\\
    \verb-"\x12+"- & \verb+#\x12+ followed by \verb-#\+-\\
    \verb+"\xabcg"+ & \verb+#\xabc+ followed by \verb+#\g+\\
    \verb+"\x00abc"+ & \verb+#\xab+ followed by \verb+#\c+\\\hline
\end{tabular}
\end{center}
\end{example}
%
\begin{note}
    At present this document refers to the ``current character set'' but
    defines no means of selecting alternative character sets.  This is to
    allow for future extensions and implementation-defined extensions
    which support more than one character set.
\end{note}
%
The function \functionref{write} outputs a re-readable form of any escaped
characters in the string.  For example, \verb+"a\n\\b"+ (input
notation) is the string containing the characters \verb+#\n+,
\verb+#\a+, \verb+#\\+ and \verb+#\b+.  The function \functionref{write}
produces \verb+"a\n\\b"+, whilst \functionref{prin} produces
%
\begin{verbatim}
a
\b
\end{verbatim}
%
The function \functionref{write} outputs characters which do not have a glyph
associated with their position in the character set as a hex insertion
in which all four hex digits are specified, even if there are leading
zeros, as in the last example in table~\ref{example:string-literal}.
The function \functionref{prin} outputs the interpretation of the characters
according to the definitions in section~\ref{character} without the
delimiting quotation marks.
%
\derivedclass{string}{character-sequence}
%
The class of all instances of \classref{string}.
%
\begin{initoptions}
%
\item[size, \classref{int}]
The number of characters in the string.  Strings are zero-based and
thus the maximum index is {\em size-1}.  If not specified the {\em
size\/} is zero.
%
\item[fill-value:, \classref{character}]
A character with which to initialize the string.  The default fill
character is \verb|#\x0|.
%
\end{initoptions}
%
\examples
%
\begin{tabular}{lcl}
\verb|(make <string>)| &\Ra& \verb|""|\\
\verb|(make <string> size: 2)| &\Ra& \verb|"\x0000\x0000"|\\
\verb|(make <string> size: 3| &\Ra& \verb|"aaa"|\\
\verb|  fill-value: #\a)|&&\\
\end{tabular}
%
\function{string?}
%
\begin{arguments}
    %
    \item[object] Object to examine.
    %
\end{arguments}
%
\result%
Returns {\em object\/} if it is a string, otherwise \nil{}.
%
\converter{symbol}
\begin{specargs}
    \item[string, \classref{string}] A string to be converted to a symbol.
\end{specargs}
%
\result%
If the result of \functionref{symbol-exists?} when applied to {\em string\/}
is a symbol, that symbol is returned.  If the result is \nil{}, then
a new symbol is constructed whose name is {\em string}.  This new
symbol is returned.
%
\method{binary=}{string}
%
\begin{specargs}
    \item[string$_1$, \classref{string}] A string.
    \item[string$_2$, \classref{string}] A string.
\end{specargs}
%
\result%
If the size of {\em string$_1$} is the same (under {\tt =}) as that of {\em
    string$_2$}, and the result of the conjunction of the pairwise application
of \methodref{binary=}{character} to the elements of the arguments is \true{}
the result is {\em string$_1$}.  If not the result is \nil{}.
%
\method{deep-copy}{string}
%
\begin{specargs}
    \item[string, \classref{string}] A string.
\end{specargs}
%
\result%
Constructs and returns a copy of {\em string\/} in which each element
is \functionref{eql} to the corresponding element in {\em string}.
%
\method{shallow-copy}{string}
%
\begin{specargs}
    \item[string, \classref{string}] A string.
\end{specargs}
%
\result%
Constructs and returns a copy of {\em string\/} in which each element
is \functionref{eql} to the corresponding element in {\em string}.
%
\method{binary<}{string}
%
\begin{specargs}
    \item[string$_1$, \classref{string}] A string.
    \item[string$_2$, \classref{string}] A string.
\end{specargs}
%
\result%
If the second argument is longer than the first, the result is \nil{}.
Otherwise, if the sequence of characters in {\em string$_1$} is pairwise less
than that in {\em string$_2$} according to \methodref{binary<}{character} the
result is \true.  Otherwise the result is \nil{}.  Since it is an error to
compare lower case, upper case and digit characters with any other kind than
themselves, so it is an error to compare two strings which require such
comparisons and the results are undefined.
%
\examples
\begin{tabular}{lcl}
\verb|(< "a" "b")| &\Ra& \verb|t|\\
\verb|(< "b" "a")| &\Ra& \verb|()|\\
\verb|(< "a" "a")| &\Ra& \verb|()|\\
\verb|(< "a" "ab")| &\Ra& \verb|t|\\
\verb|(< "ab" "a")| &\Ra& \verb|()|\\
\verb|(< "A" "B")| &\Ra& \verb|t|\\
\verb|(< "0" "1")| &\Ra& \verb|t|\\
\verb|(< "a1" "a2")| &\Ra& \verb|t|\\
\verb|(< "a1" "bb")| &\Ra& \verb|t|\\
\verb|(< "a1" "ab")| &\Ra& {\em undefined}
\end{tabular}
%
\seealso
Method \methodref{binary<}{character} (\ref{compare}) for characters (\ref{character}).
%
\method{as-lowercase}{string}
%
\begin{specargs}
    \item[string, \classref{string}] A string.
\end{specargs}
%
\result%
Returns a copy of {\em string\/} in which each character denoting an
upper case character, is replaced by a character denoting its lower
case counterpart.  The result must not be \functionref{eq} to {\em string}.
%
\method{as-uppercase}{string}
%
\begin{specargs}
    \item[string, \classref{string}] A string.
\end{specargs}
%
\result%
Returns a copy of {\em string\/} in which each character denoting an
lower case character, is replaced by a character denoting its upper
case counterpart.  The result must not be \functionref{eq} to {\em string}.
%
\method{generic-print}{string}
\begin{specargs}
    \item[string, \classref{string}] String to be ouptut on {\em stream}.
    \item[stream, \classref{stream}] Stream on which {\em string} is to be ouptut.
\end{specargs}
%
\result%
The string {\em string}.
%
Output external representation of {\em string\/} on {\em stream\/} as
described in the introduction to this section, interpreting each of
the characters in the string.  The opening and closing quotation marks
are not output.
%
\method{generic-write}{string}
\begin{specargs}
    \item[string, \classref{string}] String to be ouptut on {\em stream}.
    \item[stream, \classref{stream}] Stream on which {\em string\/} is to be ouptut.
\end{specargs}
%
\result%
The string {\em string}.
%
Output external representation of {\em string\/} on {\em stream\/} as
described in the introduction to this section, replacing single
characters with escape sequences if necessary.  Opening and closing
quotation marks are output.
\end{optDefinition}

\gdef\module{level-0}\newpage\sclause{Symbols}
\index{general}{symbol}
\index{general}{symbol!module}
\label{symbol}
\index{general}{level-0 modules!symbol}
%
\begin{optPrivate}
    Fixed definition of {\tt special-form-p} after RJB's observation and (later)
    KMP's CL fix to be called {\tt special-operator-p}.
\end{optPrivate}
%
\begin{optDefinition}
The defined name of this module is {\tt symbol}.
%
\syntaxform{symbol}
%
A \scref{symbol} is a literal \scref{identifier} and hence has the same syntax
\ref{identifier-syntax}:

\Syntax
\label{symbol-syntax}
\defSyntax{symbol}{
\begin{syntax}
    \scdef{symbol}: \\
    \>  \scref{identifier} \\
\end{syntax}}%
\showSyntaxBox{symbol}%

Because there are two escaping mechanisms and because the first character of a
token affects the interpretation of the remainder, there are many ways in which
to input the same \scref{identifier}.  If this same identifier is used as a
literal, \ie a \scref{symbol}, the results of processing each token denoting the
\scref{identifier} will be \functionref{eq} to one another.  For example, the
following tokens all denote the same \scref{symbol}:
%
\begin{center}
\verb+|123|+, \verb+\123+, \verb+|1|23+, \verb+||123+, \verb+||||123+
\end{center}
%
which will be output by the function \functionref{write} as \verb+|123|+.  If
output by \functionref{write}, the representation of the \scref{symbol} will
permit reconstruction by \functionref{read}---escape characters are
preserved---so that equivalence is maintained between \functionref{read} and
\functionref{write} for \scref{symbol}s.  For example: \verb+|a(b|+ and
\verb+abc.def+ are two \scref{symbol}s as output by \functionref{write} such
that \functionref{read} can read them as two \scref{symbol}s.  If output by
\functionref{prin}, the escapes necessary to re-\functionref{read} the
\scref{symbol} will not be included.  Thus, taking the same examples,
\functionref{prin} outputs \verb+a(b+ and \verb+abc.def+, which
\functionref{read} interprets as the \scref{symbol} \verb+a+ followed by the
start of a list, the \scref{symbol} \verb+b+ and the \scref{symbol}
\verb+abc.def+.

Computationally, the most important aspect of \scref{symbol}s is that each is
unique, or, stated the other way around: the result of processing
every syntactic token comprising the same sequence of characters which
denote an identifier is the same object.  Or, more briefly, every
identifier with the same name denotes the same \scref{symbol}.

\derivedclass{symbol}{name}
\index{general}{level-0 classes!\theclass{symbol}}
%
The class of all instances of \classref{symbol}.
%
\begin{initoptions}
    \item[string, string] The string containing the characters to be used to
    name the symbol.  The default value for string is the empty string, thus
    resulting in the symbol with no name, written \verb+||+.
\end{initoptions}

\function{symbolp}
%
\begin{arguments}
    \item[object] Object to examine.
\end{arguments}
%
\result%
Returns {\em object\/} if it is a symbol.

\function{gensym}
%
\begin{arguments}
%
    \item[\optional{string}] A string contain characters to be prepended
    to the name of the new symbol.
\end{arguments}
%
\result%
Makes a new symbol whose name, by default, begins with the character \verb+#\g+
and the remaining characters are generated by an implementation-defined
mechanism\index{general}{processor-defined!\functionref{gensym} names}.
Optionally, an alternative prefix string for the name may be specified.  It is
guaranteed that the resulting symbol did not exist before the call to
\functionref{gensym}.

\function{symbol-name}
%
\begin{arguments}
    \item[symbol] A symbol.
\end{arguments}
%
\result%
Returns a {\em string\/} which is \methodref{binary=} to that given as the
argument to the call to \functionref{make} which created {\em symbol}.  It is an
error to modify this string.

\function{symbol-exists-p}
%
\begin{arguments}
    \item[string] A string containing the characters to be used to determine the
    existence of a symbol with that name.
\end{arguments}
%
\result%
Returns the symbol whose name is {\em string\/} if that symbol has
already been constructed by \functionref{make}.  Otherwise, returns \nil{}.

\method{generic-prin}
%
\begin{specargs}
    \item[symbol, \classref{symbol}] The symbol to be output on {\em stream}.

    \item[stream, \classref{stream}] The stream on which the representation is
    to be output.
\end{specargs}
%
\result%
The symbol supplied as the first argument.
%
\remarks%
Outputs the external representation of {\em symbol\/} on {\em stream\/}
as described in the introduction to this section, interpreting each of the
characters in the name.

\method{generic-write}
%
\begin{specargs}
    \item[symbol, \classref{symbol}] The symbol to be output on {\em stream}.

    \item[stream, \classref{stream}] The stream on which the representation is
    to be output.
\end{specargs}
%
\result%
The symbol supplied as the first argument.
%
\remarks%
Outputs the external representation of {\em symbol\/} on {\em stream\/}
as described in the introduction to this section.  If any characters in the name
would not normally be legal constituents of an identifier or symbol, the output
is preceded and succeeded by multiple-escape characters.
%
\examples
\begin{tabular}{lcl}
    \verb|(write (make <symbol> 'string "abc"))| &\Ra& \verb+abc+\\
    \verb|(write (make <symbol> 'string "a c"))| &\Ra& \verb+|a c|+\\
    \verb|(write (make <symbol> 'string ").("))| &\Ra& \verb+|).(|+\\
\end{tabular}

\converter{string}
%
\begin{specargs}
    \item[symbol, \classref{symbol}] A symbol to be converted to a string.
\end{specargs}
%
\result%
A string.
%
\remarks%
This function is the same as \functionref{symbol-name}.  It is defined for the
sake of symmetry.
%
\end{optDefinition}

\gdef\module{level-0}\newpage\sclause{Tables}
\label{table}
\index{general}{table}
\index{general}{table!module}
\index{general}{level-0 modules!table}
%
\begin{optRationale}
    Operationally, tables resemble hashtables, but the actual representation is
    not defined in order to permit alternative solutions, such as various forms
    of balanced trees.
\end{optRationale}
%
\begin{optDefinition}
The defined name of this module is {\tt table}.  See also
section~\ref{collection} (collections) for further operations on tables.

\derivedclass{table}{collection}
\index{general}{level-0 classes!\theclass{table}}
%
The class of all instances of \classref{table}.
%
\begin{initoptions}
    \item[comparator, \classref{function}]%
    The function to be used to compare keys.  The default comparison function is
    \functionref{eql}.

    \item[fill-value, \classref{object}]%
    An object which will be returned as the default value for any key which does
    not have an associated value.  The default fill value is \nil{}.

    \item[hash-function, \classref{function}]%
    The function to be used to compute an unique key for each object stored in
    the table.  This function must return a fixed precision integer.  The hash
    function must also satisfy the constraint that if the comparison function
    returns \true{} for any two objects, then the hash function must return the
    same key when applied to those two objects.  The default is an
    implementation defined function which satisfies these conditions.
\end{initoptions}

\function{table?}
%
\begin{arguments}
    \item[object] Object to examine.
\end{arguments}
%
\result%
Returns {\em object\/} if it is a table, otherwise \nil{}.

\function{clear-table}
%
\begin{arguments}
    \item[table] A table.
\end{arguments}
%
\result%
An empty table.
%
\remarks%
All entries in {\em table\/} are deleted.  The result is \functionref{eq} to the
argument, which is to say that the argument is modified.

\derivedclass{hash-table}{table}
\index{general}{level-0 classes!\theclass{hash-table}}
%
Place holder for \classref{hash-table} class.

\end{optDefinition}

\gdef\module{level-0}\newpage\sclause{Vectors}
\label{vector}
\index{general}{vector}
\index{general}{vector!module}
\index{general}{level-0 modules!vector}
%
\begin{optDefinition}
The defined name of this module is {\tt vector}.  See also
section~\ref{collection} (collections) for further operations on
vectors.

\syntaxform{vector}
%
A vector\index{general}{external representation!vector} is written as
\verb+#(+{\em obj}$_1$ \ldots {\em obj}$_n$\verb+)+.  For example:
\verb+#(1 2 3)+ is a vector of three elements, the integers {\tt 1}, {\tt 2} and
{\tt 3}.  The representations of {\em obj$_i$} are determined by the external
representations defined in other sections of this definition (see
\classref{character}\ (\ref{character}), \classref{fixed-precision-integer}\
(\ref{fpi}), \classref{float}\ (\ref{float}), \classref{list}\ (\ref{list}),
\classref{string}\ (\ref{string}) and \classref{symbol}\ (\ref{symbol}), as well
as instances of \classref{vector}\ itself.  The syntax for the external
representation of vectors is defined below.
%
\Syntax
\label{vector-syntax}
\defSyntax{vector}{
\begin{syntaxx}
    \scdef{vector}: \\
    \>  \#( \scseqref{object} )
\end{syntaxx}}%
\showSyntaxBox{vector}
%
\derivedclass{vector}{sequence}
\index{general}{level-0 classes!\theclass{vector}}
%
The class of all instances of \classref{vector}.
%
\begin{initoptions}
    \item[size, \classref{fixed-precision-integer}] The number of elements in
    the vector.  Vectors are zero-based and thus the maximum index is {\em
        size-1}.  If not supplied the {\em size\/} is zero.

    \item[fill-value, \classref{object}] An object with which to initialize the
    vector.  The default fill value is \nil{}.
\end{initoptions}
%
\examples
\begin{tabular}{lcl}
    \verb|(make <vector>)| &\Ra& \verb|#()|\\
    \verb|(make <vector> 'size 2)| &\Ra& \verb|#(() ())|\\
    \verb|(make <vector> 'size 3| &\Ra& \verb|#(#\a #\a #\a)|\\
    \verb|  'fill-value #\a)|&&\\
\end{tabular}

\function{vectorp}
%
\begin{arguments}
    \item[object] Object to examine.
\end{arguments}
%
\result%
Returns {\em object\/} if it is a vector, otherwise \nil{}.

\constant{maximum-vector-index}{integer}
%
\remarks%
This is an implementation-defined constant.  A conforming processor must support
a maximum vector index of at least
32767\index{general}{implementation-defined!maximum vector
    index}\index{general}{conformity-clause!maximum vector index}.

\method{equal}
%
\begin{specargs}
    \item[vector$_1$, \classref{vector}] A vector.
    \item[vector$_2$, \classref{vector}] A vector.
\end{specargs}
%
\result%
If the size of {\em vector$_1$} is the same (under {\tt =}) as that of {\em
    vector$_2$}, then the result is the conjunction of the pairwise application
of \genericref{equal} to the elements of the arguments.  If not the result is
\nil{}.

\method{deep-copy}
%
\begin{specargs}
    \item[vector, \classref{vector}] A vector.
\end{specargs}
%
\result%
Constructs and returns a copy of {\em vector}, in which each element is the
result of calling {\em deep-copy\/} on the corresponding element of {\em
    vector}.

\method{shallow-copy}
%
\begin{specargs}
    \item[vector, \classref{vector}] A vector.
\end{specargs}
%
\result%
Constructs and returns a copy of {\em vector\/} in which each element is
\functionref{eql} to the corresponding element in {\em vector}.

\method{generic-prin}
%
\begin{specargs}
    \item[vector, \classref{vector}]
    A vector to be ouptut on stream.
    \item[stream, \classref{stream}]
    A stream on which the representation is to be output.
\end{specargs}
%
\result%
The vector supplied as the first argument.
%
\remarks%
Output the external representation of {\em vector\/} on {\em stream\/} as
described in the introduction to this section.  Calls the generic function again
to produce the external representation of the elements stored in the vector.

\method{generic-write}
%
\begin{specargs}
    \item[vector, \classref{vector}] A vector to be ouptut on stream.
    \item[stream, \classref{stream}] A stream on which the representation is to
    be output.
\end{specargs}
%
\remarks%
Output the external representation of {\em vector\/} on {\em stream\/} as
described in the introduction to this section.  Calls the generic function again
to produce the external representation of the elements stored in the vector.
%
\end{optDefinition}

\gdef\module{level-0}\newpage\input{level-0}

% -----------------------------------------------------------------------
%%% Programming Language EuLisp, Level-1
\clause{Programming Language EuLisp, Level-1}
\label{section:level-1}
This section describes the additions features in \eulisp\ level-1 including the
reflective aspects of the object system and how to program the metaobject
protocol and support for dynamic variable and related control forms.

\gdef\module{level-1}\sclause{Modules}
%
\defSyntax{defmodule-1}{
\begin{syntax}
    \scdef{defmodule-1-form}: \\
    \>  ( \syntaxref{defmodule} \scref{module-name} \\
    \>\>  \scref{module-directives} \\
    \>\>  \scseqref{level-1-module-form} ) \\
    \scdef{level-1-module-form}: \\
    \>  \scref{level-0-module-form} \\
    \>  \scref{level-1-form} \\
    \>  \scref{defining-1-form} \\
    \scdef{level-1-form}: \\
    \>  \scref{level-0-form} \\
    \>  \scref{special-1-form} \\
    \scdef{form}: \\
    \>  \scref{level-1-form} \\
    \scdef{special-form}: \\
    \>  \scref{special-1-form} \\
    \scdef{defining-1-form}: \\
    \>  \scref{defclass-1-form} \\
    \>  \scref{defgeneric-1-form} \\
    \>  \scref{defglobal-form} \\
    \scdef{special-1-form}: \\
    \>  \scref{generic-lambda-form} \\
    \>  \scref{method-lambda-form} \\
    \>  \scref{defmethod-form} \\
    \>  \scref{method-function-lambda-form} \\
    \>  \scref{catch-form} \\
    \>  \scref{throw-form}
\end{syntax}}
\showSyntaxBox{defmodule-1}

\gdef\module{level-1}\newpage\sclause{Classes and Objects}
%
\begin{optDefinition}
\defop{defclass}
\label{defclass-1}
%
\Syntax
\label{defclass-syntax}
\defSyntax{defclass-1}{
\begin{syntax}
    \scdef{defclass-1-form}: \\
    \>  ( \defopref{defclass} \scref{class-name} \scref{superclass-names} \\
    \>\>   ( \scseqref{slot-1} ) \\
    \>\>   \scseqref{class-option-1} ) \\
    \scdef{superclass-names}: \\
    \>  \nil{} \\
    \>  superclass-name \\
    \>  ( \scseqref{superclass-name} ) \\
    \scdef{slot-1}: \\
    \>  \scref{slot} \\
    \>  ( \scref{slot-name} \scseqref{slot-option-1} ) \\
    \scdef{slot-option-1}: \\
    \> \scref{slot-option} \\
    \> \scref{identifier} \scref{level-1-form} \\
    \scdef{class-option-1} \\
    \>  \scref{class-option} \\
    \>  \keywordref{class}: \scref{class-name} \\
    \>  \scref{identifier} \scref{level-1-form}
\end{syntax}}
\showSyntaxBox{defclass-1}
%
\begin{arguments}
    \item[\scref{superclass-names}] Either a symbol or a list of symbols naming
    bindings of the class/classes to be used as the superclass/superclasses of
    the new class (multiple
    inheritance\index{general}{inheritance!multiple}\index{general}{multiple
        inheritance}).  If this list of superclass names is empty, the
    superclass is \classref{object}.  This is different from \defopref{defclass}
    at level-0, where only one superclass may be specified.

    \item[\scref{slot-1}] A list of slot specifications (see
    below), comprising either a {\em slot-name} or a list of a {\em slot-name}
    followed by some {\em slot-option}s.  One of the class options (see below)
    allows the specification of the class of the slot description.

    \item[\scref{class-option-1}] A key and a value (see below).  One of
    the class options (\classref{class}) allows the specification of the class
    of the class being defined.
\end{arguments}
%
\remarks%
This defining form defines a new class.  The resulting class will be bound to
\scref{class-name}.  All the slot and class options are exactly the same as for
\defopref{defclass} (\ref{defclass}).

\noindent
The \scref{slot-option-1}\/s are interpreted as follows:
\begin{options}
    \item[\scref{identifier}, \scref{level-1-form}]%
    The symbol named by {\em identifier} and the value of {\em expression} are
    passed to \functionref{make} of the slot description class along with other
    slot options.  The values are evaluated in the lexical and dynamic
    environment of the \defopref{defclass}.  For the language defined slot
    description classes, no slot keywords are defined which are not specified by
    particular \defopref{defclass} slot options.
\end{options}
%
The \scref{class-option-1}\/s are interpreted as follows:
\begin{options}
    \item[\keyworddef{class}, \scref{class-name}]%
    The value of this option is the class of the new class.  By default, this is
    \classref{class}.  This option must only be specified once for the new
    class.

    \item[\scref{identifier}, \scref{level-1-form}]%
    The symbol named by {\em identifier} and the value of {\em expression} are
    passed to \functionref{make} on the class of the new class.  This list is
    appended to the end of the list that \defopref{defclass} constructs.  The
    values are evaluated in the lexical and dynamic environment of the
    \defopref{defclass}.  This option is used for metaclasses which need extra
    information not provided by the standard options.
\end{options}
%
\end{optDefinition}

\sclause{Generic Functions}

\begin{optDefinition}
\specop{generic-lambda}
\label{generic-lambda-1}
%
\Syntax
\defSyntax{generic-1-lambda}{
\begin{syntax}
    \scdef{generic-lambda-form}: \\
    \>  ( \specopref{generic-lambda} \scref{gf-lambda-list} \\
    \>\>  \scseqref{level-1-init-option} ) \\
    \scdef{level-1-init-option}: \\
    \>  class \scref{class-name} \\
    \>  method-class \scref{class-name} \\
    \>  method \scref{level-1-method-description} \\
    \>  \scref{identifier} \scref{level-1-form} \\
    \>  \scref{level-0-init-option} \\
    \scdef{level-1-method-description}: \\
    \>  ( \scseqref{method-init-option} \\
    \>\>  \scref{specialized-lambda-list} \\
    \>\>  \scref{body} ) \\
    \scdef{method-init-option}: \\
    \>  class \scref{class-name} \\
    \>  \scref{identifier} \scref{level-1-form}
\end{syntax}}
\showSyntaxBox{generic-1-lambda}
%
\begin{arguments}
    \item[gf-lambda-list] As level-0.  See section~\ref{defgeneric-0}.

    \item[level-1-init-option\/$^*$] Format as level-0, but with additional
    options, which are defined below.
\end{arguments}
%
\result%
A generic function.
%
\remarks%
The syntax of \specopref{generic-lambda} is an extension of the level-0
syntax allowing additional init-options.  These allow the
specification of the class of the new generic function, which defaults
to \classref{generic-function}, the class of all methods, which defaults
to \classref{method}, and non-standard options. The latter are evaluated
in the lexical and dynamic environment of \specopref{generic-lambda} and
passed to \functionref{make} of the generic function as additional
initialization arguments.  The additional {\em init-option\/}s over
level-0 are interpreted as follows:
%
\begin{options}

    \item[class, gf-class] The class of the new generic function.  This must be
    a subclass of \classref{generic-function}.  The default is
    \classref{generic-function}.

    \item[method-class, method-class] The class of all methods to be defined on
    this generic function.  All methods of a generic function must be instances
    of this class.  The {\em method-class} must be a subclass of
    \classref{method} and defaults to \classref{method}.

    \item[{\em identifier}, expression] The symbol named by {\em identifier} and
    the value of {\em expression} are passed to \functionref{make} as keywords.
    % and each new method.
    The values are evaluated in the lexical and dynamic environment of the
    \defopref{defgeneric}.  This option is used for classes which need extra
    information not provided by the standard options.
\end{options}
%
In addition, method init options can be specified for the individual
methods on a generic function.  These are interpreted as follows:
%
\begin{options}
    \item[class, method-class]%
    The class of the method to be defined. The method class must be a subclass
    of \classref{method} and is, by default, \classref{method}. The value is
    passed to \functionref{make} as the first argument. The symbol and the value
    are not passed as keywords to \functionref{make}.

    \item[{\em identifier}, expression]%
    The symbol named by {\em identifier} and the value of {\em expression} are
    passed to \functionref{make} creating a new method as keywords.  The values
    are evaluated in the lexical and dynamic environment of the
    \specopref{generic-lambda}.  This option is used for classes which need extra
    information not provided by the standard options.
\end{options}
%
\examples
 In the following example an anonymous version of {\tt gf-1} (see
\defopref{defgeneric}) is defined.  In all other respects the resulting object
is the same as {\tt gf-1}.
%
{\codeExample
(generic-lambda (arg1 (arg2 <class-a>))

  class <another-gf-class>
  class-key-a class-value-a
  class-key-b class-value-b

  method-class <another-method-class-a>

  method (class <another-method-class-b>
          method-class-b-key-a method-class-b-value-a
          ((m1-arg1 <class-b>) (m1-arg2 <class-c>))
          ...)
  method (method-class-a-key-a method-class-a-value-a
          ((m2-arg1 <class-d>) (m2-arg2 <class-e>))
          ...)
  method (class <another-method-class-c>
          method-class-c-key-a method-class-c-value-a
          ((m3-arg1 <class-f>) (m3-arg2 <class-g>))0
          ...)
)
\endCodeExample}
%
\seealso%
\defopref{defgeneric}.

\defop{defgeneric}
\label{defgeneric-1}
%
\Syntax
\defSyntax{defgeneric-1}{
\begin{syntax}
    \scdef{defgeneric-1-form}: \\
    \>  ( \defopref{defgeneric} \scref{gf-name} \scref{gf-lambda-list} \\
    \>\>  \scref{level-1-init-option} )
\end{syntax}}
\showSyntaxBox{defgeneric-1}
%
\begin{arguments}
    \item[gf name] As level-0.  See section~\ref{defgeneric-0}.

    \item[gf lambda list] As level-0.  See section~\ref{defgeneric-0}.

    \item[init option\/$^*$] As for \specopref{generic-lambda}, defined above.
    below.
\end{arguments}
%
\remarks%
This defining form defines a new generic function.  The resulting generic
function will be bound to {\em gf-name}.  The second argument is the formal
parameter list.  An error is signalled (condition:
\conditionref{non-congruent-lambda-lists}
\indexcondition{non-congruent-lambda-lists}) if any of the methods defined on
this generic function do not have lambda lists congruent to that of the generic
function.  This applies both to methods defined at the same time as the generic
function and to any methods added subsequently by \defopref{defmethod} or
\genericref{add-method}.  An {\em init-option} is a identifier followed by its
initial value.  The syntax of \defopref{defgeneric} is an extension of the
level-0 syntax.  The rewrite rules for the \defopref{defgeneric} form are
identical to those given in section~\ref{defgeneric-rewrite-rules} except that
{\em level 1 init option} replaces {\em level 0 init option}.
%
\examples
In the following example of the use of \defopref{defgeneric} a generic
function named {\tt gf-1} is defined.  The differences between this
function and {\tt gf-0} (see~\ref{defgeneric-0}) are
\begin{enumerate}
    \item The class of the generic function is specified
    (\theclass{another-gf-class}) along with some init-options related to the
    creation of an instance of that class.

    \item The default class of the methods to be attached to the generic
    function is specified (\theclass{another-method-class-a}) along with an
    init-option related to the creation of an instance of that class.

    \item In addition, some of the methods to be attached are of a different
    method class (\theclass{another-method-class-b} and
    \theclass{another-method-class-c}) also with method specific init-options.
    These method classes are subclasses of \theclass{another-method-class-a}.
\end{enumerate}
%
\begin{optPrivate}
{\begin{verbatim}
(defgeneric gf-1 (arg1 (arg2 <class-a>))
  class <another-gf-class>
  class-key-a class-value-a
  class-key-b class-value-b

  method-class <another-method-class>
  method-class-key-a method-class-value-a

  method (((m1-arg1 <class-b>) (m1-arg2 <class-c>))
          ...)
  method (((m2-arg1 <class-d>) (m2-arg2 <class-e>))
          ...)
  method (((m3-arg1 <class-f>) (m3-arg2 <class-g>))
          ...)
)\end{verbatim}}
%
\end{optPrivate}
{\codeExample
(defgeneric gf-1 (arg1 (arg2 <class-a>))

  class <another-gf-class>
  class-key-a class-value-a
  class-key-b class-value-b

  method-class <another-method-class-a>

  method (class <another-method-class-b>
          method-class-b-key-a method-class-b-value-a
          ((m1-arg1 <class-b>) (m1-arg2 <class-c>))
          ...)
  method (method-class-a-key-a method-class-a-value-a
          ((m2-arg1 <class-d>) (m2-arg2 <class-e>))
          ...)
  method (class <another-method-class-c>
          method-class-c-key-a method-class-c-value-a
          ((m3-arg1 <class-f>) (m3-arg2 <class-g>))
          ...)
)
\endCodeExample}

\sclause{Methods}

\specop{method-lambda}
\Syntax
\defSyntax{method-lambda}{
\begin{syntax}
    \scdef{method-lambda-form}: \ra{} \classref{function} \\
    \>  ( \specopref{method-lambda} \\
    \>\>  \scseqref{method-init-option} \\
    \>\>  \scref{specialized-lambda-list} \\
    \>\>  \scref{body} )
\end{syntax}}
\showSyntaxBox{method-lambda}
%
\begin{arguments}
    \item[method init option] A quoted symbol followed by an expression.

    \item[specialized lambda list] As defined under \specopref{generic-lambda}.

    \item[form] An expression.
\end{arguments}
%
\result%
This syntax creates and returns an anonymous method with the given
lambda list and body.  This anonymous method can later be added to a generic
function with a congruent lambda list via the generic function
\genericref{add-method}.  Note that the lambda list can be specialized to
specify the method's domain.  The value of the special keywords \classref{class}
determines the class to instantiate; the rest of the initlist is passed to
\functionref{make} called with this class.  The default method class is
\classref{method}.
%
\remarks%
The additional {\em method-init-option\/}s includes \classref{class}, for
specifying the class of the method to be defined, and non-standard
options, which are evaluated in the lexical and dynamic environment of
\specopref{method-lambda} and passed to \genericref{initialize} of that method.

\defop{defmethod}
\Syntax
\defSyntax{defmethod-1}{
\begin{syntax}
    \scdef{defmethod-1-form}: \\
    \>  ( \defopref{defmethod} \scref{gf-locator} \\
    \>\>  \scseqref{method-init-option} \\
    \>\>  \scref{specialized-lambda-list} \\
    \>\>  \scref{body} )
\end{syntax}}%
\showSyntaxBox{defmethod-1}%
%
\remarks%
The \defopref{defmethod} form of level-1 extends that of level-0 to accept
{\em method-init-option\/}s.  This allows for the specification of the
method class by means of the \classref{class} init option.  This class must
be a subclass of the method class of the host generic function. The
method class otherwise defaults to that of the host generic function.
In all other respects, the behaviour is as that defined in level-0.

\specop{method-function-lambda}
%
\begin{arguments}
    \item[lambda-list] A lambda list
    \item[form$^*$] A sequence of forms.
\end{arguments}
%
This macro creates and returns an anonymous method function with the given
lambda list and body. This anonymous method function can later be added to a
method using \setterref{method-function}, or as the {\tt function}
initialization value in a call of \functionref{make} on an instance of
\classref{method}.  A function of this type is also returned by the method
accessor \functionref{method-function}.  Only functions created using this macro
can be used as method functions.  Note that the lambda list must not be
specialized; a method's domain is stored in the method itself.

% hb: there are no setters specified yet!
\function{call-method}
%
\begin{arguments}
    \item[method] A method.
    \item[next-methods] A list of methods.
    \item[arg$^*$] A sequence of expressions.
\end{arguments}
%
This function calls the method {\em method} with arguments {\em args}.
The argument {\em next-methods} is a list of methods which are used as
the applicable method list for {\em args}; it is an error if this list
is different from the methods which would be produced by the method
lookup function of the generic function of {\em method}.  If {\em
method} is not attached to a generic function, its behavior is
unspecified.  The {\em next-methods} are used to determine the next
method to call when \specopref{call-next-method} is called within {\em
method-fn}.

\function{apply-method}
%
\begin{arguments}
    \item[method] A method.
    \item[next-methods] A list of methods.
    \item[form$_1$ \ldots form$_{n-1}$] A sequence of expressions.
    \item[form$_n$] An expression.
\end{arguments}
%
This function is identical to \functionref{call-method} except that its last
argument is a list whose elements are the other arguments to pass to
the method's method function.  The difference is the same as that
between normal function application and \functionref{apply}.
%
\end{optDefinition}

\gdef\module{level-1}\newpage\sclause{Object Introspection}
%
\begin{optDefinition}
\noindent
The only reflective capability which every object possesses is the ability to
find its class.

\function{class-of}
%
\begin{arguments}
    \item[object] An object.
\end{arguments}
%
\result%
The class of the object.
%
\remarks%
The function \functionref{class-of} can take any \lisp\ object as argument and
returns an instance of \classref{class} representing the class of that entity.
%
\end{optDefinition}

\sclause{Class Introspection}
%
\begin{optDefinition}
\noindent
Standard classes are not redefinable and support single inheritance only.
General multiple inheritance can be provided by extensions.  Nor is it possible
to use a class as a superclass which is not defined at the time of class
definition.  Again, such forward reference facilities can be provided by
extensions. The distinction between metaclasses and non-metaclasses is made
explicit by a special class, named \classref{metaclass}, which is the class of
all metaclasses. This is different from ObjVlisp, where whether a class is a
metaclass depends on the superclass list of the class in question.  It is
implementation-defined whether \classref{metaclass} itself is specializable or
not. This implies that implementations are free to restrict the instantiation
tree (excluding the selfinstantiation loop of \classref{metaclass}) to a depth
of three levels.  The metaclasses defined at level-1 are shown in
table~\ref{level-1-metaclass-hierarchy}.
%
\begin{table}[ht]
\caption{Level-1 metaclass hierarchy}
\label{level-1-metaclass-hierarchy}
{\tt
    \begin{tabbing}
        00\=00\=00\=00\= \kill
        $\cal A$ \classref{object}
        {\normalfont See level-0 table~\ref{level-0-class-hierarchy}} \\
        \>$\cal A$ \classref{class} \\
        \>\>$\cal C$ \classref{simple-class} \\
        \>\>$\cal C$ \classref{function-class}
    \end{tabbing}
}
\end{table}

The minimum information associated with a class metaobject is:
%
\begin{enumerate}
    \item The class precedence list, ordered most specific first, beginning with
    the class itself.
    \item The list of (effective) slot descriptions.
    \item The list of (effective) keywords.
\end{enumerate}
%
Standard classes support local slots only. Shared slots can be provided by
extensions.  The minimal information associated with a slot description
metaobject is:
%
\begin{enumerate}
    \item The name, which is required to perform inheritance computations.
    \item The default-function, called by default to compute the initial slot
    value when creating a new instance.
    \item The reader, which is a function to read the corresponding slot value
    of an instance.
    \item The writer, which is a function to write the corresponding slot of an
    instance.
    \item The keyword, which is a symbol to access the value which can be
    supplied to a \functionref{make} call in order to initialize the
    corresponding slot in a newly created object.
\end{enumerate}
%
The metaobject classes defined for slot descriptions at level-1 are shown in
table~\ref{level-1-class-hierarchy}.
%
\begin{table}[ht]
\caption{Level-1 class hierarchy}
\label{level-1-class-hierarchy}
{\tt
    \begin{tabbing}
        00\=00\=00\=00\= \kill
        $\cal A$ \classref{object}
        {\normalfont See level-0 table~\ref{level-0-class-hierarchy}} \\
        \>$\cal A$ \classref{slot} \\
        \>\>$\cal C$ \classref{local-slot} \\
        \>$\cal A$ \classref{function} \\
        \>\>$\cal A$ \classref{generic-function} \\
        \>\>\>$\cal C$ \classref{simple-generic-function} \\
        \>$\cal A$ \classref{method} \\
        \>\>$\cal C$ \classref{simple-method}
    \end{tabbing}
}
\end{table}

\class{metaclass}
%
Place holder for \classref{metaclass}.

\derivedclass{simple-class}{class}
%
Place holder for \classref{simple-class}.

\derivedclass{function-class}{class}
%
Place holder for \classref{function-class}.

\function{class-name}
%
\begin{arguments}
    \item[class] A class.
\end{arguments}
%
\result%
Returns a {\em string\/} which is \methodref{binary=}{string} to that given as
the argument to the call to \functionref{defclass} which created {\em class}.
It is an error to modify this string.

\function{class-precedence-list}
%
\begin{arguments}
    \item[class] A class.
\end{arguments}
%
\result%
A list of classes, starting with {\em class} itself, succeeded by the
superclasses of {\em class} and ending with \classref{object}. This list is
equivalent to the result of calling \functionref{compute-class-precedence-list}.
%
\remarks%
The class precedence list is used to control the inheritance of slots and
methods.

\function{class-slots}
%
\begin{arguments}
    \item[class] A class.
\end{arguments}
%
\result%
A list of slots, one for each of the slots of an instance
of {\em class}.
%
\remarks%
The slots determine the instance size (number of slots)
and the slot access.

\function{class-keywords}
%
\begin{arguments}
    \item[class] A class.
\end{arguments}
%
\result%
A list of symbols, which can be used as legal keywords to initialize
instances of the class.
%
\remarks%
The keywords correspond to the keywords specified in the {\tt keyword}
slot-option or the {\tt keywords} class-option when the
class and its superclasses were defined.
%
\end{optDefinition}

\sclause{Slot Introspection}
%
\begin{optDefinition}
\class{slot}

The abstract class of all slot descriptions.

\derivedclass{local-slot}{slot}

The class of all local slot descriptions.
%
\begin{initoptions}
    \item[name, string] The name of the slot.%; useful for debugging only.
    \item[reader, function] The function to access the slot.
    \item[writer, function] The function to update the slot.
    \item[default-function, function] The function to compute the initial value in
    the absence of a supplied value.
    \item[keyword, symbol] The key to access a supplied initial value.
\end{initoptions}

The default value for all initoptions is \nil{}.

\function{slot-name}
%
\begin{arguments}
    \item[slot] A slot description.
\end{arguments}
%
\result%
The symbol which was used to name the slot when the class, of which
the slot is part, was defined.
%
\remarks%
The slot description name is used to identify a slot description in a
class. It has no effect on bindings.

\function{slot-default-function}
%
\begin{arguments}
    \item[slot] A slot description.
\end{arguments}
%
\result%
A function of no arguments that is used to compute the initial value of the
slot in the absence of a supplied value.

\function{slot-slot-reader}
%
\begin{arguments}
    \item[slot] A slot description.
\end{arguments}
%
\result%
A function of one argument that returns the value of the slot
in that argument.

\function{slot-slot-writer}
%
\begin{arguments}
    \item[slot] A slot description.
\end{arguments}
%
\result%
A function of two arguments that installs the second argument as the value of
the slot in the first argument.
\end{optDefinition}

\sclause{Generic Function Introspection}
%
\begin{optDefinition}
The default generic dispatch scheme is class-based; that is, methods
are class specific.  The default argument precedence order is
left-to-right.

The minimum information associated with a generic function metaobject is:

\begin{enumerate}
    \item The domain, restricting the domain of each added method to a
    sub-domain.
    \item The method class, restricting each added method to be an instance
    of that class.
    \item The list of all added methods.
    \item The method look-up function used to collect and sort the
    applicable methods for a given domain.
    \item The discriminating function used to perform the generic dispatch.
\end{enumerate}

\function{generic-function-domain}
%
\begin{arguments}
    \item[generic-function] A generic function.
\end{arguments}
%
\result%
A list of classes.
%
\remarks%
This function returns the domain of a generic function. The domains of all
methods attached to a generic function are constrained to be within this domain.
In other words, the domain classes of each method must be subclasses of the
corresponding generic function domain class.  It is an error to modify this
list.

\function{generic-function-method-class}
%
\begin{arguments}
    \item[generic-function] A generic function.
\end{arguments}
%
\result%
This function returns the class which is the class of all methods of the generic
function.  Each method attached to a generic function must be an instance of
this class. When a method is created using \defopref{defmethod},
\specopref{method-lambda}, or by using the {\tt method} generic function
option in a \defopref{defgeneric} or \specopref{generic-lambda}, it will be an
instance of this class by default.

\function{generic-function-methods}
%
\begin{arguments}
    \item[generic-function] A generic function.
\end{arguments}
%
\result%
This function returns a list of the methods attached to the generic function.
The order of the methods in this list is undefined.  It is an error to modify
this list.

\function{generic-function-method-lookup-function}
%
\begin{arguments}
    \item[generic-function] A generic function.
\end{arguments}
%
\result%
A function.
%
\remarks%
This function returns a function which, when applied to the arguments given to
the generic function, returns a sorted list of applicable methods.  The order of
the methods in this list is determined by
\genericref{compute-method-lookup-function}.
%How many arguments does this function take --- 1 or n?  hb: n

\function{generic-function-discriminating-function}
%
\begin{arguments}
    \item[generic-function] A generic function.
\end{arguments}
%
\result%
A function.
%
\remarks%
This function returns a function which may be applied to the same arguments as
the generic function.  This function is called to perform the generic dispatch
operation to determine the applicable methods whenever the generic function is
called, and call the most specific applicable method function.  This function is
created by \genericref{compute-discriminating-function}.

\end{optDefinition}

\sclause{Method Introspection}
\begin{optDefinition}
The minimal information associated with a method metaobject is:

\begin{enumerate}
    \item The domain, which is a list of classes.
    \item The function comprising the code of the method.
    \item The generic function to which the method has been added, or \nil{}\/ if
    it is attached to no generic function.
\end{enumerate}
%
The metaobject classes for generic functions defined at level-1 are
shown in table~\ref{level-1-class-hierarchy}.

\class{method}
%
Place holder for \classref{method}.

\derivedclass{simple-method}{method}
%
Place holder for \classref{method-class}.

\function{method-domain}
%
\begin{arguments}
    \item[method] A method.
\end{arguments}
%
\result%
A list of classes defining the domain of a method.

\function{method-function}
%
\begin{arguments}
    \item[method] A method.
\end{arguments}
%
\result%
This function returns a function which implements the method.  The returned
function which is called when {\em method} is called, either by calling the
generic function with appropriate arguments, through a
\specopref{call-next-method}, or by using \functionref{call-method}.  A method
metaobject itself cannot be applied or called as a function.

\setter{method-function}
%
\begin{arguments}
    \item[method] A method.
    \item[function] A function.
\end{arguments}
%
\result%
This function sets the function which implements the method.

\function{method-generic-function}
%
\begin{arguments}
    \item[method] A method.
\end{arguments}
%
\result%
This function returns the generic function to which {\em method} is attached; if
{\em method} is not attached to a generic function, it returns \nil{}.
\end{optDefinition}

\sclause{Class Initialization}
%
\begin{optDefinition}
\method{initialize}{class}
%
\begin{specargs}
    \item[class, \classref{class}] A class.
    \item[initlist, \classref{list}] A list of initialization options as
    follows:

    \begin{options}
        \item[name, symbol] Name of the class being initialized.
        \item[direct-superclasses, list] List of direct superclasses.
        \item[direct-slots, list] List of direct slot specifications.
        \item[direct-keywords, list] List of direct keywords.
    \end{options}
\end{specargs}
%
\result%
The initialized class.
%
\remarks%
The initialization of a class takes place as follows:
%
\begin{enumerate}
    \item Check compatibility of direct superclasses
    % \item Transform textual slot descriptions into slot description objects
    % (this is better done under the control of class beeing initialized).

    \item Perform the logical inheritance computations of:
    \begin{enumerate}
        \item class precedence list
        \item keywords
        \item slot descriptions
    \end{enumerate}

    \item Compute new slot accessors and ensure all (new and inherited)
    accessors to work correctly on instances of the new class.

    % \item Compute predicates and constructors if required.

    \item Make the results accessible by class readers.
\end{enumerate}
%
The basic call structure is laid out in figure~\ref{call-structure}
%
\begin{optPrivate}
\begin{verbatim}
COMPATIBLE-SUPERCLASSES-P cl direct-superclasses -> boolean
  COMPATIBLE-SUPERCLASS-P cl superclass -> boolean
COMPUTE-CLASS-PRECEDENCE-LIST
cl direct-superclasses -> list(class)
COMPUTE-INHERITED-KEYWORDS
cl direct-superclasses -> list(list(keyword))
COMPUTE-KEYWORDS
cl direct-keywords <inherited-keywords> -> list(keyword)
COMPUTE-INHERITED-SLOTS
cl direct-superclasses -> list(list(slotd))
COMPUTE-SLOTS
cl direct-slotds <inherited-slotds> -> list(slotd)
  COMPUTE-DEFINED-SLOT
  cl slotd-init-list -> slotd
    COMPUTE-DEFINED-SLOT-CLASS
    cl slotd-init-list -> slotd-class
  -or-
  COMPUTE-SPECIALIZED-SLOT
  cl inherited-slotds slotd-init-list -> slotd
    COMPUTE-SPECIALIZED-SLOT-CLASS
    cl inherited-slotds slotd-init-list -> slotd-class
COMPUTE-AND-ENSURE-SLOT-ACCESSORS
cl <effective-slotds> <inherited-slotds> -> list(slotd)
  COMPUTE-SLOT-READER
  cl slotd <effective-slotds> -> function
  COMPUTE-SLOT-WRITER
  cl slotd <effective-slotds> -> function
  ENSURE-SLOT-READER
  cl slotd <effective-slotds> reader -> function
    COMPUTE-PRIMITIVE-READER-USING-SLOT
    slotd cl <effective-slotds> -> function
      COMPUTE-PRIMITIVE-READER-USING-CLASS
      cl slotd <effective-slotds> -> function
  ENSURE-SLOT-WRITER cl slotd writer -> function
    COMPUTE-PRIMITIVE-WRITER-USING-SLOT
    slotd cl <effective-slotds> -> function
      COMPUTE-PRIMITIVE-WRITER-USING-CLASS
      cl slotd <effective-slotds> -> function
\end{verbatim}
\end{optPrivate}
%
\begin{table}%
\caption{Initialization Call Structure}%
\label{call-structure}%
\tt%
\begin{tabbing}%
    00\=00\=00\=00\=00\=00\=00\=00\= \kill
    {\tt compatible-superclasses?}\\
    \>\>\>\>{\em cl direct-superclasses\/}
    $\rightarrow$ {\em boolean}\\
    \>{\tt compatible-superclass?}\\
    \>\>\>\>\> {\em cl superclass\/}
    $\rightarrow$ {\em boolean}\\
    %\hline
    {\tt compute-class-precedence-list}\\
    \>\>\>\>{\em cl direct-superclasses\/}
    $\rightarrow$ ({\em cl\/}$^*$)\\
    %\hline
    {\tt compute-inherited-keywords}\\
    \>\>\>\>{\em cl direct-superclasses\/}
    $\rightarrow$ (({\em keyword\/}$^*$)$^*$)\\
    %\hline
    {\tt compute-keywords}\\
    \>\>\>\>{\em cl direct-keywords inherited-keywords\/}\\
    \>\>\>\>$\rightarrow$ ({\em keyword\/}$^*$)\\
    %\hline
    {\tt compute-inherited-slots}\\
    \>\>\>\>{\em cl direct-superclasses\/}
    $\rightarrow$ (({\em sd\/}$^*$)$^*$)\\
    %\hline
    {\tt compute-slots}\\
    \>\>\>\>{\em cl slot-specs inherited-sds\/}
    $\rightarrow$ ({\em sd\/}$^*$)\\
    \>{\em either}\\
    \>\>{\tt compute-defined-slot}\\
    \>\>\>\>\>\>{\em cl slot-spec\/}
    $\rightarrow$ {\em sd}\\
    \>\>\>{\tt compute-defined-slot-description-class}\\
    \>\>\>\>\>\>\>{\em cl slot-spec\/}
    $\rightarrow$ {\em sd-class}\\
    \>{\em or}\\
    \>\>{\tt compute-specialized-slot}\\
    \>\>\>\>\>\>{\em cl inherited-sds slot-spec\/}
    $\rightarrow$ {\em sd}\\
    \>\>\>{\tt compute-specialized-slot-class}\\
    \>\>\>\>\>\>\>{\em cl inherited-sds slot-spec\/}\\
    \>\>\>\>\>\>\>$\rightarrow$ {\em sd-class}\\
    %\hline
    {\tt compute-instance-size}\\
    \>\>\>\>{\em cl effective-sds\/}
    $\rightarrow$ {\em integer}\\
    %\hline
    {\tt compute-and-ensure-slot-accessors}\\
    \>\>\>\>{\em cl effective-sds inherited-sds\/}
    $\rightarrow$ ({\em sd\/}$^*$)\\
    \>{\tt compute-slot-reader}\\
    \>\>\>\>\>{\em cl sd effective-sds\/}
    $\rightarrow$ {\em function}\\
    \>{\tt compute-slot-writer}\\
    \>\>\>\>\>{\em cl sd effective-sds\/}
    $\rightarrow$ {\em function}\\
    \>{\tt ensure-slot-reader}\\
    \>\>\>\>\>{\em cl sd effective-sds reader\/}
    $\rightarrow$ {\em function}\\
    \>\>{\tt compute-primitive-reader-using-slot}\\
    \>\>\>\>\>\>{\em sd cl effective-sds\/}
    $\rightarrow$ {\em function}\\
    \>\>\>{\tt compute-primitive-reader-using-class}\\
    \>\>\>\>\>\>\>{\em cl sd effective-sds\/}
    $\rightarrow$ {\em function}\\
    \>{\tt ensure-slot-writer}\\
    \>\>\>\>\>{\em cl sd effective-sds writer\/}
    $\rightarrow$ {\em function}\\
    \>\>{\tt compute-primitive-writer-using-slot}\\
    \>\>\>\>\>\>{\em sd cl effective-sds\/}
    $\rightarrow$ {\em function}\\
    \>\>\>{\tt compute-primitive-writer-using-class}\\
    \>\>\>\>\>\>\>{\em cl sd effective-sds\/}
    $\rightarrow$ {\em function}%
\end{tabbing}%
\end{table}%
%
Note that \genericref{compute-keywords} is called by the default
\genericref{initialize} method with all direct keywords as the second argument:
those specified as slot option and those specified as class option.

\generic{compute-predicate}
%
\begin{genericargs}
    \item[class, \classref{class}] A class.
\end{genericargs}
%
\result%
Computes and returns a function of one argument, which is a predicate
function for {\em class}.

\method{compute-predicate}{class}
%
\begin{specargs}
    \item[class, \classref{class}] A class.
\end{specargs}
%
\result%
Computes and returns a function of one argument, which returns \true{} when applied
to direct or indirect instances of {\em class} and \nil{}\/ otherwise.

\generic{compute-constructor}
%
\begin{genericargs}
    \item[class, \classref{class}] A class.
    \item[parameters, \classref{list}] The argument list of the function being
    created.
\end{genericargs}
%
\result%
Computes and returns a constructor function for {\em class}.

\method{compute-constructor}{class}
%
\begin{specargs}
    \item[class, \classref{class}] A class.
    \item[parameters, \classref{list}] The argument list of the function being
    created.
\end{specargs}
%
\result%
Computes and returns a constructor function, which returns a new
instance of {\em class}.

\generic{allocate}
%
\begin{genericargs}
    \item[class, \classref{class}] A class.
    \item[initlist, \classref{list}] A list of initialization arguments.
\end{genericargs}
%
\result%
An instance of the first argument.
%
\remarks%
Creates an instance of the first argument.  Users may define new
methods for new metaclasses.

\method{allocate}{class}
%
\begin{specargs}
    \item[class, \classref{class}] A class.
    \item[initlist, \classref{list}] A list of initialization arguments.
\end{specargs}
%
\result%
An instance of the first argument.
%
\remarks%
The default method creates a new uninitialized instance of the first
argument.  The initlist is not used by this \genericref{allocate} method.
%
\end{optDefinition}

\sclause{Slot Description Initialization}
%
\begin{optDefinition}
%
\method{initialize}{slot}
%
\begin{specargs}
    \item[slot, \classref{slot}] A slot description.
    \item[initlist, \classref{list}] A list of initialization options as
    follows:
    \begin{options}
        \item[name, symbol] The name of the slot.
        \item[default-function, function] A function.
        \item[keyword, symbol] A symbol.
        \item[reader, function] A slot reader function.
        \item[writer, function] A slot writer function.
    \end{options}
    % \item[initlist] A list of initialization arguments.
\end{specargs}
%
\result%
The initialized slot description.
%
\end{optDefinition}

\sclause{Generic Function Initialization}
%
\begin{optDefinition}

\method{initialize}{generic-function}
%
\begin{specargs}
    \item[gf, \classref{generic-function}] A generic function.

    \item[initlist, \classref{list}] A list of initialization options as
    follows:
    \begin{options}
        \item[name, symbol] The name of the generic function.
        \item[domain, list] List of argument classes.
        % \item[range, class] The class of the result.
        \item[method-class, class] Class of attached methods.
        \item[method, method-description] A method to be attached.  This
        option may be specified more than once.
        % hb: no!
    \end{options}
\end{specargs}
%
\result%
The initialized generic function.
%
\remarks%
This method initializes and returns the {\em generic-function}.  The specified
methods are attached to the generic function by \genericref{add-method}, and its
slots are initialized from the information passed in {\em initlist} and from the
results of calling \genericref{compute-method-lookup-function} and
\genericref{compute-discriminating-function} on the generic function.  Note that
these two functions may not be called during the call to
\genericref{initialize}, and that they may be called several times for the
generic function.

The basic call structure is:
\smallskip
\noindent
{\tt%
    add-method {\em gf method\/} -> {\em gf}\\
    compute-method-lookup-function {\em gf domain \/} -> {\em function}\\
    compute-discriminating-function {\em gf domain lookup-fn methods\/} -> {\em function}}
\smallskip
%
\end{optDefinition}

\sclause{Method Initialization}
%
\begin{optDefinition}
%
\method{initialize}{method}
%
\begin{specargs}
    \item[method, \classref{method}] A method.

    \item[initlist, \classref{list}] A list of initialization options as
    follows:
    \begin{options}
        \item[domain, list] The list of argument classes.
        % \item[range, class] The class of the result.
        \item[function, fn] A function, created with
        \specopref{method-function-lambda}.
        \item[generic-function, gf] A generic function.
    \end{options}
\end{specargs}
%
\result%
This method returns the initialized method metaobject {\em method}.  If the {\em
    generic-function\/} option is supplied, \genericref{add-method} is called to
install the new method in the {\em generic-function}.
%
\end{optDefinition}

\sclause{Inheritance Protocol}
%
\begin{optDefinition}
%
\generic{compatible-superclasses?}
%
\begin{genericargs}
    \item[class, \classref{class}] A class.
    \item[direct-superclasses, \classref{list}] A list of potential direct
    superclasses of {\em class}.
\end{genericargs}
%
\result%
Returns {\em t} if {\em class} is compatible with {\em
direct-superclasses}, otherwise \nil{}.

\method{compatible-superclasses?}{class}
%
\begin{specargs}
    \item[class, \classref{class}] A class.
    \item[direct-superclasses, \classref{list}] A list of potential direct
    superclasses.
\end{specargs}
%
\result%
Returns the result of calling \genericref{compatible-superclass?} on {\em
    class} and the first element of the {\em direct-superclasses} (single
inheritance assumption).

\generic{compatible-superclass?}
%
\begin{genericargs}
    \item[subclass, \classref{class}] A class.
    \item[superclass, \classref{class}] A potential direct superclass.
\end{genericargs}
%
\result%
Returns \true\/ if {\em subclass} is compatible with {\em superclass},
otherwise \nil{}.

\method{compatible-superclass?}{class}
%
\begin{specargs}
    \item[subclass, \classref{class}] A class.
    \item[superclass, \classref{class}] A potential direct superclass.
\end{specargs}
%
\result%
Returns \true\/ if the class of the first argument is a subclass of
the class of the second argument, otherwise \nil{}.

If the implementation wishes to restrict the instantiation tree (see
introduction to B.4), this method should return \nil{}\/ if {\em superclass}
is \theclass{metaclass}.

\method{compatible-superclass?}{class}
%
\begin{specargs}
    \item[subclass, \classref{class}] A class.
    \item[superclass, \theclass{abstract-class}] A potential direct superclass.
\end{specargs}
%
\result%
Always returns \true.

\method{compatible-superclass?}{abstract-class}
%
\begin{specargs}
    \item[subclass, \theclass{abstract-class}] A class.
    \item[superclass, \classref{class}] A potential direct superclass.
\end{specargs}
%
\result%
Always returns \nil{}.

\method{compatible-superclass?}{abstract-class}
%
\begin{specargs}
    \item[subclass, \theclass{abstract-class}] A class.
    \item[superclass, \theclass{abstract-class}] A potential direct superclass.
\end{specargs}
%
\result%
Always returns \true.

\generic{compute-class-precedence-list}
%
\begin{genericargs}
    \item[class, \classref{class}] Class being defined.
    \item[direct-superclasses, \classref{list}] List of direct superclasses.
\end{genericargs}
%
\result%
Computes and returns a list of classes which represents the linearized
inheritance hierarchy of {\em class} and the given list of direct superclasses,
beginning with {\em class} and ending with \classref{object}.

\method{compute-class-precedence-list}{list}
%
\begin{specargs}
    \item[class, \classref{class}] Class being defined.
    \item[direct-superclasses, \classref{list}] List of direct superclasses.
\end{specargs}
%
\result%
A list of classes.
%
\remarks%
This method can be considered to return a
cons of {\em class} and the class precedence list of the first element
of {\em direct-superclasses} (single inheritance assumption). If no {\em
direct-superclasses} has been supplied, the result is the list of two elements:
{\em class} and \classref{object}.

\generic{compute-slots}
%
\begin{genericargs}
    \item[class, \classref{class}] Class being defined.
    \item[direct-slot-specifications, \classref{list}] A list of direct slot
    specification.
    \item[inherited-slots, \classref{list}] A list of lists of inherited slot
    descriptions.
\end{genericargs}
%
\result%
Computes and returns the list of effective slot descriptions of {\em class}.
%
\seealso%
\genericref{compute-inherited-slots}.

\method{compute-slots}{class}
%
\begin{specargs}
    \item[class, \classref{class}] Class being defined.
    \item[slot-specs, \classref{list}] List of (direct) slot specifications.
    \item[inherited-slot-lists, \classref{list}] A list of lists (in
    fact one list in single inheritance) of inherited slot descriptions.
\end{specargs}
%
\result%
A list of effective slot descriptions.
%
\remarks%
% Update!
The default method computes two sublists:
\begin{enumerate}
    \item Calling \genericref{compute-specialized-slot} with the
    three arguments (i) {\em class}, (ii) each {\em inherited-slot}
    as a singleton list, (iii) the {\em slot-spec} corresponding (by having the
    same name) to the slot description, if it exists, otherwise \nil{}, giving a
    list of the specialized slot descriptions.

    \item Calling \genericref{compute-defined-slot} with the three
    arguments (i) {\em class}, (ii) each {\em slot-specification} which does not
    have a corresponding (by having the same name) {\em
        inherited-slot}.
\end{enumerate}
%
The method returns the concatenation of these two lists as its result.  The
order of elements in the list is significant. All specialized slot descriptions
have the same position as in the effective slot descriptions list of the direct
superclass (due to the single inheritance). The slot accessors (computed later)
may rely on this assumption minimizing the number of methods to one for all
subclasses and minimizing the access time to an indexed reference.
%
\seealso%
\genericref{compute-specialized-slot},
\genericref{compute-defined-slot},
\genericref{compute-and-ensure-slot-accessors}.

\generic{compute-keywords}
%
\begin{genericargs}
    \item[class, \classref{class}] Class being defined.
    \item[keywords, \classref{list}] List of direct keywords.
    \item[inherited-keyword-lists, \classref{list}] A list of lists of inherited
    keywords.
\end{genericargs}
%
\result%
List of symbols.
%
\remarks%
Computes and returns all legal keywords for {\em class}.
%
\seealso%
\genericref{compute-inherited-keywords}.

\method{compute-keywords}{class}
%
\begin{specargs}
    \item[class, \classref{class}] Class being defined.
    \item[keywords, \classref{list}] List of direct keywords.
    \item[inherited-keyword-lists, \classref{list}] A list of lists of inherited
    keywords.
\end{specargs}
%
\result%
List of symbols.
%
\remarks%
This method appends the second argument with the first element of the third
argument (single inheritance assumption), removes duplicates and returns the
result. Note that \genericref{compute-keywords} is called by the default
\genericref{initialize} method with all direct keywords as the second argument:
those specified as slot option and those specified as class option.

\generic{compute-inherited-slots}
%
\begin{genericargs}
    \item[class, \classref{class}] Class being defined.
    \item[direct-superclasses, \classref{list}] List of direct superclasses.
\end{genericargs}
%
\result%
List of lists of inherited slot descriptions.
%
\remarks%
Computes and returns a list of lists of effective slot descriptions.
%
\seealso%
\genericref{compute-slots}.

\method{compute-inherited-slots}{class}
%
\begin{specargs}
    \item[class, \classref{class}] Class being defined.
    \item[direct-superclasses, \classref{list}] List of direct superclasses.
\end{specargs}
%
\result%
List of lists of inherited slot descriptions.
%
\remarks%
The result of the default method is a list of one element: a list of effective
slot descriptions of the first element of the second argument (single
inheritance assumption). Its result is used by
\genericref{compute-slots} as an argument.

\generic{compute-inherited-keywords}
%
\begin{genericargs}
    \item[class, \classref{class}] Class being defined.
    \item[direct-superclasses, \classref{list}] List of direct superclasses.
\end{genericargs}
%
\result%
List of lists of symbols.
\remarks%
Computes and returns a list of lists of keywords. Its result is used by
\genericref{compute-keywords} as an argument.
%
\seealso%
\genericref{compute-keywords}.

\method{compute-inherited-keywords}{class}
%
\begin{specargs}
    \item[class, \classref{class}] Class being defined.
    \item[direct-superclasses, \classref{list}] List of direct superclasses.
\end{specargs}
%
\result%
List of lists of symbols.
%
\remarks%
The result of the default method
contains one list of legal keywords of the first element of the second
argument (single inheritance assumption).

\generic{compute-defined-slot}
%
\begin{genericargs}
    \item[class, \classref{class}] Class being defined.
    \item[slot-spec, \classref{list}] Canonicalized slot specification.
\end{genericargs}
%
\result%
Slot description.
%
\remarks%
Computes and returns a new effective slot description.  It is called by
\genericref{compute-slots} on each slot specification which has no
corresponding inherited slot descriptions.  \seealso%
\genericref{compute-defined-slot-class}.

\method{compute-defined-slot}{class}
%
\begin{specargs}
    \item[class, \classref{class}] Class being defined.
    \item[slot-spec, \classref{list}] Canonicalized slot specification.
\end{specargs}
%
\result%
Slot description.
%
\remarks%
Computes and returns a new effective slot description.  The class of the result
is determined by calling \genericref{compute-defined-slot-class}.
%
\seealso%
\genericref{compute-defined-slot-class}.

\generic{compute-defined-slot-class}
%
\begin{genericargs}
    \item[class, \classref{class}] Class being defined.
    \item[slot-spec, \classref{list}] Canonicalized slot specification.
\end{genericargs}
%
\result%
Slot description class.
%
\remarks%
Determines and returns the slot description class corresponding to {\em class}
and {\em slot-spec} .  \seealso \genericref{compute-defined-slot}.

\method{compute-defined-slot-class}{class}
%
\begin{specargs}
    \item[class, \classref{class}] Class being defined.
    \item[slot-spec, \classref{list}] Canonicalized slot specification.
\end{specargs}
%
\result%
The class \classref{local-slot}.
%
\remarks%
This method just returns the class \classref{local-slot}.

\generic{compute-specialized-slot}
%
\begin{genericargs}
    \item[class, \classref{class}] Class being defined.
    \item[inherited-slots, \classref{list}] List of inherited slot descriptions
    (each of the same name as the slot being defined).
    \item[slot-spec, \classref{list}] Canonicalized slot specification or
    \nil{}.
\end{genericargs}
%
\result%
Slot description.
%
\remarks%
Computes and returns a new effective slot description. It is called by
\genericref{compute-slots} on the class, each list of inherited
slots with the same name and with the specialising slot specification
list or \nil{}\/ if no one is specified with the same name.
%
\seealso%
\genericref{compute-specialized-slot-class}.

\method{compute-specialized-slot}{class}
%
\begin{specargs}
    \item[class, \classref{class}] Class being defined.
    \item[inherited-slots, \classref{list}] List of inherited slot descriptions.
    \item[slot-spec, \classref{list}] Canonicalized sdirect-lot specification or
    \nil{}.
\end{specargs}
%
\result%
Slot description.
%
\remarks%
Computes and returns a new effective slot description.  The class of the result
is determined by calling
\genericref{compute-specialized-slot-class}.
%
\seealso%
\genericref{compute-specialized-slot-class}.

\generic{compute-specialized-slot-class}
%
\begin{genericargs}
    \item[class, \classref{class}] Class being defined.
    \item[inherited-slots, \classref{list}] List of inherited slot descriptions.
    \item[slot-spec, \classref{list}] Canonicalized slot specification or
    \nil{}.
\end{genericargs}
%
\result%
Slot description class.
%
\remarks%
Determines and returns the slot description class corresponding to (i)
the class being defined, (ii) the inherited slot descriptions being
specialized (iii) the specializing information in {\em slot-spec}.
%
\seealso%
\genericref{compute-specialized-slot}.

\method{compute-specialized-slot-class}{class}
%
\begin{specargs}
    \item[class, \classref{class}] Class being defined.
    \item[inherited-slots, \classref{list}] List of inherited slot
    descriptions.
    \item[slot-spec, \classref{list}] Canonicalized slot specification or
    \nil{}.
\end{specargs}
%
\result%
The class \classref{local-slot}.
%
\remarks%
This method just returns the class \classref{local-slot}.
%
\end{optDefinition}

\sclause{Slot Access Protocol}
%
\begin{optDefinition}
The slot access protocol is defined via accessors (readers and writers)
only. There is no primitive like CLOS's {\tt slot-value}. The accessors are
generic for standard classes, since they have to work on subclasses and should
do the applicability check anyway.  The key idea is that the discrimination on
slots and classes is performed once at class definition time rather
than again and again at slot access time.

Each slot has exactly one reader and one writer as anonymous
objects. If a reader/writer slot-option is specified in a class definition, the
anonymous reader/writer of that slot is bound to the specified
identifier. Thus, if a reader/writer option is specified more than once, the
same object is bound to all the identifiers. If the accessor slot-option is
specified the anonymous writer will be installed as the setter of the reader.
Specialized slots refer to the same objects as those in the
superclasses (single inheritance makes that possible).  Since the
readers/writers are generic, it is possible for a subclass (at the meta-level)
to add new methods for inherited slots in order to make the
readers/writers applicable on instances of the subclass. A new method might be
necessary if the subclasses have a different instance allocation or if the slot
positions cannot be kept the same as in the superclass (in multiple inheritance
extensions).  This can be done during the initialization computations.

\generic{compute-and-ensure-slot-accessors}
%
\begin{genericargs}
    \item[class, \classref{class}] Class being defined.
    \item[slots, \classref{list}] List of effective slot
    descriptions.
    \item[inherited-slots, \classref{list}] List of lists of
    inherited slot descriptions.
\end{genericargs}
%
\result%
List of effective slot descriptions.
%
\remarks%
Computes new accessors or ensures that inherited accessors work
correctly for each effective slot description.

\method{compute-and-ensure-slot-accessors}{class}
%
\begin{specargs}
    \item[class, \classref{class}] Class being defined.
    \item[slots, \classref{list}] List of effective slot
    descriptions.
    \item[inherited-slots, \classref{list}] List of lists of
    inherited slot descriptions.
\end{specargs}
%
\result%
List of effective slot descriptions.
%
\remarks%
For each slot description in {\em slots\/} the default method
checks if it is a new slot description and not an
inherited one. If the slot description is new,
\begin{enumerate}
    % \item computes the slot position in an instance
    % calling \genericref{compute-slot-position}, and stores the result;

    \item calls \genericref{compute-slot-reader} to compute a new slot reader
    and stores the result in the slot description;

    \item calls \genericref{compute-slot-writer} to compute a new slot writer
    and stores the result in the slot description;
\end{enumerate}
%
Otherwise, it assumes that the inherited values remain valid.

Finally, for every slot description (new or inherited) it ensures the reader and
writer work correctly on instances of {\em class} by means of
\genericref{ensure-slot-reader} and \genericref{ensure-slot-writer}.

\generic{compute-slot-reader}
%
\begin{genericargs}
    \item[class, \classref{class}] Class.
    \item[slot, \classref{slot}] Slot description.
    \item[slot-list, \classref{list}] List of effective slot descriptions.
\end{genericargs}
%
\result%
Function.
%
\remarks%
Computes and returns a new slot reader applicable to instances of {\em class}
returning the slot value corresponding to {\em slot}. The third
argument can be used in order to compute the logical slot position.

\method{compute-slot-reader}{class}
%
\begin{specargs}
    \item[class, \classref{class}] Class.
    \item[slot, \classref{slot}] Slot description.
    \item[slots, \classref{list}] List of effective slot
    descriptions.
\end{specargs}
%
\result%
Generic function.
%
\remarks%
The default method returns a new generic function of one argument
without any methods. Its domain is {\em class}.

\generic{compute-slot-writer}
%
\begin{genericargs}
    \item[class, \classref{class}] Class.
    \item[slot, \classref{slot}] Slot description.
    \item[slots, \classref{list}] List of effective slot
    descriptions.
\end{genericargs}
%
\result%
Function.
%
\remarks%
Computes and returns a new slot writer applicable to instances of {\em
class} and any value to be stored as the new slot value corresponding
to {\em slot}. The third argument can be used in order to
compute the logical slot position.

\method{compute-slot-writer}{class}
%
\begin{specargs}
    \item[class, \classref{class}] Class.
    \item[slot, \classref{slot}] Slot description.
    \item[slots, \classref{list}] List of effective slot
    descriptions.
\end{specargs}
%
\result%
Generic function.
%
\remarks%
The default method returns a new generic function of two arguments
without any methods. Its domain is {\em class} $\times$ \classref{object}.

\generic{ensure-slot-reader}
%
\begin{genericargs}
    \item[class, \classref{class}] Class.
    \item[slot, \classref{slot}] Slot description.
    \item[slots, \classref{list}] List of effective slot
    descriptions.
    \item[reader, \classref{function}] The slot reader.
\end{genericargs}
%
\result%
Function.
%
\remarks%
Ensures {\em function} correctly fetches the value of the slot from
instances of {\em class}.

\method{ensure-slot-reader}{class}
%
\begin{specargs}
    \item[class, \classref{class}] Class.
    \item[slot, \classref{slot}] Slot description.
    \item[slots, \classref{list}] List of effective slot descriptions.
    \item[reader, \classref{generic-function}] The slot reader.
\end{specargs}
%
\result%
Generic function.
%
\remarks%
The default method checks if there is a method in the {\em
generic-function}. If not, it creates and adds a new one, otherwise it
assumes that the existing method works correctly. The domain
of the new method is {\em class} and the function is
{\tt
\begin{tabbing}
    00\=00\=00\=00\= \kill
    (method-function-lambda ((object {\em class}))\\
    \>(primitive-reader object))
\end{tabbing}
}
\genericref{compute-primitive-reader-using-slot} is called by
\genericref{ensure-slot-reader} method to compute the primitive reader used
in the function of the new created reader method.

\generic{ensure-slot-writer}
%
\begin{genericargs}
    \item[class, \classref{class}] Class.
    \item[slot, \classref{slot}] Slot description.
    \item[slots, \classref{list}] List of effective slot
    descriptions.
    \item[writer, \classref{function}] The slot writer.
\end{genericargs}
%
\result%
Function.
%
\remarks%
Ensures {\em function} correctly updates the value of the slot in
instances of {\em class}.

\method{ensure-slot-writer}{class}
%
\begin{specargs}
    \item[class, \classref{class}] Class.
    \item[slot, \classref{slot}] Slot description.
    \item[slot-list, \classref{list}] List of effective slot
    descriptions.
    \item[writer, \classref{generic-function}] The slot writer.
\end{specargs}
%
\result%
Generic function.
%
\remarks%
The default method checks if there is a method in the {\em
generic-function}. If not, creates and adds a new one, otherwise it
assumes that the existing method works correctly. The domain of
the new method is {\em class} $\times$ \classref{object} and the
function is:
%
{\tt
\begin{tabbing}
    00\=00\=00\=00\= \kill%
    (method-function-lambda ((obj {\em class})\\
    \>\>(new-value \classref{object}))\\
    \>(primitive-writer obj new-value))
\end{tabbing}
}
%
\genericref{compute-primitive-writer-using-slot} is called by
\genericref{ensure-slot-writer} method to compute the primitive writer used in
the function of the new created writer method.

\generic{compute-primitive-reader-using-slot}
%
\begin{genericargs}
    \item[slot, \classref{slot}] Slot description.
    \item[class, \classref{class}] Class.
    \item[slots, \classref{list}] List of effective slot
    descriptions.
\end{genericargs}
%
\result%
Function.
%
\remarks%
Computes and returns a function which returns a slot value when
applied to an instance of {\em class}.

\method{compute-primitive-reader-using-slot}{slot}
%
\begin{specargs}
    \item[slot, \classref{slot}] Slot description.
    \item[class, \classref{class}] Class.
    \item[slots, \classref{list}] List of effective slot
    descriptions.
\end{specargs}
%
\result%
Function.
%
\remarks%
Calls \genericref{compute-primitive-reader-using-class}.  This is the default
method.

\generic{compute-primitive-reader-using-class}
%
\begin{genericargs}
    \item[class, \classref{class}] Class.
    \item[slot, \classref{slot}] Slot description.
    \item[slots, \classref{list}] List of effective slot
    descriptions.
\end{genericargs}
%
\result%
Function.
%
\remarks%
Computes and returns a function which returns the slot value when
applied to an instance of {\em class}.

\method{compute-primitive-reader-using-class}{class}
%
\begin{specargs}
    \item[class, \classref{class}] Class.
    \item[slot, \classref{slot}] Slot description.
    \item[slots, \classref{list}] List of effective slot
    descriptions.
\end{specargs}
%
\result%
Function.
%
\remarks%
The default method returns a function of one argument.

\generic{compute-primitive-writer-using-slot}
%
\begin{genericargs}
    \item[slot, \classref{slot}] Slot description.
    \item[class, \classref{class}] Class.
    \item[slots, \classref{list}] List of effective slot
    descriptions.
\end{genericargs}
%
\result%
Function.
%
\remarks%
Computes and returns a function which stores a new slot value when
applied on an instance of {\em class} and a new value.

\method{compute-primitive-writer-using-slot}{slot}
%
\begin{specargs}
    \item[slot, \classref{slot}] Slot description.
    \item[class, \classref{class}] Class.
    \item[slots, \classref{list}] List of effective slot
    descriptions.
\end{specargs}
%
\result%
Function.
%
\remarks%
Calls \genericref{compute-primitive-writer-using-class}.  This is the default
method.

\generic{compute-primitive-writer-using-class}
%
\begin{genericargs}
    \item[class, \classref{class}] Class.
    \item[slot, \classref{slot}] Slot description.
    \item[slots, \classref{list}] List of effective slot
    descriptions.
\end{genericargs}
%
\result%
Function.
%
\remarks%
Computes and returns a function which stores the new slot value when
applied on an instance of {\em class} and new value.

\method{compute-primitive-reader-using-class}{class}
%
\begin{specargs}
    \item[class, \classref{class}] Class.
    \item[slot, \classref{slot}] Slot description.
    \item[slots, \classref{list}] List of effective slot
    descriptions.
\end{specargs}
%
\result%
Function.
%
\remarks%
The default method returns a function of two arguments.
%
\end{optDefinition}

\sclause{Method Lookup and Generic Dispatch}
%
\begin{optDefinition}
%
\generic{compute-method-lookup-function}
%
\begin{genericargs}
    \item[gf, \classref{generic-function}] A generic function.
    \item[domain, \classref{list}] A list of classes which cover the domain.
\end{genericargs}
%
\result%
A function.
%
\remarks%
Computes and returns a function which will be called at least once for
each domain to select and sort the applicable methods by the default
dispatch mechanism.  New methods may be defined for this function to
implement different method lookup strategies. Although only one method
lookup function generating method is provided by the system, each
generic function has its own specific lookup function which may vary
from generic function to generic function.

\method{compute-method-lookup-function}{generic-function}
%
\begin{specargs}
    \item[gf, \classref{generic-function}] A generic function.
    \item[domain, \classref{list}] A list of classes which cover the domain.
\end{specargs}
%
\result%
A function.
%
\remarks%
Computes and returns a function which will be called at least once for
each domain to select and sort the applicable methods by the default
dispatch mechanism.  It is not defined, whether each generic function
may have its own lookup function.

\generic{compute-discriminating-function}
%
\begin{genericargs}
    \item[gf, \classref{generic-function}] A generic function.
    \item[domain, \classref{list}] A list of classes which span the domain.
    \item[lookup-fn, \classref{function}] The method lookup function.
    \item[methods, \classref{list}] A list of methods attached to the {\em
        generic-function}.
\end{genericargs}
%
\result%
A function.
%
\remarks%
This generic function computes and returns a function which is called
whenever the generic function is called. The returned function
controls the generic dispatch.  Users may define methods on this
function for new generic function classes to implement non-default
dispatch strategies.

\method{compute-discriminating-function}{generic-function}
%
\begin{specargs}
    \item[gf, \classref{generic-function}] A generic function.
    \item[domain, \classref{list}] A list of classes which span the domain.
    \item[lookup-fn, \classref{function}] The method lookup function.
    \item[methods, \classref{list}] A list of methods attached to the {\em
        generic-function}.
\end{specargs}
%
\result%
A function.
%
\remarks%
This method computes and returns a function which is called whenever
the generic function is called.  This default method implements the
standard dispatch strategy:  The generic function's methods are sorted
using the function returned by \genericref{compute-method-lookup-function},
and the first is called as if by \functionref{call-method}, passing the others
as the list of next methods.  Note that \functionref{call-method} need not be
directly called.

\generic{add-method}
%
\begin{genericargs}
    \item[gf, \classref{generic-function}] A generic function.
    \item[method, \classref{method}] A method to be attached to the generic
    function.
\end{genericargs}
%
\result%
This generic function adds {\em method} to the generic function {\em
gf}.  This method will then be taken into account when {\em gf} is
called with appropriate arguments.  It returns the generic function
{\em gf}.  New methods may be defined on this generic function for new
generic function and method classes.
%
\remarks%
In contrast to CLOS, \genericref{add-method} does not remove a method with
the same domain as the method being added. Instead, a noncontinuable
error is signalled.

\method{add-method}{generic-function}
%
\begin{specargs}
    \item[gf, \classref{generic-function}] A generic function.
    \item[method, \classref{method}] A method to be attached.
\end{specargs}
%
\result%
The generic function.
%
\remarks%
This method checks that the domain classes of the method are subclasses of those
of the generic function, and that the method is an instance of the generic
function's method class.  If not, signals an error (condition:
\conditionref{incompatible-method-and-gf}
\indexcondition{incompatible-method-and-gf}).  It also checks if there is a
method with the same domain already attached to the generic function.  If so, a
noncontinuable error is signaled (condition: \conditionref{method-domain-clash}
\indexcondition{method-domain-clash}). If no error occurs, the method is added
to the generic function.  Depending on particular optimizations of the generic
dispatch, adding a method may cause some updating computations, {\em e.g.} by
calling compute-method-lookup-function and compute-discriminating-function.
%
\end{optDefinition}

\gdef\module{level-1}\newpage\sclause{Low Level Allocation Primitives}
%
\begin{optDefinition}
This module provides primitives which are necessary to implement new allocation
methods portably. However, they should be defined in such a way that objects
cannot be destroyed unintentionally.  In consequence it is an error to use
\functionref{primitive-class-of}, \functionref{primitive-ref} and their setters
on objects not created by \functionref{primitive-allocate}.

\function{primitive-allocate}
%
\begin{arguments}
    \item[class] A class.
    \item[size] An integer.
\end{arguments}
%
\result%
An instance of the first argument.
%
\remarks%
This function returns a new instance of the first argument which has a
vector-like structure of length {\em size}. The components of the new instance
can be accessed using \functionref{primitive-ref} and updated using
\setterref{primitive-ref}.  It is intended to be used in new
\genericref{allocate} methods defined for new metaclasses.

\function{primitive-class-of}
%
\begin{arguments}
    \item[object] An object created by \functionref{primitive-allocate}.
\end{arguments}
%
\result%
A class.
%
\remarks%
This function returns the class of an object. It is similar to
\functionref{class-of}, which has a defined behaviour on any object. It is an
error to use \functionref{primitive-class-of} on objects which were not created
by \functionref{primitive-allocate}.

\setter{primitive-class-of}
%
\begin{arguments}
    \item[object] An object created by \functionref{primitive-allocate}.
    \item[class] A class.
\end{arguments}
%
\result%
The {\em class}.
%
\remarks%
This function supports portable implementations of
\begin{enumerate}
    \item dynamic classification like {\tt change-class} in CLOS.
    \item automatic instance updating of redefined classes.
\end{enumerate}

\function{primitive-ref}
%
\begin{arguments}
    \item[object] An object created by \functionref{primitive-allocate}.
    \item[index] The index of a component.
\end{arguments}
%
\result%
An object.
%
\remarks%
Returns the value of the objects component corresponding to the supplied index.
It is an error if {\em index} is outside the index range of {\em object}.  This
function is intended to be used when defining new kinds of accessors for new
metaclasses.

\setter{primitive-ref}
%
\begin{arguments}
    \item[object] An object created by \functionref{primitive-allocate}.
    \item[index] The index of a component.
    \item[value] The new value, which can be any object.
\end{arguments}
%
\result%
The new value.
%
\remarks%
Stores and returns the new value as the objects component corresponding to the
supplied index.  It is an error if {\em index} is outside the index range of
{\em object}.  This function is intended to be used when defining new kinds of
accessors for new metaclasses.

\end{optDefinition}

\gdef\module{level-1}\newpage\defModule{dynamic}{Dynamic Binding}
%
\begin{optDefinition}
%
The name of this module is {\tt dynamic}.
%
\specop{dynamic}
%
\Syntax
\defSyntax{dynamic}{
\begin{syntax}
    \scdef{dynamic-form}: \ra{} \classref{object} \\
    \>  ( \specopref{dynamic} \scref{identifier} )
\end{syntax}}%
\showSyntaxBox{dynamic}
%
\begin{arguments}
    \item[identifier] A symbol naming a dynamic binding.
\end{arguments}
%
\result%
The value of closest dynamic binding of \scref{identifier} is returned.  If no
visible binding exists, an error is signaled (condition:
\conditionref{unbound-dynamic-variable}
\indexcondition{unbound-dynamic-variable}).

\specop{dynamic-setq}
%
\Syntax
\defSyntax{dynamic-setq}{
\begin{syntax}
    \scdef{dynamic-setq-form}: \ra{} \classref{object} \\
    \>  ( \specopref{dynamic-setq} \scref{identifier} \scref{form} )
\end{syntax}}%
\showSyntaxBox{dynamic-setq}
%
\begin{arguments}
    \item[identifier] A symbol naming a dynamic binding to be updated.

    \item[form] An expression whose value will be stored in the dynamic binding
    of \scref{identifier}.
\end{arguments}
%
\result%
The value of \scref{form}.
%
\remarks%
The \scref{form} is evaluated and the result is stored in the closest dynamic
binding of \scref{identifier}.  If no visible binding exists, an error is
signalled (condition: \conditionref{unbound-dynamic-variable}
\indexcondition{unbound-dynamic-variable}).

\condition{unbound-dynamic-variable}{general-condition}
%
\begin{initoptions}
    \item[symbol, symbol] A symbol naming the (unbound) dynamic variable.
\end{initoptions}
%
\remarks%
Signalled by \specopref{dynamic} or \specopref{dynamic-setq} if the given
dynamic variable has no visible dynamic binding.

\specop{dynamic-let}
%
\Syntax
\defSyntax{dynamic-let}{
\begin{syntax}
    \scdef{dynamic-let-form}: \ra{} \classref{object} \\
    \>  ( \specopref{dynamic-let} \scseqref{binding} \\
    \>\>  \scref{body} )
\end{syntax}}%
\showSyntaxBox{dynamic-let}
%
\begin{arguments}
    \item[binding\/$^*$] A list of binding specifiers.

    \item[body] A sequence of forms.
\end{arguments}
%
\result%
The sequence of \scref{form}s is evaluated in order, returning the value of the
last one as the result of the \specopref{dynamic-let} form.
%
\remarks%
A binding specifier is either an identifier or a two element list of an
identifier and an initializing form.  All the initializing forms are evaluated
from left to right in the current environment and the new bindings for the
symbols named by the identifiers are created in the dynamic environment to hold
the results.  These bindings have dynamic scope and dynamic extent
\index{general}{scope and extent!of \specopref{dynamic-let} bindings}.  Each
form in \scref{body} is evaluated in order in the environment extended by the
above bindings.  The result of evaluating the last form in \scref{body} is
returned as the result of \specopref{dynamic-let}.

\defop{defglobal}
%
\Syntax
\defSyntax{defglobal}{
\begin{syntax}
    \scdef{defglobal-form}: \ra{} \classref{object} \\
    \> ( \defopref{defglobal} \scref{identifier} \scref{level-1-form} )
\end{syntax}}%
\showSyntaxBox{defglobal}
%
\begin{arguments}
    \item[identifier] A symbol naming a top dynamic binding containing the value
    of \scref{form}.

    \item[form] The \scref{form} whose value will be stored in the top dynamic
    binding of \scref{identifier}.
\end{arguments}
%
\remarks%
The value of \scref{form} is stored as the top dynamic value of the symbol named
by \scref{identifier} \index{general}{binding!top dynamic}.  The binding created
by \defopref{defglobal} is mutable.  An error is signaled (condition:
\conditionref{dynamic-multiply-defined}
\indexcondition{dynamic-multiply-defined}), on processing this form more than
once for the same \scref{identifier}.
%
\begin{note}
    The problems engendered by cross-module reference necessitated by a single
    top-dynamic environment are leading to a reconsideration of the defined
    model.  Another unpleasant aspect of the current model is that it is not
    clear how to address the issue of importing (or hiding) dynamic
    variables---they are in every sense global, which conflicts with the
    principle of module abstraction.  A model, in which a separate top-dynamic
    environment is associated with each module is under consideration for a
    later version of the definition.
\end{note}

\condition{dynamic-multiply-defined}{general-condition}
%
\begin{initoptions}
    \item[symbol, symbol] A symbol naming the dynamic variable which has already
    been defined.
\end{initoptions}
%
\remarks%
Signalled by \defopref{defglobal} if the named dynamic variable already
exists.
\end{optDefinition}

\gdef\module{level-1}\newpage\defModule{exit-1}{Exit Extensions}
%
\begin{optPrivate}
GN would like arbitrary tags for \specopref{catch} and \specopref{throw}---not
just symbols.
\end{optPrivate}
%
\begin{optDefinition}
%
The name of this module is {\tt exit-1}.
%
\specop{catch}
%
\Syntax
\defSyntax{catch}{
    \begin{syntax}
        \scdef{catch-form}: \ra{} \classref{object} \\
        \>  ( \specopref{catch} \scref{tag} \scref{body} ) \\
        \scdef{tag}: \\
        \>  \scref{symbol}
    \end{syntax}}%
\showSyntaxBox{catch}
%
\remarks%
The \specopref{catch} operator is similar to \specopref{block}, except that the
scope of the name (\scref{tag}) of the exit function is dynamic.  The catch
\scref{tag} must be a \syntaxref{symbol} because it is used as a dynamic
variable to create a dynamically scoped binding of \scref{tag} to the
continuation of the \specopref{catch} form.  The continuation can be invoked
anywhere within the dynamic extent of the \specopref{catch} form by using
\specopref{throw}.  The \scref{form}s are evaluated in sequence and the value of
the last one is returned as the value of the \specopref{catch} form.
%
\rewriterules
%
\begin{RewriteTable}{catch}{lll}
    (\specopref{catch}) & \rewrite &
    {\rm Is an error}\\
    (\specopref{catch} \scref{tag}) & \rewrite &
    (\specopref{progn} \scref{tag} ())\\
    (\specopref{catch} \scref{tag} \scref{body}) & \rewrite &
    \begin{minipage}[t]{0.3\columnwidth}
        \begin{tabbing}
            00\=00\= \kill
            (\specopref{let/cc} tmp\\
            \>(\specopref{dynamic-let} ((\scref{tag} tmp))\\
            \>\>\scref{body}))
        \end{tabbing}
    \end{minipage}
\end{RewriteTable}

Exiting from a \specopref{catch}, by whatever means, causes the restoration of
the lexical environment and dynamic environment that existed before the
\specopref{catch} was entered.  The above rewrite for \specopref{catch}, causes
the variable {\tt tmp} to be shadowed.  This is an artifact of the above
presentation only and a conforming processor must not shadow any variables that
could occur in the body of \specopref{catch} in this way.
%
\seealso%
\specopref{throw}.

\specop{throw}
%
\Syntax
\defSyntax{throw}{
    \begin{syntax}
        \scdef{throw-form}: \ra{} \classref{object} \\
        \>  ( \specopref{throw} \scref{tag} \scref{body} )
    \end{syntax}}%
\showSyntaxBox{throw}
%
\remarks%
In \specopref{throw}, the \scref{tag} names the continuation of the
\specopref{catch} from which to return.  \specopref{throw} is the invocation of
the continuation\index{general}{continuation} of the catch named \scref{tag}.
The \scref{body} is evaluated and the value are returned as the value of the
catch named by \scref{tag}.  The \scref{tag} is a symbol because it used to
access the current dynamic binding of the symbol, which is where the
continuation is bound.
%
\rewriterules
%
\begin{RewriteTable}{throw}{lll}
    (\specopref{throw}) & \rewrite &
    {\rm Is an error}\\
    (\specopref{throw} \scref{tag}) & \rewrite &
    ((\specopref{dynamic} \scref{tag}) ())\\
    (\specopref{throw} \scref{tag} \scref{form}) & \rewrite &
    ((\specopref{dynamic} \scref{tag}) \scref{form})
\end{RewriteTable}

\seealso%
\specopref{catch}.
%
\end{optDefinition}

\gdef\module{level-1}\newpage\sclause{Syntax of Level-1 objects}
\label{object-1-syntax-summary}
\index{general}{syntax-1}
\index{general}{object!syntax-1}
\begin{optDefinition}
\raggedbottom
%
This section gives the syntax of all level-1 forms:

\showSyntax{forms}

Any productions undefined here appear elsewhere in the definition, specifically:
the syntax of most expressions and definitions is given in the section defining
level-0.

\ssclause{Syntax of Level-1 modules}
%
\showSyntax{defmodule-1}

\ssclause{Syntax of Level-1 defining forms}
%
\showSyntax{defclass-1}
\showSyntax{defgeneric-1}
\showSyntax{defmethod-1}
\showSyntax{defglobal}

\ssclause{Syntax of Level-1 special forms}
%
\showSyntax{dynamic}
\showSyntax{dynamic-setq}
\showSyntax{dynamic-let}
\showSyntax{generic-1-lambda}
\showSyntax{method-lambda}
\showSyntax{catch}
\showSyntax{throw}
%
\flushbottom
%
\end{optDefinition}


% -----------------------------------------------------------------------
%%% Bibliography
\begin{optDefinition}
\bibannex
\begin{references}

\reference
    {Alberga, C.N., Bosman-Clark, C., Mikelsons, M., Van Deusen, M.,
     \& Padget, J.A.,}
    {Experience with an Uncommon LISP,}
    {Proceedings of 1986 ACM Symposium on LISP and Functional Programming, ACM,
     New York, 1986 (also available as IBM Research Report RC-11888).}
    \label{lisp/vm}

\reference
    {Berrington N., Deroure D. \& Padget J.A.,}
    {Guaranteeing Unpredictability,}
    {in preparation.}
    \label{lisp-processes}

\reference
    {Bobrow D.G., DiMichiel L.G., Gabriel R.P., Keene S.E, Kiczales G.
     \& Moon D.A,}
    {Common Lisp Object System Specification,}
    {SIGPLAN Notices, Vol. 23, September 1988.}
    \label{clos}

\reference
    {Chailloux J., Devin M. \& Hullot J-M.,}
    {LELISP: A Portable and Efficient Lisp System,}
    {Proceedings of 1984 ACM Symposium on Lisp and Functional Programming,
     Austin, Texas, pp113-122, published by ACM Press, New York.}
    \label{le-lisp}

\reference
    {Chailloux J., Devin M., Dupont F., Hullot J-M., Serpette B.,
     \& Vuillemin J.,}
    {Le-Lisp de l'INRIA, Version 15.2, Manuel de r\'ef\'erence,}
    {INRIA, Rocquencourt, May 1987.}
\label{lelisp-manual}

\reference
    {Clinger W. \& Rees J.A. (eds.),}
    {The Revised\/$^3$ Report on Scheme,}
    {SIGPLAN Notices, Vol. 21, No. 12, 1986.}
    \label{scheme-3}

\reference
    {Cointe P.,}
    {Metaclasses are First Class: the ObjVlisp model,}
    {Proceedings of OOPSLA '87, published as SIGPLAN Notices, Vol 22,
     No 12 pp156-167.}
    \label{objvlisp}

\reference
    {Fitch J.P. \& Norman A.C.,}
    {Implementing Lisp in a High-Level Language,}
    {Software Practice and Experience, Vol 7, pp713-725.}
    \label{cambridge-lisp}

\reference
    {Friedman D. \& Haynes C.,}
    {Constraining Control,}
    {Proceedings of 11th Annual ACM
     Symposium on Principles of Programming Languages, pp245-254,
     published by ACM Press, New York, 1985.}
    \label{constraining}

\reference
    {Hudak P. \& Wadler P., (eds.)}
    {Report on the Functional Programming Language Haskell,}
    {Yale University, Department of Computer Science, Research
     Report YALEU/DCS/RR-666, December 1988.}
    \label{haskell}

\reference
    {Landin P.J.,}
    {The Next 700 Programming Languages,}
    {Communications of the ACM, Vol 9, No 3., 1966, pp156-166.}
    \label{iswim}

\reference
    {Lang K.J. \& Pearlmutter B.A.,}
    {Oaklisp: An Object-Oriented Dialect of Scheme,}
    {Lisp and Symbolic Computation, Vol. 1, No. 1, June 1988, pp39-51,
     published by Kluwer Academic Publishers, Boston.}
    \label{oaklisp}

\reference
    {MacQueen D., {\em et al,}}
    {Modules for Standard ML,}
    {Proceedings of 1984 ACM Symposium on Lisp and Functional Programming,
     Austin, Texas, pp198-207, published by ACM Press, New York.}
    \label{ML-modules}

\reference
    {Milner R., {\em et al,}}
    {Standard ML,}
    {Laboratory for the Foundations of Computer Science,
     University of Edinburgh, Technical Report.}
    \label{std-ml}

\reference
    {Padget J.A., {\em et al,}}
    {Desiderata for the Standardisation of Lisp,}
    {Proceedings of 1986 ACM Conference on Lisp and Functional Programming,
     pp54-66, published by ACM Press, New York, 1986.}
    \label{desiderata}

\reference
    {Pitman K.M.,}
    {An Error System for Common Lisp,}
    {ISO WG16 paper N24.}
    \label{Pit-errors}

\reference
    {Rees J.A.,}
    {The T Manual,}
    {Yale University Technical Report, 1986.}
    \label{t-manual}

\reference
    {Slade S.,}
    {The T Programming Language, a Dialect of Lisp,}
    {Prentice-Hall 1987.}
    \label{t-book}

\reference
    {Shalit A.,}
    {Dylan, an object-oriented dynamic language,}
    {Apple Computer Inc., 1992.}
    \label{dylan}

\reference
    {Steele G.L. Jr.,}
    {Common Lisp the Language,}
    {Digital Press, 1984.}
    \label{CLtl}

\reference
    {Steele G.L. Jr.,}
    {Common Lisp the Language (second edition),}
    {Digital Press, 1990.}
    \label{CLtlt}

\reference
    {Stoyan H. {\em et al,}}
    {Towards a Lisp Standard,}
    {published in the Proceedings of the 1986 European Conference on Artificial
     Intelligence.}
    \label{stoyan}

\reference
    {Teitelman W.,}
    {The Interlisp Reference Manual,}
    {Xerox Palo Alto Research Center, 1978.}
    \label{interlisp}

\reference
    {Bretthauer, H. and Kopp, J.,}
    {The Meta-Class-System MCS. A Portable Object System for Common Lisp.
     -Documentation-.}
    {Arbeitspapiere der GMD 554, Gesellschaft fur Mathematik und
     Datenverarbeitung (GMD), Sankt Augustin (July 1991).}
    \label{mcs}

\end{references}
\end{optDefinition}


% -----------------------------------------------------------------------
%%% Indexes
\printindex{module}{Module Index}
\printindex{class}{Class Index}
\printindex{special}{Special Forms Index}
\printindex{function}{Function Index}
\printindex{generic}{Generic Function Index}
\printindex{condition}{Condition Index}
\printindex{constant}{Constant Index}
\printindex{general}{Index}

% -----------------------------------------------------------------------
\end{document}
% -----------------------------------------------------------------------
