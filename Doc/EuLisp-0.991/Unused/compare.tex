\sclause{Comparison}
\label{compare}
\index{general}{level-0 modules!compare}
\index{general}{compare!module}
\begin{optPrivate}
Move comparing, copying, class-of, etc to here

The definition of {\tt =} is generally badly thought out.

What is the relationship of {\tt equal} and its methods to {\tt eq}
and to symbols?

Is copy of integer, float and character the identity function?
Depends on the implementation/representation.

How are we to handle multiple character sets in {\tt equal}?
\end{optPrivate}
\begin{optDefinition}
The defined name of this module is {\tt compare}.
There are four functions for comparing objects for equality, of which
{\tt =} is specifically for comparing numeric values and {\tt eq},
{\tt eql} and {\tt equal} are for all objects.  The latter three are
related in the following way:

\begin{center}
\begin{tabular}{rcccl}
{\tt (eq {\em a} {\em b})} & $\Rightarrow$ & {\tt (eql {\em a} {\em
b})} & $\Rightarrow$ & {\tt (equal {\em a} {\em b})}\\
{\tt (eq {\em a} {\em b})} & $\not\Leftarrow$ & {\tt (eql {\em a} {\em
b})} & $\not\Leftarrow$ & {\tt (equal {\em a} {\em b})}\\
\end{tabular}
\end{center}

There is one function for comparing objects by order, which is called
{\tt <}, and which is implemented by the generic function {\tt
binary<}.  A summary of the comparison functions and the classes for
which they have defined behaviour is given below.

\framebox[\linewidth]
{
\begin{tabular}{|ll|}\hline
{\tt eq}: & \objectclass$\times$\objectclass\\\hline
{\tt eql}: & \objectclass$\times$\objectclass\\
           & \characterclass$\times$\characterclass \Ra {\tt equal}\\
           & \fpintclass$\times$\fpintclass \Ra {\tt binary=}\\
           & \doublefloatclass$\times$\doublefloatclass \Ra {\tt binary=}\\\hline
{\tt equal}: & \objectclass$\times$\objectclass\\
             & \characterclass$\times$\characterclass\\
             & \nullclass$\times$\nullclass\\
             & \numberclass$\times$\numberclass \Ra {\tt eql}\\
             & \pairclass$\times$\pairclass\\
             & \stringclass$\times$\stringclass\\
             & \vectorclass$\times$\vectorclass\\\hline
{\tt =}: & \numberclass$\times$\numberclass \Ra {\tt binary=}\\\hline
{\tt binary=}: & \fpintclass$\times$\fpintclass\\
               & \doublefloatclass$\times$\doublefloatclass\\\hline
{\tt <}: & \objectclass$\times$\objectclass \Ra {\tt binary<}\\\hline
{\tt binary<}: & \characterclass$\times$\characterclass\\
               & \fpintclass$\times$\fpintclass\\
               & \doublefloatclass$\times$\doublefloatclass\\
               & \stringclass$\times$\stringclass\\\hline
\end{tabular}
}

\function{eq}

\begin{arguments}
\item[object$_1$] An object.
\item[object$_2$] An object.
\end{arguments}

\result%
Compares {\em object$_1$} and {\em object$_2$} and returns {\tt t} if
they are the {\em same\/} object, otherwise {\tt ()}.  {\em Same\/} in
this context means ``identifies the same memory location''.

\remarks%
In the case of numbers and characters the behaviour of {\tt eq} might
differ between processors because of implementation choices about
internal representations.  Therefore, {\tt eq} might return {\tt t} or
{\tt ()} for numbers which are {\tt =} and similarly for characters
which are {\tt eql}, depending on the
implementation\ttindex{eq!implementation-defined
behaviour}\index{general}{implementation-defined!behaviour of {\tt eq}}.

\examples
\begin{tabular}{lcl}
\verb+(eq 'a 'a)+ & \Ra & \verb+t+\\
\verb+(eq 'a 'b)+ & \Ra & \verb+()+\\
\verb+(eq 3 3)+ & \Ra & \verb+t+ or \verb+()+\\
\verb+(eq 3 3.0)+ & \Ra & \verb+()+\\
\verb+(eq 3.0 3.0)+ & \Ra & \verb+t+ or \verb+()+\\
\verb+(eq (cons 'a 'b) (cons 'a 'c))+ & \Ra & \verb+()+\\
\verb+(eq (cons 'a 'b) (cons 'a 'b))+ & \Ra & \verb+()+\\
\verb+(eq '(a . b) '(a . b))+ & \Ra & \verb+t+ or \verb+()+\\
\verb+(let ((x (cons 'a 'b))) (eq x x))+ & \Ra & \verb+t+\\
\verb+(let ((x '(a . b))) (eq x x))+ & \Ra & \verb+t+\\
\verb+(eq #\a #\a)+ & \Ra & \verb+t+ or \verb+()+\\
\verb+(eq "string" "string")+ & \Ra & \verb+t+ or \verb+()+\\
\verb+(eq #('a 'b) #('a 'b))+ & \Ra & \verb+t+ or \verb+()+\\
\verb+(let ((x #('a 'b))) (eq x x))+ & \Ra & \verb+t+\\
\end{tabular}

\function{eql}

\begin{arguments}
\item[object$_1$] An object.
\item[object$_2$] An object.
\end{arguments}

\result%
If the class of {\em object$_1$} and of {\em object$_2$} is the same and is
a subclass of {\tt number}, the result is that of comparing them under
{\tt =}.  If the class of {\em object$_1$} and of {\em object$_2$} is the
same and is a subclass of {\tt character}, the result is that of
comparing them under {\tt equal}.  Otherwise the result is that of
comparing them under {\tt eq}.

\examples
Given the same set of examples as for {\tt eq}, the same result is
obtained except in the following cases:

\begin{tabular}{lcl}
%\verb+(eql 'a 'a)+ & \Ra & \verb+t+\\
%\verb+(eql 'a 'b)+ & \Ra & \verb+()+\\
\verb+(eql 3 3)+ & \Ra & \verb+t+\\
%\verb+(eql 3 3.0)+ & \Ra & \verb+()+\\
\verb+(eql 3.0 3.0)+ & \Ra & \verb+t+\\
%\verb+(eql (cons 'a 'b) (cons 'a 'c))+ & \Ra & \verb+()+\\
%\verb+(eql (cons 'a 'b) (cons 'a 'b))+ & \Ra & \verb+()+\\
%\verb+(eql '(a . b) '(a . b))+ & \Ra & \verb+t+ or \verb+()\\
%\verb+(let ((x (cons 'a 'b))) (eql x x))+ & \Ra & \verb+t+\\
%\verb+(let ((x '(a . b))) (eql x x))+ & \Ra & \verb+t+\\
\verb+(eql #\a #\a)+ & \Ra & \verb+t+\\
%\verb+(eql "string" "string")+ & \Ra & \verb+t+ or \verb+()+\\
%\verb+(eql #('a 'b) #('a 'b))+ & \Ra & \verb+t+ or \verb+()+\\
%\verb+(let ((x #('a 'b))) (eql x x))+ & \Ra & \verb+t+\\
\end{tabular}

\generic{equal}

\begin{arguments}
\item[object$_1$] An object.
\item[object$_2$] An object.
\end{arguments}

\result%
Returns true or false according to the method for the class(es) of
{\em object$_1$} and {\em object$_2$}. It is an error if either or
both of the arguments is self-referential.
% It is implementation-defined whether or not {\tt equal}
% will terminate on self-referential
% structures\index{general}{implementation-defined!behaviour of {\tt equal}}.

\seealso%
Class specific methods on {\tt equal} are defined for characters
(\ref{character}), lists (\ref{list}), numbers (\ref{number}), strings
(\ref{string}) and vectors (\ref{vector}).  All other cases are
handled by the default method.

\method{equal}

\begin{specargs}
\item[object$_1$, \objectclass] An object.
\item[object$_2$, \objectclass] An object.
\end{specargs}

\result%
The result is as if {\tt eql} had been called with the arguments
supplied.

\remarks%
Note that in the case of this method being invoked from {\tt equal},
the arguments cannot be characters or numbers.

\function{=}

\begin{arguments}
\item[{number$_1$ \ldots}] A non-empty sequence of numbers.
\end{arguments}

\result%
Given one argument the result is true.  Given more than one argument
the result is determined by {\tt binary=}, returning true if all the
arguments are the same, otherwise {\tt ()}.

\generic{binary=}

\begin{genericargs}
\item[number$_1$, \numberclass] A number.
\item[number$_2$, \numberclass] A number.
\end{genericargs}

\result%
One of the arguments, or {\tt ()}.

\remarks%
The result is either a number or {\tt ()}.  This is determined by
whichever class specific method is most applicable for the supplied
arguments.

\seealso%
Class specific methods on {\tt binary=} are defined for fixed
precision integer (\ref{spint}) and double float (\ref{double-float}).

%\method{binary=}
%
%\begin{specargs}
%\item[obj$_1$, \objectclass] An object.
%\item[obj$_2$, \objectclass] An object.
%\end{specargs}
%
%\result%
%This is the default method for {\tt biary=}.  The result is always
%{\tt ()}.

\function{<}

\begin{arguments}
\item[object$_1$ \ldots] A non-empty sequence of objects.
\end{arguments}

\result%
Given one argument the result is true.  Given more than one argument
the result is true if the sequence of objects {\em object$_1$} up to
{\em object$_n$} is strictly increasing according to the generic
function {\tt binary<}.  Otherwise, the result is {\tt ()}.

\generic{binary<}

\begin{genericargs}
\item[object$_1$, \objectclass] An object.
\item[object$_2$, \objectclass] An object.
\end{genericargs}

\result%
The first argument if it is less than the second, according to the
method for the class of the arguments, otherwise {\tt ()}.

\seealso%
Class specific methods on {\tt binary<} are defined for characters
(\ref{character}), strings (\ref{string}), fixed precision integers
(\ref{spint}) and double floats (\ref{double-float}).

\function{max}

\begin{arguments}

\item[object$_1$ \ldots] A non-empty sequence of objects.

\end{arguments}

\result%
The maximal element of the sequence of objects {\em object$_1$} up to
{\em object$_n$} using the generic function {\tt binary<}.  Zero
arguments is an error.  One argument returns {\em object$_1$}.

\function{min}

\begin{arguments}

\item[object$_1$ \ldots] A non-empty sequence of objects.

\end{arguments}

\result%
The minimal element of the sequence of objects {\em object$_1$} up to
{\em object$_n$} using the generic function {\tt binary<}.  Zero
arguments is an error.  One argument returns {\em object$_1$}.

\end{optDefinition}
