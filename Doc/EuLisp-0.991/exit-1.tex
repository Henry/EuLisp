\defModule{exit-1}{Exit Extensions}
%
\begin{optPrivate}
GN would like arbitrary tags for \specopref{catch} and \specopref{throw}---not
just symbols.
\end{optPrivate}
%
\begin{optDefinition}
%
The name of this module is {\tt exit-1}.
%
\specop{catch}
%
\Syntax
\defSyntax{catch}{
    \begin{syntax}
        \scdef{catch-form}: \ra{} \classref{object} \\
        \>  ( \specopref{catch} \scref{tag} \scref{body} ) \\
        \scdef{tag}: \\
        \>  \scref{symbol}
    \end{syntax}}%
\showSyntaxBox{catch}
%
\remarks%
The \specopref{catch} operator is similar to \specopref{block}, except that the
scope of the name (\scref{tag}) of the exit function is dynamic.  The catch
\scref{tag} must be a \syntaxref{symbol} because it is used as a dynamic
variable to create a dynamically scoped binding of \scref{tag} to the
continuation of the \specopref{catch} form.  The continuation can be invoked
anywhere within the dynamic extent of the \specopref{catch} form by using
\specopref{throw}.  The \scref{form}s are evaluated in sequence and the value of
the last one is returned as the value of the \specopref{catch} form.
%
\rewriterules
%
\begin{RewriteTable}{catch}{lll}
    (\specopref{catch}) & \rewrite &
    {\rm Is an error}\\
    (\specopref{catch} \scref{tag}) & \rewrite &
    (\specopref{progn} \scref{tag} ())\\
    (\specopref{catch} \scref{tag} \scref{body}) & \rewrite &
    \begin{minipage}[t]{0.3\columnwidth}
        \begin{tabbing}
            00\=00\= \kill
            (\specopref{let/cc} tmp\\
            \>(\specopref{dynamic-let} ((\scref{tag} tmp))\\
            \>\>\scref{body}))
        \end{tabbing}
    \end{minipage}
\end{RewriteTable}

Exiting from a \specopref{catch}, by whatever means, causes the restoration of
the lexical environment and dynamic environment that existed before the
\specopref{catch} was entered.  The above rewrite for \specopref{catch}, causes
the variable {\tt tmp} to be shadowed.  This is an artifact of the above
presentation only and a conforming processor must not shadow any variables that
could occur in the body of \specopref{catch} in this way.
%
\seealso%
\specopref{throw}.

\specop{throw}
%
\Syntax
\defSyntax{throw}{
    \begin{syntax}
        \scdef{throw-form}: \ra{} \classref{object} \\
        \>  ( \specopref{throw} \scref{tag} \scref{body} )
    \end{syntax}}%
\showSyntaxBox{throw}
%
\remarks%
In \specopref{throw}, the \scref{tag} names the continuation of the
\specopref{catch} from which to return.  \specopref{throw} is the invocation of
the continuation\index{general}{continuation} of the catch named \scref{tag}.
The \scref{body} is evaluated and the value are returned as the value of the
catch named by \scref{tag}.  The \scref{tag} is a symbol because it used to
access the current dynamic binding of the symbol, which is where the
continuation is bound.
%
\rewriterules
%
\begin{RewriteTable}{throw}{lll}
    (\specopref{throw}) & \rewrite &
    {\rm Is an error}\\
    (\specopref{throw} \scref{tag}) & \rewrite &
    ((\specopref{dynamic} \scref{tag}) ())\\
    (\specopref{throw} \scref{tag} \scref{form}) & \rewrite &
    ((\specopref{dynamic} \scref{tag}) \scref{form})
\end{RewriteTable}

\seealso%
\specopref{catch}.
%
\end{optDefinition}
